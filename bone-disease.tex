\chapter{Bone and spine disease and malformations}

\subsection{Congenital spinal meningocele}

\textbf{Congenital spinal meningoceles}are developmental anomalies of meningothelial elements displaced into the skin and subcutaneous tissues.

Please refer to the meningocele article for a broad overview of all types of this condition.

\paragraph{Pathology}

It is a defect of the neural tube, an embryonic structure that gives rise to the spinal cord and vertebral column. This defect leads to protrusion of the membranes that cover the spine and part of the spinal cord through a bone defect in the vertebral column.

\subparagraph{Types}

\begin{itemize}
	\tightlist
	\item
	dorsal meningocele
	\item
	anterior meningocele
	\item
	lateral meningocele
	
	\begin{itemize}
		\tightlist
		\item
		lateral sacral meningocele
		\item
		lateral lumbar meningocele
	\end{itemize}
\end{itemize}

\paragraph{Differential diagnosis}

Possible considerations include

\begin{itemize}
	\tightlist
	\item
	pseudomeningocele- post traumatic
	\item
	myelomeningocele
	\item
	lipomyelomeningocele
\end{itemize}

\subsection{Progressive postnatal pansynostosis}

\textbf{Progressive postnatal pansynostosis} (\textbf{PPP}) is a rare form of craniosynostosis characterized by the late fusion of all cranial sutures.

\paragraph{Epidemiology}

This type of craniosynostosis occurs insidiously after birth and presents later in life unlike other types of craniosynostosis which occur during the prenatal period. Most patients have an associated syndrome; Crouzon syndrome is the most common .

\paragraph{Clinical presentation}

The late fusion of cranial sutures means that the diagnosis is often delayed and typically presents signs of increased intracranial pressure with relatively normal, albeit smaller than average, head size .

\paragraph{Radiographic features}

CT is the imaging modality of choice. It shows:

\begin{itemize}
	\item
	signs of increased intracranial pressure with effacement of the ventricles, basal cisterns, and other CSF spaces
	\item
	bone window shows copper beaten skull, characteristic endocortical scalloping
	\item
	signs of venous hypertension like widened occipital mastoid emissary foramina and prominent subgaleal veins
	\item
	3D volume rendered bone window shows fusion of all major cranial sutures (pansynostosis)
\end{itemize}

\begin{tcolorbox}[colback=green!5!white,colframe=green!75!white,title=Differential diagnosis]
Includes other causes of raised intracranial pressure which may cause the copper beaten skull appearance.

\begin{itemize}
	\item
	other types of craniosynostosis
	\item
	obstructive hydrocephalus
	\item
	intracranial masses
\end{itemize}

Hypophosphatasia: pansynostosis occurs in patients with hypophosphatasia associated with other bony features of hypophosphatasia.
\end{tcolorbox}

\subsection{Saethre-Chotzen syndrome}

\textbf{Saethre-Chotzen} \textbf{syndrome} (also known as \textbf{type III acrocephalosyndactyly}) is characterized by limb and skull abnormalities.

\paragraph{Epidemiology}

It is the most common craniosynostosis syndrome and affects 1:25 - 50,000 individuals.

Males and females are equally affected.

\paragraph{Clinical presentation}

The spectrum of observed clinical features include

\begin{itemize}
	\tightlist
	\item
	craniosynostosis: typically coronal
	\item
	syndactyly: syndactyly of digits two and three of the hand is variably present
	\item
	hypertelorism
	\item
	ptosis
	\item
	strabismus
	\item
	characteristic appearance of ears (small pinna with a prominent superior and/or inferior crus)
\end{itemize}

\paragraph{Pathology}

\subparagraph{Genetics}

It is thought to be due to mutations in the \emph{TWIST1}(twist transcription factor 1) gene located on chromosome 7p21 .The condition is inherited in an autosomal dominant pattern.

\paragraph{Differential diagnosis}

Consider other forms of acrocephalosyndactyly such as

\begin{itemize}
	\tightlist
	\item
	Crouzon syndrome
	\item
	Apert syndrome
	\item
	Pfeiffer syndrome
\end{itemize}