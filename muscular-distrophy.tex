\chapter{Muscular distrophies}

\section{Muscular dystrophy}

\textbf{Muscular dystrophies} refer to a broad group of conditions that result in increasing weakening and breakdown of skeletal musculature over time.

These include

\begin{itemize}
	\tightlist
	\item
	Duchenne muscular dystrophy (considered most common)
	\item
	Becker muscular dystrophy
	\item
	facioscapulohumeral muscular dystrophy
	\item
	congenital muscular dystrophies (CMD)
	
	\begin{itemize}
		\tightlist
		\item
		CMD 1
		\item
		CMD 2:Fukuyama congenital muscular dystrophy (FCMD)
		\item
		CMD 3:Santavuori muscle-eye-brain (MEB) Finnish-type
		\item
		CMD4:Walker-Warburg syndrome
	\end{itemize}
	\item
	Emery-Dreiffus muscular dystrophy
	\item
	limb-girdle muscular dystrophy (LGMD)
	\item
	distal muscular dystrophy
	\item
	myotonic muscular dystrophy
	
	\begin{itemize}
		\tightlist
		\item
		myotonic dystrophy type 1 (DM1) (Steinert disease)
		\item
		myotonic dystrophy type 2(DM2) (proximal myotonic myopathy)
	\end{itemize}
	\item
	oculopharyngeal muscular dystrophy
\end{itemize}

\section{Emery-Dreiffus muscular dystrophy}

\textbf{Emery-Dreifuss muscular dystrophy} is a rare form of muscular dystrophy characterized by childhood onset of contractures, humeroperoneal muscle atrophy, and cardiac conduction abnormalities.

\paragraph{Clinical course}

Weakness is slowly progressive, but there is a broad spectrum of clinical severity.

\paragraph{Pathology}

X-linked (EDMD1) and autosomal dominant (EDMD2) forms of the disease are noted which appears clinically similar and are caused by defects of the nuclear membrane proteins named emerin and lamins A/C, respectively.

\section{Oculopharyngeal muscular dystrophy}

\textbf{Oculopharyngeal muscular dystrophy} is rare form of muscular dystrophy characterized by ptosis and swallowing difficulties due to selective involvement of the muscles of the eyelid and pharynx. It can also affect other muscles such as the soleus and adductor magnus .

\paragraph{Pathology}

It is thought to be caused by an abnormal expansion of GCN triplets within the \emph{PABPN1}gene.