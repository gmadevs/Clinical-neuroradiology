\chapter{Posterior fossa malformations}

\subsection{Lemon sign}

The \textbf{lemon sign}, noted on antenatal imaging, is one of the many notable fruit-inspired signs.It is a feature when there appears to be an indentation of the frontal bone (depicting that of a lemon). It is classically seen as a sign of a Chiari II malformationand also seen in the majority (90-98\%) of fetuses with spina bifida.

\paragraph{Pathology}

\subparagraph{Associations}

The following conditions are associated with the lemon sign:

\begin{itemize}
	\tightlist
	\item
	Chiari II malformation
	\item
	spina bifida
	\item
	encephalocele
	\item
	Dandy Walker malformation
	\item
	thanatophoric dysplasia
	\item
	cystic hygroma
	\item
	diaphragmatic hernia
	\item
	corpus callosal agenesis
	\item
	fetal hydronephrosis 
	\item
	umbilical vein varix
\end{itemize}

It is also associated with the banana sign.

\paragraph{Radiographic assessment}

The lemon sign is seen on axial imaging (usually antenatal ultrasound, although antenatal MRI will also demonstrate this sign) through the head and relates to concavity (not just flattening) of the frontal bones.

Several diagnostic points should be remembered about this sign:

\begin{itemize}
	\tightlist
	\item
	significant anterior angulation for obtaining images of the calvaria should be avoided as fetal orbits could simulate a lemon sign
	\item
	this sign usually disappears \textgreater24 weeks, which may be due to the reduced pliability of the fetal calvaria with advancing gestational age or an increase in the intracranial pressure with associated hydrocephalus
	\item
	this sign may be rarely seen in normal patients (\textasciitilde1\% of cases)and in those with other non-neural axis abnormalities
\end{itemize}

\paragraph{Differential diagnosis}

For an abnormal head shape resembling the lemon sign, one should also consider craniosynostosis, most notably of the bicoronal (or bilateral coronal) type.