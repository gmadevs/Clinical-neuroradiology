\chapter{Tuberculosis, fungal and parasitic infections}

\subsection{Sinonasal mucormycosis}

\textbf{Sinonasal mucormycosis} refers to an uncommon form of invasive fungal sinus infection. Given its highly invasive nature, it can involve orbits and/or intracranial structures.

\paragraph{Clinical presentation}

The presentation can vary, ranging from exophthalmos, rhinorrhea, and ophthalmoplegia with loss of visual acuity and peripheral facial palsies occurring rarely .

\paragraph{Pathology}

It originates in the paranasal sinuses and can frequently invade to orbital and cerebral regions.If detected and treated early, involvement can be limited to the nasal cavity and paranasal sinuses.

It is caused by fungi of order Mucorales which can include

\begin{itemize}
	\tightlist
	\item
	\emph{Mucor}spp.
	\item
	\emph{Rhizopus}spp.
	\item
	\emph{Absidia} spp.
\end{itemize}

The fungi themselves are ubiquitous, subsisting on decaying vegetation and diverse organic material . Given the opportunity, fungal spores can invade the nasal mucosa (which are often not phagocytised due to poor immune response). They then germinate, forming angioinvasive hyphae that cause infarction of the involved tissue, giving in a ``dry'' gangrene appearance.

\subparagraph{Risk groups}

\begin{itemize}
	\tightlist
	\item
	diabetics: especially those with poor control 
	\item
	immunocompromised states
\end{itemize}

\paragraph{Radiographic features}

\subparagraph{General}

Can show varying degrees of sinus opacification with most having a tumefactive nature . They generally demonstrate a rim of soft-tissue thickness along the paranasal sinuses. Complete sinus opacification, gas-fluid levels and obliteration of the nasopharyngeal tissue planes can also occur.

\subparagraph{MRI}

Reported signal characteristics on MRI of the sinuses and brain include:

\begin{itemize}
	\tightlist
	\item
	\textbf{T1:} isointense lesions relative to the brain in most cases (\textasciitilde80\%) 
	\item
	\textbf{T2}
	
	\begin{itemize}
		\tightlist
		\item
		variable with around 20\% of patients showing high T2 signal 
		\item
		fungal elements themselves tend to have low signal on T2
	\end{itemize}
	\item
	\textbf{T1 C+ (Gd):} the devitalised mucosa appears on contrast-enhanced MR imaging as contiguous foci of non-enhancing tissue, leading to the black turbinate sign
	\item
	\textbf{DWI:} \textbf{} increased signal intensity may be seen 
\end{itemize}

\paragraph{Treatment and prognosis}

The condition in general carries high morbidity.Management options include reversal of immunosuppression, systemic amphotericin B and surgical debridement in selected cases. Untreated cases can rapidly progress and can be aggressive . Complications associated with wider intracranial extension can be potentially fatal .

\subparagraph{Complications}

\begin{itemize}
	\tightlist
	\item
	invasion:
	
	\begin{itemize}
		\tightlist
		\item
		orbital spread (rhino-orbital mucormycosis)
		\item
		intracranial extension (rhinocerebral mucormycosis)
		\item
		both orbital and intracranial extension (rhino-orbitocerebral mucormycosis)
	\end{itemize}
	\item
	vascular thrombosis (from extension):including the cavernous sinus thrombosis
	\item
	subsequent infarction
\end{itemize}