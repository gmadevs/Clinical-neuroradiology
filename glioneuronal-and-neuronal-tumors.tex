\chapter{Glioneuronal and neuronal tumors}

\subsection{Dysplastic cerebellar gangliocytoma}

\textbf{Dysplastic cerebellar gangliocytoma},perhaps better known as \textbf{Lhermitte-Duclos disease},is a rare tumor of the cerebellum appearing as thickening and increase in T2 signal of the cerebellar folia giving this lesion a characteristic striated appearance.

\paragraph{Epidemiology}

Dysplastic cerebellar gangliocytomas typically present in young adults, although they have been encountered at all ages .

\subparagraph{Associations}

A number of associated conditions have been described , including:

\begin{itemize}
	\item
	Cowden syndrome(as part of COLD syndrome, see below)
	\item
	disorders of cortical formation
	
	\begin{itemize}
		\item
		megalencephaly
		\item
		grey matter heterotopia
		\item
		polymicrogyria
	\end{itemize}
	\item
	polydactyly
	\item
	hydromyelia
	\item
	macroglossia
	\item
	localized gigantism
	\item
	leontiasis ossea
\end{itemize}

\paragraph{Clinical presentation}

Small tumors may be asymptomatic or only present with relatively subtle cerebellar signs (e.g. dysmetria). When larger, symptoms are typically related to raised intracranial pressure,obstructive hydrocephalusand to a lesser degree, cerebellar dysfunction .


\paragraph{Pathology}

Dysplastic cerebellar gangliocytomas are designated as WHO grade 1 tumors and considered one of a number of glioneuronal and neuronal tumors in the WHO classification of CNS tumors.

\subparagraph{Genetics}

Interestingly the genetics of childhood-onset appears different from the more common adult-onset form.In the adult form,\emph{PTEN mutations}are invariably found,lending additional weight to Lhermitte-Duclos disease being a manifestation of Cowden syndrome. In such cases, it is termed \textbf{COLD syndrome}(Cowden-Lhermitte-Duclos syndrome).In contrast, in children, \emph{PTEN} mutations are absent .

\subparagraph{Macroscopic appearance}

Dysplastic cerebellar gangliocytomas are usually single and unilateral, presenting as a discrete region of cerebellar hypertrophy .

\subparagraph{Microscopic appearance}

Derangement of the normal laminar cellular organization of the cerebellum is present. There is thickening of the outer molecular cell layer, loss of the middle Purkinje cell layer,and infiltration of the inner granular cell layer with dysplastic ganglion cells of various sizes .

\subparagraph{Immunophenotype}

\begin{itemize}
	\item
	synaptophysin: positive
	\item
	loss of PTEN protein expression (Cowden syndrome/COLD syndrome)
\end{itemize}

\paragraph{Radiographic features}

The abnormal tissue involves the cerebellar cortex and is usually confined to one hemisphere, occasionally extending to the vermis but only rarely extending to the contralateral hemisphere .

\subparagraph{CT}

\begin{itemize}
	\item
	may show a non-specific hypoattenuating cerebellar mass
	\item
	calcification is sometimes seen 
\end{itemize}

\subparagraph{MRI}

Widened cerebellar folia with a striated/tigroid appearance. Also described as "corduroy/laminated" appearance .

\begin{itemize}
	\item
	\textbf{T1:}hypointense 
	\item
	\textbf{T2:}hyperintense with apparently preserved cortical striations 
	\item
	\textbf{DWI:}similar to normal cortex
	
	\begin{itemize}
		\item
		may show hyperintensity due to T2 shine-through effect
	\end{itemize}
	\item
	\textbf{T1 C+ (Gd)}
	
	\begin{itemize}
		\item
		enhancement is rare
		\item
		if present usually superficial, possibly due to vascular proliferation 
	\end{itemize}
	\item
	\textbf{MR spectroscopy}
	
	\begin{itemize}
		\item
		elevated lactate 
		\item
		slightly reduced NAA (by about 10\%) 
		\item
		reduced \emph{myo}-inositol (by 30-80\%)
		\item
		reduced choline (by 20-50\%)
		\item
		reduced Cho/Cr ratio 
	\end{itemize}
\end{itemize}


\subparagraph{PET/SPECT}

\begin{itemize}
	\item
	\textbf{FDG-PET:}increased uptake
	\item
	\textbf{Tl-201 SPECT:}increased uptake
\end{itemize}


\paragraph{Treatment and prognosis}

The dysplastic mass grows very slowly, and initial treatment revolves around treating hydrocephalus. Surgical resection is often curative, with only a few case reports of recurrence . Importantly it is crucial to remember the association with Cowden syndrome, hence, increased risk of other neoplasms such as breast, endometrial and thyroid cancers. Therefore, a recommendation for further imaging or clinical assessment of possible tumors in these locations should be included in the radiologist's report.

\paragraph{Differential diagnosis}

The appearance is very characteristic and usually little differential exists, particularly when appearances are typical.

In the setting of sepsis or acute deterioration, one should consider cerebellitisor subacute cerebellar infarction.

The appearance may be mimicked by extensively nodular medulloblastoma (SHH molecular subgroup).