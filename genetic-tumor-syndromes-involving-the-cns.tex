\chapter{Genetic tumor syndromes}

\subsection{Multiple endocrine neoplasia type 1}

\textbf{Multiple endocrine neoplasia type 1 (MEN1)},also known as \textbf{Wermer syndrome}, is an autosomal dominant genetic disease that results in proliferative lesions in multiple endocrine organs, particularly the pituitary gland, pancreas, and parathyroid glands.

There are other multiple endocrine neoplasia syndromesand these are discussed separately.

\paragraph{Epidemiology}

MEN1 has an estimated prevalence of 2 per 100,000, and there is no gender predilection .

\subparagraph{Associations}

In addition to the characteristic lesions involving the pituitary, parathyroid, and pancreas, numerous other lesions are encountered with greater frequency in patients with MEN1. These include:

\begin{itemize}
	\item
	lipomas
	\item
	angiofibromas
	\item
	adrenal corticallesions
	
	\begin{itemize}
		\item
		adrenal adenomas
		\item
		adrenocortical hyperplasia
		\item
		cortisol-secreting adenomas
		\item
		adrenal carcinomas(rare)
	\end{itemize}
	\item
	neuroendocrine tumors
	\item
	hepatic focal nodular hyperplasia
	\item
	breast carcinoma
	\item
	meningiomas
\end{itemize}

\paragraph{Clinical presentation}

Primary hyperparathyroidismis the commonest presentation, followed by pancreatic neuroendocrine tumors with associated hypersecretion syndromes; gastrinomas are most common and associated with Zollinger-Ellison syndrome .

MEN1 is an autosomal dominant syndrome characterized by :

\begin{itemize}
	\item
	pituitary adenomas
	
	\begin{itemize}
		\item
		prolactinoma(most common)
		\item
		30\% of patients
	\end{itemize}
	\item
	pancreatic neuroendocrine tumors
	
	\begin{itemize}
		\item
		gastrinoma(most common: \textgreater50\%),followed by insulinoma (4-6\%), and glucagonoma (\textless3\%) 
		\item
		50-80\% of patients
		\item
		significant cause of mortality
	\end{itemize}
	\item
	parathyroid proliferative diseases
	
	\begin{itemize}
		\item
		parathyroid hyperplasia(most common)
		
		\begin{itemize}
			\item
			hyperparathyroidism(80-95\%)
		\end{itemize}
		\item
		parathyroid adenoma
		\item
		parathyroid carcinoma(rare)
	\end{itemize}
\end{itemize}

Handy mnemonics for recalling MEN1:

\begin{itemize}
	\item
	PPP or PiParPanc
\end{itemize}


\paragraph{Pathology}

\subparagraph{Genetics}

The abnormality is related to \emph{MEN1}, a tumor suppressor gene located on chromosome 11q13 which produces menin, a nuclear protein important for the regulation of gene expression.

\paragraph{Treatment and prognosis}

Treatment is directed to each individual manifestation. These are therefore discussed separately.

Pancreatic malignancy is the leading cause of mortality in MEN1.