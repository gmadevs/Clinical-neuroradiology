\chapter{Hydrosyringomyelia}

\section{Syringocephalus}

\textbf{Syringocephalus} , also known as \textbf{syringoencephalomyelia} or \textbf{syringocephalia}, is a very rare entity and refers to a syrinx that extends into the cerebrum .

\paragraph{Clinical presentation}

Patients with this condition demonstrate a wide variety of focal neurological symptoms depending on where the syrinx is located .


\paragraph{Pathology}

When present, syringocephalus is seen in continuity with a long syrinxthat demonstrates syringomyelia,syringobulbia, syringopontia, and syringomesencephaly. Most commonly the syrinxextends into the centrum semiovale, however has less commonly been reported to extend into the basal ganglia, internal capsule,and cerebral cortex instead .

There are numerous causes and associations, and these are discussed in more depth in the general article on syrinx.


\paragraph{Radiographic features}

Syringocephalus has the same radiographic characteristics on all imaging modalities as any other syrinx . See syrinx for an in-depth discussion of these characteristics.


\paragraph{Treatment and prognosis}

When symptomatic, neurosurgical intervention may be required .

\section{Syringomesencephaly}

\textbf{Syringomesencephaly}is a very rare entity and refers to a syrinxthat extends into the midbrain.

\paragraph{Clinical presentation}

Patients with this condition demonstrate a wide variety of neurological symptoms localized to the brainstemand spinal cord,depending on where exactly the syrinx is located .


\paragraph{Pathology}

When present, it is seen in continuity with a long syrinxthat demonstrates syringomyelia,syringobulbia,and syringopontia. As with syringopontia, the syrinx may or may not have a communication with the fourth ventricle .In rare cases when it is present without syringomyelia, the term keyhole aqueduct syndrome has been employed by some authors (see syringopontia for further discussion) .

There are numerous causes and associations, and these are discussed in more depth in the general article on syrinx.


\paragraph{Radiographic features}

Syringomesencephaly has the same radiographic characteristics on all imaging modalities as any other syrinx . See syrinx for an in-depth discussion of these characteristics.


\paragraph{Treatment and prognosis}

When symptomatic, neurosurgical intervention may be required .

\section{Syringopontia}

\textbf{Syringopontia} is a rare entity and refers to a syrinxthat extends into the pons. In rare cases when syringopontia is present without syringomyelia, the term \textbf{keyhole aqueduct syndrome} has been employed by some authors .

\paragraph{Clinical presentation}

Patients with this condition demonstrate a wide variety of neurological symptoms localized to the pons, medulla oblongata, and spinal cord, depending on where exactly the syrinx is located . For example, in keyhole aqueduct syndrome, the most common clinical features include internuclear ophthalmoplegia and nystagmus .


\paragraph{Pathology}

When present, it is seen in continuity with a long syrinxthat demonstrates syringomyeliaand syringobulbia. The syrinx may or may not have a communication with the fourth ventricle .

There are numerous causes and associations, and these are discussed in more depth in the general article on syrinx.


\paragraph{Radiographic features}

Syringopontia has the same radiographic characteristics on all imaging modalities as any other syrinx . See syrinx for an in-depth discussion of these characteristics.


\paragraph{Treatment and prognosis}

When symptomatic, neurosurgical intervention may be required .