\chapter{General neurology}

\subsection{Investigating altered consciousness (summary)}

\textbf{Altered consciousness} or confusion is a common reason for admission to hospital, and involvement of radiology .

Although the most common reason for acute confusion is intoxication which will improve - when altered neurology is present or the changes persists, more serious reasons need investigated.

\paragraph{Summary}

\begin{itemize}
	\item
	\textbf{questions}
	
	\begin{itemize}
		\item
		any systemic cause for confusion?
		
		\begin{itemize}
			\item
			e.g. sepsis, hypoglycemia and drug interactions
		\end{itemize}
		\item
		any direct causes?
		
		\begin{itemize}
			\item
			e.g. trauma, pressure effects, infarction or infection
		\end{itemize}
		\item
		are there any focal neurological signs?
		\item
		is there any relevant medication history?
		
		\begin{itemize}
			\item
			e.g. anticoagulants, alcohol, narcotics?
		\end{itemize}
	\end{itemize}
	\item
	\textbf{investigations}
	
	\begin{itemize}
		\item
		CT head is the first line investigation, especially in the acute setting
		
		\begin{itemize}
			\item
			10\% of patients will have a cause found 
		\end{itemize}
		\item
		MRI can be performed, but usually only after CT
	\end{itemize}
	\item
	\textbf{making the request}
	
	\begin{itemize}
		\item
		what is the likely underlying cause of confusion?
		\item
		what is the urgency of the study?
	\end{itemize}
	\item
	\textbf{common pathology}
	
	\begin{itemize}
		\item
		systemic infection (rarely cerebral)
		\item
		dementia (Alzheimer/vascular)
		\item
		ischemic stroke
		\item
		intracranial hemorrhage, e.g. hemorrhagic stroke, subarachnoid hemorrhage, subdural hemorrhage, epidural hemorrhage
		\item
		space-occupying lesions, e.g. tumor, abscess
	\end{itemize}
\end{itemize}


\paragraph{Teaching playlist}

\begin{itemize}
	\item
	altered consciousness playlist
\end{itemize}

\subsection{Foster Kennedy syndrome}

\textbf{Foster Kennedy syndrome} describes the clinical syndrome of unilateral optic atrophy with contralateral papilledema caused by an ipsilateral compressive mass lesion.

\paragraph{Clinical presentation}

The syndrome consists of two cardinal features, in relation to a mass lesion :

\begin{enumerate}
	\item
	ipsilateral optic nerve atrophy presenting with central scotoma
	\item
	contralateral optic disc swelling due to papilledema
\end{enumerate}

Other common clinical features include :

\begin{itemize}
	\item
	ipsilateral anosmia
	\item
	headache
	\item
	nausea and vomiting
\end{itemize}

\paragraph{Pathology}

Foster Kennedy syndrome, by definition, is caused by a compressive mass . This mass directly compresses one optic nerve, accounting for ipsilateral optic nerve atrophy, and causes chronic raised intracranial pressure resulting in contralateral papilledema. Thus, in order to cause such a constellation of symptoms, masses are usually located in the olfactory groove, falx cerebri, sphenoid wing, or subfrontal region . The most commonly reported mass is a meningioma, although a number of other causes have been reported such as craniopharyngiomas, pituitary adenomas, neuroblastomas, and rarely aneurysms and frontal lobe abscesses.

The same syndromic features has also been reported to occur due to non-mass lesions or mass lesions that do not directly compress an optic nerve, and these cases are referred to as \textbf{pseudo-Foster Kennedy syndrome}. Indeed, these are considered to be more common as the etiology for this constellation of clinical features . Some causes include :

\begin{itemize}
	\item
	mass lesions that cause indirect unilateral optic nerve compression but do not directly compress that optic nerve
	\item
	bilateral sequential ischemic optic neuropathy: non-arteritic anterior ischemic optic neuropathy is more common with the new neuropathy developing in the eye with optic disc swelling
	\item
	retrobulbar optic neuritis
	\item
	chronic unilateral optic atrophy
	\item
	hypertrophic pachymeningitis
	\item
	idiopathic intracranial hypertension: bilateral papilledema is far more common
	\item
	unilateral optic nerve hypoplasia
	\item
	vitamin B\textsubscript{12 deficiency}
	\item
	neurosyphilis
\end{itemize}

Furthermore, the description \textbf{pseudo-pseudo-Foster Kennedy syndrome} has been employed in one case report . This report describes a case of Foster Kennedy syndrome alongside concurrent contralateral non-arteritic ischemic optic neuropathy such that one nerve was atrophied due to the direct compression from a meningioma and the other nerve was swollen due to both raised intracranial pressure and the non-arteritic ischemic optic neuropathy . This is likely to be a very rare coincidental entity.

Somewhat comically, \textbf{pseudo-pseudo-pseudo-Foster Kennedy syndrome}has also been suggested as a descriptor for a mass lesion that causes indirect unilateral optic nerve compression . However, this entity should properly fall under the definition of pseudo-Foster Kennedy syndrome.

\paragraph{Radiographic features}

Radiographic features vary depending on the exact cause of Foster Kennedy syndrome, but will generally show a mass lesion compressing one optic nerve resulting in features of papilledema contralaterally.


\paragraph{Treatment and prognosis}

Treatment options vary depending on the exact cause but generally 'true'Foster Kennedy syndrome requires neurosurgical intervention as part of management .

\subsection{Vitamin B6 deficiency}

\textbf{Vitamin B\textsubscript{6} deficiency}(also known as \textbf{hypovitaminosis B\textsubscript{6}}) is rare, as the B\textsubscript{6 vitamers} are present in many commonly-consumed foodstuffs. It is most commonly seen in the context of chronic ethanol excess, although many other risk factors are known. In children, deficiency may manifest as seizures. But in adults marked hypovitaminosis B\textsubscript{6} more usually presents with altered mentation and skin rashes. Other manifestations might include a normocytic anemia, angular cheilitis, and glossitis. A depressive illness may be a feature. The deficiency usually shows a good response to oral/IV administration of pyridoxine.

Spaceflight-induced cerebral changes

\textbf{Spaceflight-induced cerebral changes}, or \textbf{microgravity-induced cerebral changes},refer to the effects of prolonged microgravity exposure, through spaceflight, on the brain and surrounding structures.

\paragraph{Epidemiology}

As the name suggests, spaceflight-induced cerebral changes are limited to patients who have experienced spaceflight and microgravity, i.e. astronauts. Therefore, patients with these changes are not likely to be encountered in the vast majority of radiology departments.

\paragraph{Clinical presentation}

Numerous clinical syndromes have been described relating to space travel, and possibly the effects of microgravity, however this is an ongoing field of research:

\begin{itemize}
	\tightlist
	\item
	space adaptation syndrome (SAS) or space motion sickness (SMS)
	
	\begin{itemize}
		\tightlist
		\item
		common clinical syndrome that occurs during the first few days of spaceflights of any duration 
		\item
		the clinical features are similar to those of motion sickness experienced on Earth without spaceflight 
	\end{itemize}
	\item
	visual impairment and intracranial pressure (VIIP) syndrome or spaceflight-associated neuro-ocular syndrome (SANS)
	
	\begin{itemize}
		\tightlist
		\item
		exclusively occurs after long-duration spaceflights 
		\item
		clinical features include varying degrees of visual acuity degradation, but VIIP syndrome can also be clinically asymptomatic 
		\item
		furthermore, various ocular changes may be appreciated by neuro-ophthalmological examination such as hyperopic shift, choroidal folds, papilledema, and cotton wool spots 
		\item
		in patients with this syndrome, post-flight CSF opening pressures are often raised, and thus this syndrome has been likened in many studies to idiopathic intracranial hypertension
	\end{itemize}
	\item
	`space fog'
	
	\begin{itemize}
		\tightlist
		\item
		a vaguely defined syndrome that encompasses cognitive effects of spaceflight 
		\item
		most clearly, there are disturbances in visuo-motor tracking and dual-task performance, whereas elementary and complex cognitive functions or spatial processing are generally spared 
	\end{itemize}
	\item
	`Charlie Brown effect'
	
	\begin{itemize}
		\tightlist
		\item
		a vaguely defined colloquialism that encompasses changes to the face,possibly secondary to cephalad fluid shifts 
		\item
		symptoms reported and attributed to this effect include puffiness of the face and head, and associated changes in taste and smell 
	\end{itemize}
	\item
	changes in psychological mental state 
\end{itemize}

\paragraph{Pathology}

Pathophysiology of space-related cerebral changes remains uncertain; however the cornerstone of many theories considers the cephalad fluid shift that occurs due to microgravity . During exposure to microgravity, fluid is redistributed almost immediately to the upper body and head due to loss of hydrostatic gradients of the lower body venous system . This is likely to be the cause of facial puffiness and altered senses seen as part of the colloquial `Charlie Brown effect' due to facial fluid congestion .

Furthermore, this cephalad fluid shift is also thought to increase intracranial pressure (ICP), although the exact reasons for this are yet to be elucidated . One theory suggests that venous distension in the head, a well-documented phenomenon of fluid shift from microgravity exposure, may also cause cerebral venous congestion . This congestion may contribute to increased ICP by itself, but may further cause impairments of cerebrospinal fluid (CSF) outflow which may be a second mechanism for raising the ICP . While this theory is promising, it is yet to be proven, and alternative theories have been proposed. Regardless, it is thought that this hypothesized increased ICP may be responsible for clinical and imaging features seen related to the VIIP syndrome .

SAS, on the other hand, is thought to be due to the vestibular system adapting to microgravity, and thus is transient and self-limiting in presentation . However, contributory effects of raised ICP have also been proposed .

The mechanism behind and significance of `space fog' is uncertain . It is thought that perhaps both microgravity effects on the brain and inherent cognitive adaptations (including neuroplasticity) to altered gravity conditions may be responsible for clinical observations .

\paragraph{Radiographic features}

\subparagraph{MRI}

Radiological studies examining brains of astronauts are few in number and inherently limited by small sample sizes. Observations that have been made comparing brain MRI scans pre- and post-spaceflight, mainly after spaceflights of long duration, include:

\begin{itemize}
	\tightlist
	\item
	upward shift of the brain 
	
	\begin{itemize}
		\tightlist
		\item
		also appreciated in patients without spaceflight following a long-term 6° head-down tilt bed rest 
	\end{itemize}
	\item
	narrowing of CSF spaces at the vertex 
	
	\begin{itemize}
		\tightlist
		\item
		also appreciated in patients without spaceflight following a long-term 6° head-down tilt bed rest 
	\end{itemize}
	\item
	narrowing of the central sulcus
	\item
	increased volume of the sensorimotor cortex 
	\item
	altered CSF flow velocities in the cerebral aqueduct 
	\item
	optic nerve sheath distention and other radiological features of papilledema
	\item
	increased T2 white matter hyperintensities, especially in a periventricular distribution 
\end{itemize}

While some of these radiographic findings correlate with known clinical syndromes of spaceflight and microgravity, such as radiographic features of papilledema seen in VIIP syndrome, the clinical significance and correlation of many other features remains uncertain and unexplored .

\paragraph{Treatment and prognosis}

Case series-level evidence suggests that promethazine is effective for managing SAS . Management of VIIP syndrome and other clinical manifestations is uncertain and a focus of ongoing research .