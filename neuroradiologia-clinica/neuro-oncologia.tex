\makeatletter
\def\@makechapterhead#1{%
	{\parindent \z@ \normalfont
		% Riquadro azzurro (3cm di altezza, tutta la larghezza) sotto il riquadro grigio
		\begin{tikzpicture}[remember picture, overlay]
			\fill[neurooncologia] (current page.north west) rectangle 
			([yshift=-6cm]current page.north east);
			% Titolo del capitolo con numero in bianco, allineato a sinistra
			\node[anchor=west, font=\bfseries\Huge\color{white}, xshift=1cm, yshift=-4.5cm] 
			at (current page.north west) {\thechapter \  #1};
		\end{tikzpicture}
		
		% Riquadro  (1cm di altezza, 1/3 della larghezza della pagina) in alto a destra
		\begin{tikzpicture}[remember picture, overlay]
			\fill[neurooncologia2] ([xshift=-\paperwidth/3, yshift=-0cm]current page.north east) 
			rectangle 
			([yshift=-2cm]current page.north east);
			\node[anchor=north east, font=\bfseries\color{white}, yshift=-1cm, xshift=-0.5cm] 
			at (current page.north east) {\partnamebox};
		\end{tikzpicture}
		
		\vspace{6cm} % Per spingere il contenuto sotto i riquadri
	}
}
\makeatother

% Rimuove intestazioni nella prima pagina del capitolo
\fancypagestyle{plain}{%
	\fancyhf{} % Rimuove intestazione e piè di pagina
	\renewcommand{\headrulewidth}{0pt} % Rimuove la linea dell'intestazione
}

\fancypagestyle{neurooncologia}{% normal pages
	
	\fancyhf{}
	\renewcommand{\headrulewidth}{0pt}
	\fancyhead[L]{\color{white}\leftmark \bfseries\color{white}\ \thepage
		\begin{tikzpicture}[remember picture, overlay]
			\fill[neurooncologia] (current page.north west) rectangle 
			([yshift=-1.5cm]current page.north east);
		\end{tikzpicture}
	}
	\fancyhead[R]{\color{white}\rightmark \bfseries\color{white}\ \thepage}
	
}
\pagestyle{neurooncologia}
% Nome della parte (da cambiare per ogni parte)
\renewcommand{\partnamebox}{Neuro-Oncologia} 

\clearpage