\chapter{Ischaemic stroke}

\subsection{Large vessel occlusion}

\textbf{Large vessel occlusion (LVO)}, also termed \textbf{proximal large vessel occlusion (PLVO)}, describes occlusion of a proximal and large-sized intracranial artery resulting in impending acute ischemic stroke. The definition of large vessel occlusion varies significantly among clinical trials of endovascular clot retrieval .

\paragraph{Definition}

One consensus definition of "large vessel" suggests intracranial arteries with a luminal diameter of \textgreater2.0 mm . This is in-keeping with occlusion of the following arteries, which are generally included within the definition of large vessel occlusion across multiple large clinical trials :

\begin{itemize}
	\item
	intracranial internal carotid arteries
	\item
	M1 segments of the middle cerebral arteries
	\item
	basilar artery
	\item
	intracranial vertebral arteries
\end{itemize}

It should be noted that anatomical anomalies of these proximal arteries may have an impact on their inclusion under the luminal diameter definition of "large vessel", such as hypoplasia (e.g. hypoplastic vertebral artery) or duplication (e.g. duplicated middle cerebral artery), which may render the luminal diameter to be \textless2.0 mm . Regardless of this, expert consensus is to include occlusion of these arteries in the definition of large vessel occlusion because the occlusion is still of a proximal intracranial artery .

The M2 segments of the middle cerebral arteries can be highly variable in angioarchitecture between patients (1.1-2.1 mm) and thus, its definition as a large- or medium-sized vessel is a subject of conjecture. In instances of a dominant M2 segment whereby the artery can have a luminal diameter of \textgreater2.0 mm, if occluded, this may be classified as a large vessel occlusion. Indeed, some patients with M2 segment occlusions were included in the original randomized control trials concerning endovascular clot retrieval in anterior circulation large vessel occlusions. However, two negative trials of endovascular clot retrieval in only medium vessel occlusion (MeVO) also included patients with M2 occlusions. Thus, occlusion of the M2 may be variably described as either a large vessel occlusion or medium vessel occlusion .

Similarly, the A1 segments of the anterior cerebral arteries and the P1 segments of the posterior cerebral arteries can also be variable in angioarchitecture between patients and thus, are also a subject of definitional conjecture . Depending on the exact size of the vessel occluded, these may variably be defined as either large vessel occlusions or medium vessel occlusions .
\paragraph{Clinical importance}

\begin{itemize}
	\item
	large vessel occlusion accounts for up to 38\% of acute ischemic stroke 
	\item
	identifying large vessel occlusions on neuroimaging is vital in the work-up and management of acute ischemic stroke as these patients may be eligible for endovascular clot retrieval 
\end{itemize}

\subsection{Medium vessel occlusion}

\textbf{Medium vessel occlusion (MeVO)}, also termed \textbf{distal medium vessel occlusion (DMVO)} or \textbf{distal vessel occlusion (DVO)}, describes occlusion of a medium-sized intracranial artery resulting in impending acute ischemic stroke.

\paragraph{Definition}

One consensus definition of "medium vessel" suggests intracranial arteries with a luminal diameter of 0.75-2.0 mm. Thus, arteries included in the definition of medium vessel occlusion include:

\begin{itemize}
	\item
	M3 and M4 segments of the middle cerebral arteries
	\item
	A2, A3, A4, and A5 segments of the anterior cerebral arteries
	\item
	P2, P3, P4, and P5 segments of the posterior cerebral arteries
	\item
	posterior inferior cerebellar arteries
	\item
	anterior inferior cerebellar arteries
	\item
	superior cerebellar arteries
\end{itemize}

The M2 segments of the middle cerebral arteries, A1 segments of the anterior cerebral arteries and the P1 segment of the posterior cerebral arteries have heterogenous angioarchitecture among patients and as a result, may be variably defined as either sites of medium vessel occlusion or large vessel occlusion (LVO).

Pragmatically, randomized clinical trials of endovascular clot retrieval in medium vessel occlusion have employed the following definitions of 'medium vessel':

\begin{itemize}
	\item
	ESCAPE-MeVO trial :
	
	\begin{itemize}
		\item
		M2 or M3 segments of the middle cerebral arteries
		\item
		A2 or A3 segments of the anterior cerebral arteries
		\item
		P2 or P3 segments of the posterior cerebral arteries
	\end{itemize}
	\item
	DISTAL trial :
	
	\begin{itemize}
		\item
		nondominant or codominant M2 segment of the middle cerebral arteries
		\item
		M3 or M4 segments of the middle cerebral arteries
		\item
		A1, A2, or A3 segments of the anterior cerebral arteries
		\item
		P1, P2, or P3 segments of the posterior cerebral arteries
	\end{itemize}
\end{itemize}


\paragraph{Clinical importance}

\begin{itemize}
	\item
	medium vessel occlusion accounts for 25-40\% of acute ischemic stroke 
	\item
	multi-phase CT angiography with CT perfusion is more sensitive than single-phase CT angiography to detect medium vessel occlusions 
	\item
	in CT perfusion, the traditional Tmax delay cut-off of \textgreater6 seconds may not detect perfusion defects due to medium vessel occlusion in 10\% of cases, and in those cases milder Tmax delay (e.g. \textgreater4 seconds) may be present 
	\item
	although initial research was somewhat equivocal , randomized control trials have not found endovascular clot retrieval to be superior to best medical therapy in patients with acute ischemic stroke due to medium vessel occlusion 
\end{itemize}


\subsection{Stroke protocol (CT)}

A \textbf{CT stroke protocol,} often referred to as a \textbf{code stroke CT}, has become a fairly widespread and standardized approach to imaging patients presenting with acute neurological symptoms that may represent cerebral infarction or cerebral hemorrhage (together grouped under the vague term stroke).

\paragraph{Indications}

A CT stroke protocol is obtained in the emergency setting to rapidly diagnose and quantify patients presenting with probable ischemic strokes and to enable appropriate urgent management (e.g. endovascular clot retrieval or intravenous thrombolysis).

In most centers, CT is favored over MRI in the ultra-acute setting due to time and access constraints, despite acknowledging that MRI, and particularly diffusion-weighted imaging, is superior in identifying small infarcts and defining infarct core.

\subparagraph{Purpose}

The purpose of this protocol is three-fold:

\begin{enumerate}
	\item
	to assess the brain for established infarcts or alternative diagnoses
	\item
	to identify the location and physiological effects of arterial blockage
	\item
	to assess vascular anatomy that may impact endovascular access
\end{enumerate}

To achieve this,stroke protocol CT usually includes 3 concatenated scans :

\begin{enumerate}
	\item
	non-contrast CT (brain)
	\item
	CT perfusion (brain)
	\item
	CT angiography (aortic arch to the vertex of the skull)
\end{enumerate}

It should be noted that this is not uniformly accepted and some centers do not perform perfusion routinely .

There is an increasing trend to perform multi-pass CTA of the brain to perform multiphase CT angiography collateral score in acute stroke.

\subparagraph{Contraindications}

As is the case with other contrast studies,contraindications, such as chronic renal failure and allergy may be important. It is important to note, however, that in the hyperacute setting of evolving stroke, this information is not always known. Furthermore, even if known, with the exception of severe life-threatening allergy, complications of contrast administration may be deemed less important than the appropriate assessment of the stroke.

\paragraph{Non-contrast CT}

A non-contrast CT of the brain, usually obtained volumetrically and reformatted in three planes (sagittal, axial and coronal), is obtained first. In addition to a rapid overview of the brain (see an approach to CT head) that may demonstrate unexpected non-stroke findings (e.g. tumors) it specifically allows for the following stroke-related features to be sought:

\begin{itemize}
	\item
	intracerebral hemorrhages
	\item
	hyperdense artery sign
	\item
	established acute cerebral infarction
	\item
	calculation of ASPECT score
\end{itemize}

\textbf{See:} CT head (technique)

\paragraph{CT perfusion}

Intravenous contrast is then administered and various parameters of cerebral perfusion calculated. Typically these include

\begin{itemize}
	\item
	cerebral blood volume (CBV)
	\item
	cerebral blood flow (CBF)
	\item
	mean transit time (MTT)
	\item
	time-to-maximum (Tmax) or time to peak (TTP)
\end{itemize}

These allow not only the diagnosis and quantification of areas of impaired perfusion but also the identification of infarct core and penumbra that are important in selecting patients for thrombolysis/endovascular clot retrieval.

\textbf{See:} CT cerebral perfusion (technique)

\paragraph{CT angiography}

The last component is CT angiography usually performed from the arch or the aorta to the vertex of the skull. It is performed using the arterial phase of intravascular contrast. It not only allows for the visualization numerous intracranial features relevant to the stroke setting but also anatomy that may be relevant to the endovascular intervention.

\begin{itemize}
	\item
	occlusive thromboembolism
	\item
	arterial dissection
	\item
	aneurysms and arteriovenous malformations
	\item
	spot sign in cerebral hemorrhage
	\item
	bovine arch
\end{itemize}

It should be noted that there is increased interest in the use of multiphase CTA particularly to accurately assess the degree of collateral circulation .

\textbf{See:} CT angiography of the cerebral arteries (technique)

\paragraph{Practical points}

\begin{itemize}
	\item
	the code stroke, although not the most technically demanding radiological protocol, can be a high-stress situation with a variety of extrinsic factors weighing on the radiographer and surrounding healthcare teams
	\item
	workflow will differ between institutions and be based around what works best in that environment
	\item
	some centers will perform the CT angiogram before the perfusion study in order to give the interventional team additional minutes to inspect and plan for a potential clot retrieval as the perfusion study is being performed. This requires advanced trained radiographers, customized pressure injection protocols, and custom CT protocols
\end{itemize}

\subsection{CT angiography of the circle of Willis (protocol)}

\textbf{CT angiography of the circle of Willis} (\textbf{CTA COW}) is a technique that allows visualization of the intracranial arteries; specifically the circle of Willis. While digital subtraction angiography (DSA) remains the gold standard for the diagnosis of intracranial aneurysms especially, CTA is a less invasive, cost-effective, and more widely available technique .

\emph{NB: This article is intended to outline some general principles of protocol design. The specifics will vary depending on CT hardware and software, radiologists' and referrers' preference, institutional protocols, patient factors (e.g. allergy) and time constraints.}

\paragraph{Indications}

CT angiography of the circle of Willis is indicated when characterization of cerebral arterial circulation is required. Indications include:

\begin{itemize}
	\item
	subarachnoid hemorrhage (SAH)
	
	\begin{itemize}
		\item
		when subarachnoid blood is visualized on non-contrast imaging or a high index of suspicion remains to detect responsible aneurysms
		\item
		70\%-85\% of spontaneous SAH are caused by ruptured intracranial aneurysms 
		\item
		CTA negative SAH will require further imaging with DSA
	\end{itemize}
	\item
	arteriovenous malformation
	
	\begin{itemize}
		\item
		diagnosis, monitoring, and planning
		\item
		cannot be ruled out on CTA in patients with high suspicion and requires DSA studies to rule out completely 
	\end{itemize}
	\item
	intracerebral aneurysm
	
	\begin{itemize}
		\item
		\hspace{0pt}monitoring or concern in patients with significant family history
		\item
		while screening is usually not recommended in children or adolescents, patients with two first-degree relatives with an intracerebral aneurysm or autosomal dominant polycystic kidney disease may be considered to undergo screening 
		\item
		screening of patients with no risk factors is not recommended 
	\end{itemize}
	\item
	ischemic stroke
	\item
	intracerebral hemorrhage
	\item
	reversible cerebral vasoconstriction syndrome
	\item
	central nervous system vasculitis
\end{itemize}

\paragraph{Purpose}

The purpose of a CTA COW is to achieve maximum opacification of the circle of Willis in order to identify vascular structure abnormalities or bleeding subarachnoid vessels. Ideally, there should be minimal contrast within the dural venous sinuses; such an instance indicates that the timing is somewhat late.

\paragraph{Contraindications}

\begin{itemize}
	\item
	previous severe reactions to iodinated contrast
	\item
	patient non-compliance (movement)
\end{itemize}


\paragraph{Technique}

\begin{itemize}
	\item
	\textbf{patient position}
	
	\begin{itemize}
		\item
		\textbf{\hspace{0pt}}supine with arms by the patient's side
		\item
		hard palate perpendicular to the table (chin down)
	\end{itemize}
	\item
	\textbf{scout}
	
	\begin{itemize}
		\item
		C2 to vertex
	\end{itemize}
	\item
	\textbf{scan extent}
	
	\begin{itemize}
		\item
		C2 to vertex
	\end{itemize}
	\item
	\textbf{scan direction}
	
	\begin{itemize}
		\item
		caudocranial
	\end{itemize}
	\item
	\textbf{contrast injection considerations}
	
	\begin{itemize}
		\item
		monitoring slice
		
		\begin{itemize}
			\item
			level of C2
		\end{itemize}
		\item
		threshold
		
		\begin{itemize}
			\item
			manual trigger when contrast is seen within vertebral/carotid arteries
		\end{itemize}
		\item
		injection
		
		\begin{itemize}
			\item
			18 g to 20 g in antecubital fossa (preferably)
			\item
			60 mL of non-ionic iodinated contrastwith 50ml saline bolusat 4.5/5 mL/s
		\end{itemize}
	\end{itemize}
	\item
	\textbf{scan delay}
	
	\begin{itemize}
		\item
		minimal scan delay:wait for approximately 5 seconds before contrast monitoring
	\end{itemize}
\end{itemize}

\begin{tcolorbox}[colback=purple!5!white,colframe=purple!75!white,title=Practical points]
	Take care when monitoring for contrast to appear before manually triggering the helical acquisition. Structures such as the styloid processes may be mistaken for arterial blush if the operator is overly zealous. Pay attention to any bony opacities on the monitoring slice before the injection of contrast to ensure there is no confusion as contrast begins to perfuse the arterial circulation.
	
	While contrast within the venous system denotes that the scan is late, this usually does not warrant repeat imaging if the circle of Willis is well opacified. Some venous opacification is unavoidable as the circulation time from arteries to veins is around 3-6 seconds .
	
	If unsure whether to include the carotid arteries and perform a CTA carotids, seek a radiologist's opinion.
	
	Whether or not to include a non-contrast and/or post-contrast brain will depend on local protocol and indications for the scan.
\end{tcolorbox}

\paragraph{Postprocessing}

The processing of data is similar to that of a non-contrast CT brain in regard to image orientation. Axial, coronal, and sagittal multiplanar reformat may be generated at thicknesses to the preference of local departments. Windowing should ideally differentiate between IV contrast, calcified plaques, and soft tissues (e.g WW650 WL150).

Additional post-processing may include maximum intensity projections, curved reformats, and shaded surface display volume rendering (SS-VRT).

\subsection{Stroke protocol (MRI)}

\textbf{MRI protocol for stroke assessment} is a group of MRI sequences put together to best approach brain ischemia.

CT is still the choice as the first imaging modality in acute stroke institutional protocols, not only because the availability and the easy and fast access to a CT scanner, but also due the better sensitivity for intracerebral hemorrhage (ICH) diagnosis. Some institutions also apply a quick MRI stroke protocol for code stroke patients assessment within the narrow time window for thrombolytic therapy.

\emph{Note: This article is intended to outline some general principles of protocol design. The specifics will vary depending on MRI hardware and software, radiologist's and referrer's preference, institutional protocols, patient factors (e.g. allergy) and time constraints.}

\paragraph{Sequences}

A good protocol involves at least:

\begin{itemize}
	\tightlist
	\item
	\textbf{T1 weighted}
	
	\begin{itemize}
		\tightlist
		\item
		plane:sagittal (or volumetric 3D)
		\item
		sequence:fast-spin echo (T1 FSE) or gradient (e.g.T1 MPRAGE)
		\item
		purpose:an anatomical evaluation.Cortical laminar necrosis or pseudolaminar necrosis may be seen as a ribbon of intrinsic high T1 signal, usually after 2 weeks (although it can be seen earlier) 
	\end{itemize}
	\item
	\textbf{T2 weighted}
	
	\begin{itemize}
		\tightlist
		\item
		plane:axial
		\item
		sequence:T2 FSE
		\item
		purpose:
		
		\begin{itemize}
			\tightlist
			\item
			loss of normal signal void in large arteries may be visible immediately
			\item
			after 6-12 hours infarcted tissue becomes high signal 
			\item
			sulcal effacement and mass effect develop and become maximal in the first few days
		\end{itemize}
	\end{itemize}
\end{itemize}

\begin{itemize}
	\tightlist
	\item
	\textbf{FLAIR}
	
	\begin{itemize}
		\tightlist
		\item
		plane:axial
		\item
		sequence:FLAIR
		\item
		purpose:
		
		\begin{itemize}
			\tightlist
			\item
			after 6-12 hours infarcted tissue becomes high signal 
			\item
			sulcal effacement and mass effect develop and become maximal in the first few days
		\end{itemize}
	\end{itemize}
	\item
	\textbf{diffusion-weighted imaging (DWI)}
	
	\begin{itemize}
		\tightlist
		\item
		plane:axial
		\item
		sequence:DWI: B=0, B=1000 and ADC
		\item
		purpose:
		
		\begin{itemize}
			\tightlist
			\item
			early identification of ischemic stroke: diffusion restriction may be seen within minutes following the onset of ischemia 
			\item
			correlates well with infarct core
			\item
			differentiation of acute from chronic stroke
		\end{itemize}
	\end{itemize}
	\item
	\textbf{susceptibility-weighted imaging (SWI)}
	
	\begin{itemize}
		\tightlist
		\item
		plane:axial
		\item
		sequence:susceptibility weighted imaging(ideal) or T2*
		\item
		purpose:highly sensitive in the detection of hemorrhage
	\end{itemize}
\end{itemize}

\begin{itemize}
	\tightlist
	\item
	\textbf{\textbf{MR angiography (MRA)}}
	
	\begin{itemize}
		\tightlist
		\item
		\textbf{\textbf{\hspace{0pt}}\textbf{\hspace{0pt}}}plane:axial with reconstructions
		\item
		sequence:time of flight angiography
		\item
		purpose:assess for luminal diameter and occlusions
	\end{itemize}
\end{itemize}

\subsection{Quick stroke protocol (MRI)}

\textbf{MRI protocol for a quick stroke assessment}corresponds to a short protocol, usually just the DWI/ADC, adopted by some institutions as a complementary tool to the CT imaging in stroke code patients. The aim is to provide additional information to support stroke diagnosis within the narrow time window for thrombolytic therapy.

\emph{Note: This article is intended to outline some general principles of protocol design. The specifics will vary depending on MRI hardware and software, radiologist's and referrer's preference, institutional protocols, patient factors (e.g. allergy) and time constraints.}

\paragraph{\texorpdfstring{Sequences\textbf{\hspace{0pt}\hspace{0pt}}}{Sequences\hspace{0pt}\hspace{0pt}}}

\begin{itemize}
	\tightlist
	\item
	\textbf{diffusion-weighted imaging (DWI)}
	
	\begin{itemize}
		\tightlist
		\item
		plane:axial
		\item
		sequence:DWI: B=0, B=1000 and ADC
		\item
		purpose: early identification of ischemic stroke: diffusion restriction may be seen within minutes following the onset of ischemia 
	\end{itemize}
\end{itemize}

\subsection{Code stroke CT (an approach)}

A \textbf{code stroke CT} can be daunting to interpret as not only does it involve many sequences but it also includes CT perfusion with which many radiologists and clinicians alike are relatively unfamiliar. If that wasn't challenging enough, there is usually the added pressure to make the diagnosis rapidly as treatment is time-critical. As such, having a standardized approach to these studies will not only reduce your stress and make you more resilient to interruptions but also allow you to make the correct diagnosis in a timely fashion.

As with all such articles, there is no single "correct" approach. What is presented below is merely "an approach" and the reader is invited to adapt it in light of local and personal preferences.

\paragraph{Key findings}

It should go without saying that identifying occlusive thromboembolism is the primary purpose of a stroke protocol CT.However, it is also important to recognize that many other diagnoses may have caused an acute stroke-like presentation and that many of these will be visible on CT.

The key findings that should be sought include:

\begin{itemize}
	\item
	ischemic stroke findings
	
	\begin{itemize}
		\item
		localizing thromboembolism and arterial occlusions
		\item
		characterizing the size of the infarct core and penumbra
		\item
		pertinent vascular features relevant to intervention
	\end{itemize}
	\item
	non-ischemic stroke findings
	
	\begin{itemize}
		\item
		intracranial hemorrhages
		\item
		cerebral venous thrombosis
		\item
		other incidental pathology (e.g. aneurysms, tumors, trauma)
	\end{itemize}
\end{itemize}

\paragraph{Order}

Although all features of the three components of a code stroke (non-contrast CT of the brain, perfusion of the brain and CT angiography of the head and neck) need to eventually be looked at in detail, a strong argument can be made for having a systematic order that maximizes the efficiency with which key findings are identified.

Given the time-critical nature of assessing acute strokes, a two-pass approach can be useful and the first pass can be performed quickly, even at the CT workstation, prior to all reconstructions being available.

\begin{figure}
	\centering
	\includegraphics[width=0.7\linewidth]{img/vascular/cerebral-perfusion-parameters}
	\caption{Figures demonstrating the effect of decreased perfusion pressure on various parameters and how they related to infarct core and penumbra. Case courtesy of Frank Gaillard, Radiopaedia.org, rID: 82390}
	\label{fig:cerebral-perfusion-parameters}
\end{figure}

\paragraph{First pass review}

First, quickly review the non-contrast CT scan for intracerebral hemorrhages (both parenchymal and extra-axial), and obvious established infarcts, tumors or other unexpected pathology. Take a few seconds to attempt to identify proximal hyperdense arteries (terminal internal carotid artery, proximal middle cerebral artery, basilar tip). This should only take 15-30 seconds.

Next move onto CT perfusion maps, starting with mean transit time (MTT). These will usually quickly identify large areas of brain with abnormal perfusion. If MTT is completely normal, it is unlikely that a sizable acute occlusive lesion is present.

If there is a region of prolongation, then review cerebral blood volume (CBV).

If the MTT region of abnormality is mostly infarct core, then you most likely will see a matched defect - in other words, an area of reduced CBV will conform to the area of MTT abnormality. This brain is unlikely to be able to be salvaged and clot lysis or retrieval is, therefore, less likely to be of benefit and may, in fact, be contraindicated.

If CBV is normal or elevated, in the context of acute neurological dysfunction, you are most likely looking at a sizable penumbra and this patient needs the most rapid management as they have the largest amount of salvageable tissue. Often, both core and penumbra will co-exist and therefore an attempt should be made to quantify each component as both are important for patient selection. For example "small core, large penumbra" or "large core, no appreciable penumbra"

It is worth noting that successful auto-regulation (for example due to long-standing carotid stenosis) or benign oligemia will also have prolonged MTT and normal or elevated CBV; however, they will usually be asymptomatic and thus not presenting as an acute stroke.

Cerebral blood flow (CBF) is also useful but as it is reduced in both penumbra and infarct core visually assessing it can be challenging and if relied upon to define core it will often lead to an overestimate of core and an underestimate of the penumbra. It is worth noting that some automated software solutions that calculate core and penumbra volume rely on CBF, but they used thresholds to distinguish between the two that are not readily visible to the harried naked eye.

Finally, review the intracranial portion of CT angiogram which should extend from the arch of the aorta to the vertex of this skull. Your search will be made much more targeted by the perfusion scans. If they were normal, it is unlikely that there is a retrievable large vessel occlusion. Conversely, if you identified an area of abnormal perfusion, you will be able to identify the vessel involved due to the distribution, which may be a large vessel occlusion or a medium vessel occlusion.

By the end of this first pass, that should only take a total of a minute or two, you will in most instances have triaged patients accurately into one of four categories:

\begin{enumerate}
	\item
	no obvious abnormality and therefore unlikely to benefit from time-critical therapy
	\item
	obvious non-thromboembolic pathology (e.g. hemorrhage, tumor etc\ldots)
	\item
	obvious thromboembolic stroke who are unlikely to benefit from clot retrieval or thrombolysis (e.g. established infarct with large infarct core and little if any penumbra)
	\item
	obvious thromboembolic stroke who are likely candidates for emergency therapy (e.g. large territory occlusive thromboembolism with small infarct core and large penumbra, or basilar tip thromboembolism)
\end{enumerate}

\paragraph{Next actions}

Before going on to the second pass review you will likely need to/want to communicate your findings to relevant parties. Generally, it is worth letting the stroke team know regardless of the findings as even a message of "there is a large subdural hematoma" or "there is no obvious abnormality" has consequences for immediate management.

If your patient falls into the last category, an obvious acute thromboembolic stroke either with a large penumbra or of the basilar tip, then a few additional features that impact management should be sought before contacting the stroke team and/or neurointerventionalist. This will only take a few seconds.

\begin{itemize}
	\item
	is the thromboembolism calcified?
	\item
	how long is the thromboembolism?
	\item
	how good are the collaterals?
	\item
	what is the arterial access like?
\end{itemize}

\subparagraph{Calcified thromboembolism}

Calcific emboli are less likely to respond to thrombolysis and are more dangerous to retrieve. Depending on the demographics of the patient and clinical presentation (e.g. NIHSS) this may change treatment strategies.

\subparagraph{Length of thromboembolism}

The length of thromboembolism impacts the likelihood of successful thrombolysis, with longer clots less likely to be lysed than short ones. Unfortunately, CT angiography can overestimate the length considerably as visualization of the distal end of the clot relies on backfilling of the vessel with contrast by collaterals (see below). Thus, assessing length based on artery hyperdensity on the thinnest images available windowed to accentuate the clot hyperdensity can be helpful.

Additionally, reviewing the raw perfusion images in some patients can clearly delineate the distal end of the occlusion, provided they have adequate collateral circulation. Perfusion scans are obtained over a relatively long period of time and thus give collaterals a better chance of opacifying the blood just distal to the clot.

\subparagraph{Collateral circulation}

The better the collateral arterial supply to the affected territory, the better the outcome will be. These patients will tend to be able to tolerate ischemia longer before their penumbra progresses to infarction. They are known as slow progressors. In contrast, poor collateral supply will result in rapid progression from penumbra to infarct core. Unfortunately, poor collaterals also increase the likelihood of treatment complications. Overall, therefore, poor collateral circulation is correlated with poor outcome.

There are numerous described grading scales for collateral supply but most boil down to assessing how well the vessels distal to the occlusion are opacified. If no opacification is seen, then collateral supply is poor. In contrast, if they are normal or even more pronounced than the normal contralateral territory, then collateral supply is good.

\subparagraph{Vascular access}

Successful endovascular clot retrieval relies on gaining access to the thromboembolism with a variety of catheters of various diameters and stiffness. Care must therefore be taken to examine the vessel that needs to be catheterized to access the clot from the aortic arch to the site of occlusion.

If, for example, the aortic arch anatomy is unfavorable (e.g. bovine origin of the left common carotid or highly tortuous vessels) accessing the cervical carotid or vertebral arteries may be more challenging, necessitate alternative choice of catheters or require a radial artery approach.

Alternatively, if a tandem lesion is present (i.e. occlusion or high-grade stenosis of the proximal internal carotid artery as well as ipsilateral thromboembolism to the terminal internal carotid or middle cerebral artery) then plans for treatment of both lesions with stents and/or balloons will be required.

It is worth being aware of carotid pseudo-occlusion that refers to apparent continuous cervical internal carotid artery occlusion actually resulting from a stagnant column of unopacified blood proximal to a terminal "T-junction" internal carotid artery occlusion. This can mimic a tandem lesion or cervical carotid dissection.

\paragraph{Second pass review}

Satisfaction of search errors are particularly likely in situations where other abnormalities have already been identified and the patient has been taken for emergent treatment and an incidental aneurysm or tumor or abnormal soft tissues in the neck are easy to overlook if you know the patient is having a basilar tip thromboembolism retrieved.

As such, the second pass is a more leisurely and traditional approach to the entire study and need not have a specific approach in the setting of stroke compared to any other presentation. Rather, approach each component of the study independently (e.g. see CT head - an approach) paying particular attention to unexpected incidental findings that may be overlooked during the first pass.

\subsection{Hyperdense vessel sign}

The \textbf{hyperdense vessel sign} is a radiological sign appreciated on non-contrast CT brain whereby there is focal hyperattenuation (mean of approximately 55 HU) within an intracranial blood vessel . Notably, this is a distinct sign to the hypodense vessel sign.

\paragraph{Terminology}

The hyperdense vessel sign is often used synonymously with \textbf{hyperdense artery sign}, which may be appreciated in the setting of hyperacute ischemic stroke, but hyperdense vessel sign may also be used more liberally to refer to a hyperdense cerebral vein or venous sinus in the setting of cerebral venous sinus thrombosis .

\paragraph{Radiographic features}

\subparagraph{CT}

In the context of hyperactive ischemic stroke, the hyperdense vessel sign represents the acute intraluminal thrombus, and is the earliest radiographic feature of ischemic stroke . It may be appreciated in any large vessel occlusion, including occlusions of the proximal middle cerebral artery segments (see hyperdense MCA sign and MCA dot sign), intracranial internal carotid artery, anterior cerebral artery, posterior cerebral artery, basilar artery, and intracranial vertebral artery . Rarely, the hyperdense vessel segment may have a higher radiodensity suggesting calcium (e.g. \textasciitilde160 HU), due to calcified cerebral embolism . The MRI equivalent of a hyperdense artery sign in this setting is the susceptibility vessel sign .

Similarly, in context of cerebral venous sinus thrombosis, the hyperdense vessel sign again represents the intraluminal thrombus . It is less commonly seen than its arterial counterpart . When present (see dense vein sign and cord sign ), the hyperdense vessel sign is most seen in the superior sagittal sinus, transverse sinus, straight sinus, and cortical veins .

\begin{tcolorbox}[colback=green!5!white,colframe=green!75!white,title=Differential diagnosis]
\begin{itemize}
	\item
	intravascular calcification (e.g. intracranial atherosclerotic disease)
	\item
	elevated hematocrit (e.g. polycythemia)
	\item
	relative vessel hyperattenuation due to surrounding parenchymal hypoattenuation (e.g. in HSV encephalitis mimicking a hyperdense MCA sign)
	\item
	beam hardening artifact (e.g. in interpretation of the basilar artery for this sign)
\end{itemize}
\end{tcolorbox}
\subsection{CT angiography source image ASPECTS}

\textbf{CT angiography source image ASPECTS} (\textbf{CTA-SI ASPECTS}) is an adaptation of ASPECTS for CTA and is a semiquantitative scoring system to characterize the extent and severity of mainly middle cerebral artery ischemic stroke, although it can be adapted to other vascular territories as well.

The added value of CTA-SI ASPECTS, over the usual ASPECTS generated from non-contrast CT, is that it directly shows the degree of collateral circulation and better delineates the infarct territory during the acute phase . Furthermore, differences in contrast enhancement are typically less subtle than the often discrete hypoattenuation caused by early-phase ischemia on noncontrast scans.

\paragraph{Method}

CTA-SI ASPECTS is evaluated using axially reconstructed brain window scans generated from the CTA source images .

For the MCA territory a 10-point scoring system is used identically to the original ASPECTSsystem for non-contrast CT, where any region of relatively diminished contrast enhancement should be registered as abnormal .

\paragraph{Clinical use}

It has been demonstrated that a good CTA-SI ASPECTS score (8-10) is a superior predictor of the final infarct size and clinical outcome to the standard ASPECTS, and most importantly it was found to be a better indicator of the possible outcome of endovascular treatment .

\subsection{Multiphase CT angiography collateral score in acute stroke}

\textbf{Multiphase CT angiography (mCTA)} \textbf{collateral score} is a simple scoring system that allows quick evaluation of collateral vessel filling delay in acute ischemic stroke. In some studies, it has been shown to be a better predictor of clinical outcomes and eligibility for endovascular therapy(ECT) than a decision based on single-phase CT angiography .

\paragraph{mCTA collateral score}

A score on a scale of 0 to 5 is given, with 5 being the best and 0 the worst :

\begin{itemize}
	\tightlist
	\item
	\textbf{5}: no filling delay compared to the asymptomatic contralateral hemisphere, normal pial vessels in the affected hemisphere
	\item
	\textbf{4}:a filling delay of one phase in the affected hemisphere, but the extent and prominence of pial vessels is the same
	\item
	\textbf{3}:a filling delay of two phases in the affected hemisphere, or a delay of one phase with a significantly reduced number of vessels in the ischemic territory
	\item
	\textbf{2}: a filling delay of two phases in the affected hemisphere with a significantly reduced number of vessels in the ischemic territory, or one phase delay showing regions without visible vessels
	\item
	\textbf{1}: only a few vessels are visible in the affected hemisphere in any phase
	\item
	\textbf{0}: no vessels visible in the affected hemisphere in any phase
\end{itemize}

Generally, a score of 3 or less indicates a poor prognosis . Therefore, some investigators have dichotomized the scores such that 0-3 is "poor" and 4-5 is "good" collateral status .

\subsection{ADC pseudonormalization}

\textbf{ADC pseudonormalization} is a normal phase encountered in the subacute stage of ischemic stroke and represents an apparent return to normal healthy brain values on ADC maps which does not, however, represent true resolution of ischemic damage.

ADC pseudonormalization is seen typically around 1 week following ischemic stroke and is thought to be due to a combination of cell wall breakdown and increase of extracellular edema (both of which result in facilitated diffusion - increased ADC values) . At some point, these processes, combined with residual true abnormal restricted diffusion due to cellular swelling, result in ADC values ostensibly returning to those of a normal, healthy brain.

It is important to realize that at the point that ADC values normalize, DWI images (e.g. b = 1000 s/mm) will continue to show increased signal due to T2 shine-through. It is only later (around 10-15 days) that DWI pseudonormalization occurs, when ADC values are higher than normal, offsetting the gradually waning T2 signal.

Note\textbf{:}ADC pseudonormalization should not be confused with early DWI reversal(a.k.a. diffusion lesion reversal),which is seen early in the course of ischemic infarction, particularly in the setting of reperfusion therapy. Nor should it be confused with T2 washout seen a little later.

\begin{tcolorbox}[colback=blue!5!white,colframe=blue!75!white,title=Dense vein sign]
	
	The \textbf{dense vein sign} is a type of hyperdense vessel sign and refers to hyperattenuating thrombus within a cortical vein or dural venous sinus due to acute venous thrombosis.
	
	When located in the superior sagittal sinus, particularly posteriorly, it is sometimes referred to as the \textbf{delta, triangle} or \textbf{pseudodelta sign}. It is really the same as the cord sign except the vessel is imaged in cross-section.
	
	A potential pitfall is interpreting the distal superior sagittal sinus as being hyperdense near the torcula herophili;it is important to appreciate that normal blood within the dural sinuses is usually of slightly increased density relative to brain parenchyma and that true hyperdensity is the key to recognizing thrombosis.The walls at this location can be thick, measuring up to 2-3 mm.
\end{tcolorbox}

\subsection{Anterior circulation infarction}

\textbf{Anterior circulation infarction} describes any infarct in an area of the brain that is within the vascular territory of the anterior circulation, which includes most of the supratentorial structures excluding the occipital lobes. These structures derive their arterial supply from the internal carotid arteries which form part of the anastomotic circle of Willis.

Refer to each specific article for detailed discussion of the various anterior circulation infarcts:

\begin{itemize}
	\item
	anterior cerebral artery (ACA) infarct
	\item
	middle cerebral artery (MCA) infarct
	\item
	anterior choroidal artery infarct
	\item
	lacunar infarct
	
	\begin{itemize}
		\item
		lenticulostriate infarct
	\end{itemize}
	\item
	striatocapsular infarct
\end{itemize}

\paragraph{Epidemiology}

Anterior circulation infarctions account for 70-80\% of all ischemic strokes .

\paragraph{Clinical presentation}

The acute stroke syndrome associated with an anterior circulation infarct is highly dependent on the structures affected. Dysphasia and the presence of both sensory and motor deficits on the same side of the body are generally suggestive of anterior circulation involvement . In all cases, clinicians must maintain a high index of suspicion for posterior circulation stroke by assessing for brainstem signs (e.g. dizziness/vertigo, dysarthria, dysphagia, diplopia, ataxia) .

Differentiating anterior from posterior circulation disease based solely on clinical neurologic deficits is a perilous task; neuroimaging is vital to accurately localize cerebral infarction .

\paragraph{Pathology}

The Bamford classification separates anterior circulation stroke into two clinical types, based on the presence or absence of three components :

\begin{enumerate}
	\item
	unilateral weakness and/or sensory deficit of at least two areas (out of face, arm and leg)
	\item
	homonymous visual field defect
	\item
	higher cerebral dysfunction (e.g. dysphasia, visuospatial disorder)
\end{enumerate}

Under the Bamford classification, total anterior circulation stroke (TACS) requires the presence of all three of the above components, whereas partial anterior circulation stroke (PACS) requires only two of the three. Isolated higher cerebral dysfunction is also classified as PACS .

Once imaging evidence of infarct is obtained, the type of stroke may be coded as total anterior circulation infarct (TACI) or partial anterior circulation infarct (PACI) .

\subparagraph{Etiology}

Broadly, the most common cause of anterior circulation infarct is embolism, which includes thromboembolism of cardiac origin, alongside atherosclerotic embolism from proximal intracranial vessels, the carotids and aorta . Less common causes include vasculopathies, hematologic disorders, hypercoagulability and arterial dissection. Internal carotid artery dissection is a cause of anterior circulation infarct that may occur at any age but is disproportionately common in younger patients .

\paragraph{Radiographic features}

Refer to each specific article for detailed discussion of the radiographic features of infarction in the various arterial territories within the anterior circulation, in addition to the general article on ischemic stroke.

\paragraph{Treatment and prognosis}

For treatment of patients with anterior circulation infarction secondary to large vessel occlusion, there is evidence from randomized controlled trials to support endovascular clot retrieval in addition to best medical therapy as being superior to best medical therapy alone .

\subsection{Isolated insular infarct}

An \textbf{isolated insular infarct} is a form of insular infarct where the infarct is confined to the insular cortex or region supplied by the long insular artery. Infarcts in insula can additionally by supplied by the MCA branches as well as the lenticulostriate branches.

\paragraph{Clinical presentation}

An isolated insular infarct may manifest with a combination of deficits which can include 

\begin{itemize}
	\tightlist
	\item
	somatosensory deficits
	\item
	gustatory/speaking deficits
	\item
	coordination issues/vestibular-like syndrome with
	
	\begin{itemize}
		\tightlist
		\item
		"vestibular-like syndrome" has been reported in around a third of patients 
		\item
		dizziness, gait instability, and tendency to fall, but no nystagmus
	\end{itemize}
	\item
	cardiovascular/autonomic disturbances - hypertensive episodes
	\item
	neuropsychological/cognitive disorders
	
	\begin{itemize}
		\tightlist
		\item
		aphasia (left posterior insula)
		\item
		dysarthria
		\item
		transient somatoparaphrenia (right posterior insula)
	\end{itemize}
\end{itemize}

\subparagraph{Associations}

\begin{itemize}
	\tightlist
	\item
	there can be a high frequency detection of cardiac disturbances (e.g. atrial fibrillation) 
\end{itemize}

\paragraph{Pathology}

\subparagraph{Etiology}

The etiology can vary from being cardioembolic to large artery disease to cryptogenic .

\paragraph{Treatment and prognosis}

Isolated insular infarcts have a generally better overall long term outcome compared to a large territory stroke .

\begin{tcolorbox}[colback=blue!5!white,colframe=blue!75!white,title=Three territory sign]
	The \textbf{three-territory sign} is a radiological sign described in ischemic stroke and is highly specific to hypercoagulability due to malignancy (Trousseau syndrome) being the etiology.However, this sign is not pathognomonic, and may be seen with cardioembolic stroke (e.g. due to atrial fibrillation, endocarditis, cardiac tumors) or stroke due to other prothrombotic states (e.g. due to COVID-19) .
	
	The three-territory sign describes ischemic strokesinvolving three vascular territories including involvement of the bilateral anterior and posterior circulations . Often, the individual ischemic strokes are small, rather than being large vessel occlusions . The sign is best appreciated with diffusion-weighted imaging on MRI .
	
	In one study, the three-territory sign was found to be highly specific (96.4\%) but not sensitive (23.4\%) for hypercoagulability due to malignancy (Trousseau syndrome).Notably, in the same study, the three-territory sign was six times more likely observed in patients with underlying malignancy when compared to patients with underlying atrial fibrillation who may have had multi-territory cardioembolic ischemic stroke (i.e. embolic shower) . In another study, the three-territory sign was found to be an independent marker of increased mortality in patients with acute ischemic stroke in the setting of malignancy .
\end{tcolorbox}

\subsection{Acute basilar artery occlusion}

\textbf{Acute occlusion of the basilar artery} may cause brainstem or thalamic ischemia or infarction.It is a true neuro-interventional emergency, and if not treated early, brainstem infarction results in rapid deterioration in the level of consciousness and ultimately death. It is one of the posterior circulation infarctions.

\paragraph{Epidemiology}

Occlusions of the posterior circulation arteries comprise about a fifth of all strokes but basilar artery occlusion is rare (\textasciitilde1\% of all strokes).

\paragraph{Clinical presentation}

Patients with acute occlusion of the basilar artery will present with sudden and dramatic neurological impairment, the exact characteristics of which will depend on the site of occlusion:

\begin{itemize}
	\item
	sudden death/loss of consciousness
	\item
	top of the basilar syndrome
	
	\begin{itemize}
		\item
		visual and oculomotor deficits
		\item
		behavioral abnormalities
		\item
		somnolence, hallucinations, and dream-like behavior
		\item
		motor dysfunction is often absent
	\end{itemize}
	\item
	proximal and mid portions of the basilar artery (pons) can result in patients being "locked in" 
	
	\begin{itemize}
		\item
		complete loss of movement (quadriparesis and lower cranial dysfunction) and respiratory muscle paralysis
		\item
		preserved consciousness
		\item
		preserved ocular movements (often only vertical gaze) , as the oculomotor nerve is not affected
	\end{itemize}
\end{itemize}

\paragraph{Pathology}

Acute occlusion of the basilar artery can be due to either thromboembolism, atherosclerosis, or propagation of intracranial dissection. Although these may occur anywhere, each of these has a predilection for different segments of the basilar artery:

\begin{itemize}
	\item
	vertebrobasilar junction
	
	\begin{itemize}
		\item
		thromboembolism (e.g. cardioembolic)
		\item
		atherosclerosis with thrombosis
		\item
		propagation of vertebral arterial dissection (rare)
	\end{itemize}
	\item
	midsegment
	
	\begin{itemize}
		\item
		atherosclerosis with thrombosis
	\end{itemize}
	\item
	distal third or basilar tip
	
	\begin{itemize}
		\item
		thromboembolic (e.g. top of the basilar syndrome)
	\end{itemize}
\end{itemize}

\paragraph{Radiographic features}


\subparagraph{Ultrasound}

\begin{itemize}
	\item
	transcranial Doppler
	
	\begin{itemize}
		\item
		absence of signal in the basilar artery
		\item
		indirect signs such as abnormal waveforms in the vertebral arteries and collateral flow
	\end{itemize}
\end{itemize}

\subparagraph{CT}

\begin{itemize}
	\item
	non-contrast CT
	
	\begin{itemize}
		\item
		hyperdense vessel sign of the basilar artery (the basilar artery equivalent of the hyperdense MCA sign), present in \textasciitilde65\% 
		\item
		a high index of suspicion is needed in the correct clinical setting as the diagnosis can easily be missed (often only present on 1 or 2 slices); additionally it is well recognized that acute clots are of lower attenuation than chronic clots 
		\item
		hypoattenuation delineates tissue with ischemic damage (beam-hardening artifacts limit visualization of the brainstem on CT)
	\end{itemize}
	\item
	contrast-enhanced CT
	
	\begin{itemize}
		\item
		CTA: filling defect within the vessel
		\item
		CT perfusion: distinguishes ischemic penumbra area from an irreversibly damaged area (infarct core)
	\end{itemize}
\end{itemize}

\subparagraph{Angiography (DSA)}

Angiography remains the gold standard for the diagnosis of basilar artery occlusion. However, DSA is used only after non-invasive imaging for therapeutic recanalization. Images demonstrate a filling defect within the vessel.

\subparagraph{MRI}

\begin{itemize}
	\item
	loss of flow void within the basilar artery on spin-echo and FLAIR images
	\item
	\textbf{DWI:}restricted diffusion within infarcted tissue
	\item
	\textbf{T2/FLAIR:}hyperintense signal within infarcted tissue
\end{itemize}

\paragraph{Treatment and prognosis}

Acute occlusion of the basilar artery is a life threatening event, which carries a terrible prognosis:\textasciitilde90\% mortality depending on the location, and high morbidity in the survivors .

Multidisciplinary consensus for individualized management is difficult to achieve in a time-critical fashion.

Treatment usually involves catheter-directed intra-arterial thrombolysis and intravenous heparin, which carries a risk of hemorrhage of up to 15\%. Mechanical embolectomy with a clot retrieval device has been used in selected cases.

\subparagraph{Predictors of outcome after mechanical thrombectomy}
\begin{description}
	\item[Age and gender] Analysis of the BASICS randomized control trial reports no significant differences between age groups observed for recanalization rate and incidence of symptomatic intracranial hemorrhage. Patients ≥75 years with basilar artery occlusion have an increased risk of poor outcome compared with younger patients, but a substantial group of patients ≥75 years survive with a good functional outcome .No significant gender differences for outcome and recanalization were observed, regardless of treatment modality .
	
	\item[Collateral flow] Several studies, including a series of 21 patients and another of 104 patients, have found that the presence of bilateral posterior communicating arteries on pretreatment CTA was associated with more favorable outcomes after mechanical thrombectomy in basilar artery occlusion .
	
	\item[Vertebral artery stenosis] From the BASICS study, in patients with acute basilar artery occlusion, unilateral vertebral artery occlusion or stenosis ≥50\% is frequent, but not associated with an increased risk of poor outcome or death. Patients with basilar artery occlusion and bilateral vertebral occlusion had a slightly increased risk of poor outcomes .
	
	\item[Vertebrobasilar artery calcification] In a cohort study of 64 patients, vertebrobasilar artery calcification was found to be an independent predictor of outcome and associated with reduced functional independence and increased mortality in this demographic .
	
	\item[Posterior circulation Acute Stroke Prognosis Early CT score (pc-ASPECTS)] An analysis of BASICS suggested that a cerebral blood volume (CBV) pc-ASPECTS \textless8 may indicate patients with high case fatality. However, further evidence is needed as CTA and CT perfusion were available in only 27/592 (5\%) of BASICS patients.
\end{description}

\subsection{Embolic shower}

\subsubsection{Field}

The term \textbf{embolic shower} is a commonly used radiological description of a specific pattern of ischemic stroke, however, it is poorly defined in the medical literature.

\paragraph{Terminology}

Embolic shower is usually used to describe numerous, bilateral, often small, acute ischemic strokes, involving multiple vascular territories seemingly at random, which have all occurred at one single or similar time point . The term can also be used more liberally, referring to emboli causing infarcts to multiple organs, however, this article will focus on its use in relation to ischemic stroke.

\paragraph{Pathology}

The term is typically employed with reference to a 'central' source of embolism as the cause for ischemic strokes. This is typically in relation to the heart (e.g. atrial fibrillation, infective endocarditis, during/after cardiac surgery), but may also include the aortic arch (e.g. aortic arch atheroma) . However, a similar pattern of acute ischemic stroke can also occur due to embolism from more distal arteries or other etiologies (e.g. hypercoagulable states from malignancy, cerebral fat embolism, hypotension) .

\paragraph{Radiographic features}

An embolic shower is best appreciated on MRI on diffusion weighted imaging , whereby the ischemic strokes are classically, but not exclusively, affecting the external (cortical) border zone . However, evidence of an embolic shower may also be appreciated on CT if the ischemic strokes are established and/or large enough, or if the emboli are calcific (known as the salted pretzel sign) .

\paragraph{Treatment and prognosis}

The prognostic value of this radiological pattern is uncertain given there is no formally accepted definition for what constitutes an embolic shower. One study focusing on acute ischemic strokes involving multiple arterial territories, regardless of etiology, found that patients with this pattern of infarction were more likely to have altered conscious state, seizures, and generally poorer outcomes .

\begin{tcolorbox}[colback=blue!5!white,colframe=blue!75!white,title=String of pearls sign (watershed infarction)}
The \textbf{string of pearls sign} is seen on diffusion-weighted imaging of T2/FLAIR as a series of rounded areas of signal abnormality adjacent to, but separate from, the lateral ventricle. This represents a deep border zone infarct between the penetrating cortical arteries and ascending perforating arteries .
\end{tcolorbox}

\section{Common pitfalls}

\subsection{Diffusion-negative acute ischemic stroke}

\textbf{Diffusion-negative acute ischemic stroke} refers to a clinically diagnosed acute ischemic stroke without cerebral restricted diffusion on DWI on brain MRI.

Although DWI is highly sensitive for acute ischemic strokes, it fails in a minority of cases in its detection .

\paragraph{Epidemiology}

It is not as rare as previously thought and has a reported prevalence of 6.8\% in a meta-analysis of 3,236 ischemic strokes .

\paragraph{Radiographic features}

DWI is reported to fail in the detection of ischemic strokes involving:

\begin{itemize}
	\tightlist
	\item
	posterior circulation infarction:5x more likely to be DWI-negative than anterior circulation ischemia, especially within the first 48 hours 
	\item
	small strokes, particularly small brainstem infarcts 
	\item
	hyperacute ischemia: within 3 hours of symptom onset 
\end{itemize}

\paragraph{Differential diagnosis}

The differential is that of a neurological deficit with a normal brain MRI :

\begin{itemize}
	\tightlist
	\item
	transient ischemic attack
	\item
	vertebrobasilar insufficiency
	\item
	migraine
	\item
	functional neurological disorder
\end{itemize}

\subsection{Decompression illness}

\textbf{Decompression illness} \textbf{(DCI)} encompasses decompression sickness (DCS) and arterial gas embolism. The term decompression illness refers to inert bubble-induced dysbaric disease regardless of the location of the bubbles, which may be in the tissues or in the intravascular spaces. The bubbles arise due to a rapid drop in atmospheric pressure or due to pulmonary barotrauma.

\paragraph{Epidemiology}

Decompression illness can occur in:

\begin{itemize}
	\item
	divers if ascent to the surface is too rapid
	
	\begin{itemize}
		\item
		the reported incidence in sports divers is 3 in 10,000 dives
		\item
		in commercial divers the incidence is up to 10 in 10,000 dives
	\end{itemize}
	\item
	workers leaving a caisson (pressurised chamber)
	\item
	during unpressurised flight to high altitude
	\item
	extra-vehicular activity in space
\end{itemize}

\paragraph{Clinical presentation}

Symptoms depend on the location of the bubbles, and vary from mild to severe:

\begin{itemize}
	\item
	bone and/or muscle pain, typically in the shoulders, and less frequently in the elbows, knees or ankles
	\item
	headache and focal neurological deficits, such as paralysis, visual disturbances or vertigo
	\item
	death
\end{itemize}

For a further discussion please see decompression sickness.

\paragraph{Pathology}

According to Henry's Law, the solubility of a gas in a liquid is proportional to the partial pressure of a gas over the liquid. For divers, pressure increases by one atmosphere for every additional 10.06 meters in depth, causing more gas to be dissolved in blood and body fluids. Oxygen is metabolized, but gases like nitrogen and helium accumulate. On ascent these gases leave solution and the gas bubbles expand. If ascent is too rapid this causes decompression illness.

Pulmonary barotrauma can occur during diving descent (lung squeeze) or ascent (pulmonary overinflation syndrome). Gas can enter lung parenchyma or vessels through tears and can cause arterial gas embolism.

Bubbles can block vessels, cause spasm or cause endothelial damage which activates both the clotting cascade and inflammatory mediators leading to increased permeability, edema and ischemia.

\paragraph{Radiographic features}

Chest radiograph is important to exclude pneumothorax prior to treatment in a hyperbaric chamber.

For further discussion please see decompression sickness.

\paragraph{Treatment and prognosis}

General treatment options include:

\begin{itemize}
	\item
	FiO\textsubscript{2} 100\% oxygen
	\item
	hyperbaric chamber if there is no inner ear barotrauma
\end{itemize}
\subsection{Early DWI reversal in ischemic stroke}

\textbf{Early DWI reversal in ischemic stroke} (also referred to as \textbf{diffusion lesion reversal}) is sometimes encountered early in the course of ischemic stroke.

Hyperintensity on DWI develops within minutes of ischemia and was believed to be highly sensitive and specific in defining the ischemic core. Over time, however, a number of exceptions to this rule have been described and DWI reversal is one such example.

\paragraph{Terminology}

DWI reversal generally refers to the resolution of some or all hyperintensity seen on initial imaging when this is compared to subsequent MRI. Which sequences are used for this later comparison, as well as precise definitions of what is meant by DWI reversal, is variable, resulting in heterogenous incidence and outcomes .

Generally, the initial DWI findings are compared to either subsequent DWI studies (so-called DWI-based studies), or FLAIR/T2 (so-called FLAIR/T2-based studies).

Depending on which sequence is used, the incidence of DWI reversal varies:

\begin{itemize}
	\item
	DWI-based studies: 26.5\% 
	\item
	FLAIR/T2-based studies: 6\% 
\end{itemize}

These figures are, however, for partial reversal. Complete reversal is rare, occurring only in 0.8\% of cases acute ischemic stroke .

Note: Early DWI reversal should not be confused with ADC pseudonormalization, which occurs later in the evolution of ischemic stroke.

\paragraph{Treatment and prognosis}

DWI reversal is most frequently encountered in the setting of reperfusion (endovascular clot retrieval of thrombolysis) within 3 to 6 hours of onset .

Overall presence of DWI reversal is associated with some improvement in clinical outcomes (e.g. NIHSS and modified Rankin Scale)  although in some/many cases, it is transient, and the abnormal DWI signal returns .

\begin{tcolorbox}[colback=green!5!white,colframe=green!75!black,title=Differential diagnosis]
	A similar appearance is encountered in other clinical contexts (e.g. hemiplegic migraine, seizure, transient ischemic attack, transient global amnesia, and in some instances of PRES - those that show a degree of restricted diffusion (26\% of cases)), wherein DWI changes resolve .
\end{tcolorbox}
	
\subsection{Carotid pseudo-occlusion}

\textbf{Carotid pseudo-occlusion} refers to apparent occlusion of the cervical internal carotid artery on CT angiography or digital subtraction angiography due to a stagnant column of unopacified blood proximal to terminal T-junction occlusion by thromboembolism.

It is important not to mistake this for true occlusion, carotid dissectionor a tandem lesion.

\paragraph{Terminology}

The term carotid pseudo-occlusion has also been used to describe carotid near-occlusion when the internal carotid artery distal to high-grade stenosis at its origin is underfilled.
\subsection{Mitochondrial encephalomyopathy with lactic acidosis and stroke-like episodes (MELAS)}

\textbf{Mitochondrial encephalomyopathy with lactic acidosis and stroke-like episodes} (\textbf{MELAS}) is one of many mitochondrial disorders. As mitochondria, which have their own DNA,are exclusively passed on from the mother, these disorders are only maternally inherited.

On imaging, it manifests as multifocal stroke-like cortical lesions in different stages of evolution ("shifting spread" pattern), crossing the cerebral vascular territories, and showing a certain predilection to the posterior parietal and occipital lobes.MR spectroscopy may demonstrate elevated lactate in an otherwise normal appearing brain .

\paragraph{Epidemiology}

As the name suggests, MELAS is characterized by 'stroke-like' episodes, typically in childhood or early adulthood (90\% present before 40 years of age).

\paragraph{Clinical presentation}

MELAS usually has a relapsing-remitting course, with or without superimposed accretion of permanent deficits.Clinical presentation is highly variable between patients (even from the same affected family), with potential clinical features including :

\begin{itemize}
	\item
	stroke-like episodes (e.g. hemiparesis, hemianopia)
	\item
	lactic acidosis
	\item
	encephalopathy, including seizures and migraine-like headaches
	\item
	dementia
	\item
	proximal muscle weakness (myopathy)
	\item
	sensorineural hearing loss
	\item
	diabetes mellitus
	\item
	colonic pseudo-obstruction
	\item
	peripheral neuropathy
	\item
	short stature
\end{itemize}

\paragraph{Pathology}

The defect involves the respiratory chain, which is responsible for energy production . A point mutation at mtDNA nucleotide 3243 (A to G translocation) which encodes for transfer RNA (tRNA) for leucine is the most common cause (\textasciitilde80\%) of the condition . It is therefore thought that this abnormality results in abnormal protein production throughout the mitochondria and affects multiple parts of the respiratory chain. The exact mechanism notwithstanding, the net result is depletion of NAD+ and NADH+. This, in turn, results in a shift to anaerobic metabolism accounting for the buildup of lactic acid and renders the cortex susceptible to neuronal death .

As some mitochondria are passed in the ovum, not all will have the mutant mtDNA. The percentage of mutated genes will affect the severity of clinical manifestations .

\subparagraph{Diagnosis}

To make the diagnosis of MELAS identification of the most common pathogenic mtDNA variant (m.3243A\textgreater G) can be made on peripheral blood samples in 80\% of patients. To identify non-m.3243A\textgreater G mutations additional testing or muscle biopsy may be required .

\paragraph{Radiographic features}


\subparagraph{CT}

\begin{itemize}
	\item
	multiple infarcts
	
	\begin{itemize}
		\item
		involving multiple vascular territories
		\item
		may be either symmetrical or asymmetrical
		\item
		parieto-occipital and parieto-temporal involvement is most common
	\end{itemize}
	\item
	basal ganglia calcification 
	
	\begin{itemize}
		\item
		more prominent feature in older patients
	\end{itemize}
	\item
	atrophy 
\end{itemize}


\subparagraph{MRI}

\begin{itemize}
	\item
	acute cortical infarct-like lesions
	
	\begin{itemize}
		\item
		swollen gyri with increased T2 signal
		\item
		may enhance
		\item
		subcortical white matter involved
		\item
		increased signal on DWI (T2 shine through) with little if any change on ADC, thought to represent vasogenic rather than cytotoxic edema 
	\end{itemize}
	\item
	chronic cortical infarct-like lesions
	
	\begin{itemize}
		\item
		involving multiple vascular territories
		\item
		may be either symmetrical or asymmetrical
		\item
		the black toenail sign may be present in the subacute phase
		\item
		parieto-occipital and parieto-temporal regions most common
	\end{itemize}
	\item
	calcification of basal ganglia (intrinsic T1 hyperintensity, SWI hypointensity)
\end{itemize}

\textbf{MR spectroscopy}: may demonstrate elevated lactate in otherwise normal appearing brain parenchyma or in CSF .


\subparagraph{Digital subtraction angiography (DSA)}

\begin{itemize}
	\item
	usually normal
	\item
	enhancing gyri, presumably due to the breakdown of the blood-brain barrier and reperfusion hyperemia correlating with a blush on angiography 
\end{itemize}

\paragraph{Treatment and prognosis}

There is no disease-modifying treatment. Substances such as L-arginine, taurine and coenzyme Q10 are thought to aid in increasing energy production by mitochondria and may slow the effects of the disease.

Antiseizure medications should be used for symptomatic treatment of seizures. However, those that interfere with respiratory chain function, such as sodium valproate, are avoided due to the potential of aggravating manifestations of MELAS.

\paragraph{Differential diagnosis}

Possible differential considerations include:

\begin{itemize}
	\item
	other mitochondrial disorders
	
	\begin{itemize}
		\item
		MERRF
		\item
		Leigh syndrome
		\item
		Kearns-Sayre syndrome
	\end{itemize}
	\item
	status epilepticus
	\item
	viral encephalitis
	\item
	cerebral vasculitis
	\item
	Creutzfeldt-Jakob disease
	\item
	ischemic strokes, due to
	
	\begin{itemize}
		\item
		embolism
		\item
		dissection
		\item
		moyamoya syndrome
		\item
		CADASIL: lesions are not subcortical
	\end{itemize}
\end{itemize}