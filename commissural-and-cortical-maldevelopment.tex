\chapter{Commissural and cortical anomalies}

\subsection{Lissencephaly-pachygyria spectrum}

\textbf{Lissencephaly-pachygyria spectrum} describes the spectrum of diseases that cause relative smoothness of the brain surface, and includes :

\begin{itemize}
	\item
	\textbf{agyria}:no gyri
	\item
	\textbf{pachygyria}:broad gyri
	\item
	\textbf{lissencephaly}: smooth brain surface
\end{itemize}

It is a basket term for a number of congenital cortical malformationscharacterized by absent or minimal sulcation.

\paragraph{Types}

Lissencephaly-pachygyria can be further divided into types I (classic) and type II (cobblestone). They differ in clinical presentation, underlying genetic abnormalities, as well as microscopic and macroscopic (including imaging) appearances . They themselves represent a heterogeneous group of disorders. This article highlights a few generalities and outlines the differences between the two types, which are otherwise discussed separately :

\begin{itemize}
	\item
	type I (classic) lissencephaly
	\item
	type II (cobblestone complex) lissencephaly
\end{itemize}

\paragraph{Clinical presentation}

Type I (classic) lissencephaly typically presents with marked hypotonia and paucity of movement.

Type II lissencephaly is associated with muscular dystrophy-like syndromes and includes Walker-Warburg syndrome,Fukuyama syndrome, and muscle-eye-brain (MEB) disease.

\paragraph{Radiographic features}

Although lissencephaly can be identified on all cross-sectional modalities (antenatal and neonatal ultrasound, CT and MRI), MRI is the modality of choice to fully characterize the abnormalities.

\subparagraph{MRI}

Type I and type II lissencephaly demonstrate vaguely similar appearances (thus the common term lissencephaly) but different macroscopic and imaging appearances are visible.

Type I (classic) lissencephaly can appear as the classic hourglassor figure-8 appearance or with a few poorly formed gyri (pachygyria) and a smooth outer surface. It is usually associated with band heterotopia.

Type II lissencephaly, on the other hand, has a microlobulated surface referred to as a cobblestone complex. Band heterotopia is not evident and the cortex is thinner than in type I.

\paragraph{Differential diagnosis}

Microcephaly with a simplified gyral pattern which describes a reduced number of gyri and shallow sulci with a normal cortical thickness and architecture.

\subsection{Linear scleroderma}

\textbf{Linear scleroderma},also known as \textbf{scleroderma}\textbf{en coup de sabre}, is a very focal form of sclerodermaclassically characterized by a linear band of atrophy involving the frontal or frontoparietal scalp and subjacent thinned calvaria associated with ipsilateral focal brain abnormalities.

Linear scleroderma may coexist with progressive facial hemiatrophy (Parry-Romberg syndrome).

\paragraph{Radiographic features}

In the brain parenchyma beneath the skin lesion, focal atrophy and blurring of the gray-white matter interface can be identified Calcification and cerebral microhemorrhageshave also been reported .
