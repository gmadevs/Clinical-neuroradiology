\chapter{Lysosomial storage}

Lysosomal storage disorders

\textbf{Lysosomal storage disorders} (\textbf{LSDs}) form a large group of clinical entities, more than forty now described, with the common etiological theme being the presence of dysfunctional lysosomal proteins, with the secondary accumulation of toxic metabolites inside the cellular lysosomes.

\paragraph{Epidemiology}

The prevalence of these individual disorders ranges from 1 in 57 000 for Gaucher disease to 1 in 4.2 million for sialidosis . As a group of disorders the prevalence is 1 per 7 700 live births .

\begin{itemize}
	\tightlist
	\item
	alpha-mannosidosis
	\item
	aspartylglucosaminuria
	\item
	cholesteryl ester storage disease
	\item
	chronic hexosaminidase A deficiency
	\item
	cystinosis
	\item
	Danon disease
	\item
	Fabry disease
	\item
	Farber disease
	\item
	fucosidosis
	\item
	galactosialidosis
	\item
	Gaucher disease
	\item
	GM1 gangliosidosis
	\item
	GM2 gangliosidosis
	
	\begin{itemize}
		\tightlist
		\item
		Tay-Sachs disease
		\item
		Sandhoff disease
		\item
		AB variant
	\end{itemize}
	\item
	I-cell disease/mucolipidosis II
	\item
	infantile free sialic acid storage disease
	\item
	juvenile hexosaminidase A deficiency
	\item
	Krabbe disease
	\item
	lysosomal acid lipase deficiency
	\item
	metachromatic leukodystrophy
	\item
	mucopolysaccharidoses
	\item
	multiple sulfatase deficiency
	\item
	Niemann-Pick disease
	\item
	neuronal ceroid lipofuscinoses
	\item
	Pompe disease
	\item
	pycnodysostosis
	\item
	Schindler disease
	\item
	Salla disease
	\item
	Wolman disease
\end{itemize}