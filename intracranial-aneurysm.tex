\chapter{Intracranial aneurysms}

\subsection{Intracranial aneurysm (overview)}

\textbf{Intracranial aneurysms}, also called \textbf{cerebral aneurysms},are aneurysms of the intracranial arteries. The most common morphologic type is the saccular aneurysm, although a number of other morphologies and etiologies occur.

Due to a combination of thinner/weaker walls and Laplace's law, aneurysms tend to enlarge progressively. This, in turn, can lead to rupture (andsubarachnoid hemorrhage) or partial thrombosis (due to slow turbulent flow).

\paragraph{Pathology}

There is not a universal classification for the types of intracranial aneurysms, resulting in a heterogeneous mix of terms based on the morphology, size, location, and etiology :

\begin{itemize}
	\item
	saccular intracranial aneurysm
	
	\begin{itemize}
		\item
		giant intracranial aneurysm
	\end{itemize}
	\item
	fusiform intracranial aneurysm
	\item
	blood blister-like aneurysm
	\item
	dissecting intracranial aneurysm
	\item
	mycotic (infectious) intracranial aneurysm
	\item
	traumatic intracranial aneurysm
	\item
	neoplastic (oncotic) intracranial aneurysm
\end{itemize}
\subsection{Saccular cerebral aneurysm}

\textbf{Saccular cerebral aneurysms}, also known as \textbf{berry aneurysms},are intracranial aneurysms with a characteristic rounded shape. They account for the vast majority of intracranial aneurysms and are the most common cause of non-traumatic subarachnoid hemorrhage.

\paragraph{Terminology}

Those larger than 25 mm in the maximal dimension are called giant cerebral aneurysms.

Charcot-Bouchard aneurysmsare minute aneurysms which develop as a result of chronic hypertension and appear most commonly in the basal gangliaand other areas such as the thalamus,pons, and cerebellum, where there are small penetrating vessels (diameter \textless300 micrometers).

\paragraph{Epidemiology}

Prevalence of saccular cerebral aneurysms in the asymptomatic general population has been reported over a wide range (0.2-8.9\%) when examined angiographically, and in 15-30\% of these patients, multiple aneurysms are found .

A familial tendency to aneurysms is also well recognized, with patients who have more than one first-degree relative affected, having a \textasciitilde30\% (range 17-44\%) chance of themselves having an aneurysm .

\subparagraph{Associations}

Numerous associations have been identified, most relating to abnormal connective tissue. Associations include:

\begin{itemize}
	\item
	Ehlers-Danlos syndrome(type IV)
	\item
	Marfan syndrome(controversial )
	\item
	autosomal dominant polycystic kidney disease
	\item
	coarctation of aorta
	\item
	bicuspid aortic valve
	\item
	neurofibromatosis type 1
	\item
	hereditary hemorrhagic telangiectasia
	\item
	alpha-1-antitrypsin deficiency
	\item
	cerebral arteriovenous malformation: a flow related aneurysm
	\item
	fibromuscular dysplasia
	\item
	thoracicand abdominal aortic aneurysms
\end{itemize}

\paragraph{Clinical presentation}

Saccular cerebral aneurysms are most often asymptomatic until rupture (see ruptured saccular aneurysm), at which point they usually result in subarachnoid hemorrhage (e.g. thunderclap headache, reduced conscious state), with or without associated intracerebral hemorrhage and/or subdural hemorrhage (rarely, in up to 8\% of cases) . Occasionally larger aneurysms (e.g. giant cerebral aneurysms) will result in symptoms without (or prior to) rupture, such as those from mass effect .

\paragraph{Pathology}

Macroscopically, aneurysms are rounded lobulated focal outpouchings,usually arising at arterial bifurcations.

Most intracranial aneurysms are true aneurysms. The aneurysmal pouch is composed of thickened hyalinised intima with the muscular wall and internal elastic lamina being absent as the normal muscularis and elastic lamina terminate at the neck of an aneurysm. As an aneurysm grows it may become irregular in outline, and may have mural thrombus. Typically rupture occurs from dome .

\subparagraph{Location}

Cerebral aneurysms typically occur at branch points of larger vessels but can occur at the origin of small perforators which may not be seen on imaging. Approximately 90\% of such aneurysms arise from the anterior circulation, and 15-30\% of these patients have multiple aneurysms .

\begin{itemize}
	\item
	anterior circulation: \textasciitilde90\%
	
	\begin{itemize}
		\item
		ACA/ACoA complex: 30-40\%
		\item
		supraclinoid ICA and ICA/PCoA junction: \textasciitilde30\%
		\item
		MCA (M1/M2 junction) bi/trifurcation: 20-30\%
	\end{itemize}
	\item
	posterior circulation: \textasciitilde10\%
	
	\begin{itemize}
		\item
		basilar tip
		\item
		SCA
		\item
		PICA
	\end{itemize}
\end{itemize}

\subparagraph{Size}

The Unruptured Cerebral Aneurysm Study (UCAS) classified cerebral aneurysms depending on size, however, this has not been universally adopted :

\begin{itemize}
	\item
	small (\textless5 mm)
	\item
	medium (5-10 mm)
	\item
	large (10-25 mm)
	\item
	giant (\textgreater25 mm)
\end{itemize}

\paragraph{Radiographic features}

Saccular aneurysms can be imaged in a variety of methods:

\begin{itemize}
	\item
	CT angiography
	\item
	MR angiography
	\item
	digital subtraction (catheter) angiography (DSA)
\end{itemize}

Each of these confers certain advantages and disadvantages, although in general digital subtraction catheter angiography, especially with 3D acquisitions, is considered the gold standard in most institutions.

\subparagraph{CT}

The appearance depends upon the presence of thrombosis within an aneurysm. An aneurysm appears as a well-defined round, slightly hyperattenuating lesion, most apparent on maximum intensity projection (MIP)reformatted images.

\begin{itemize}
	\item
	calcification can be present
	\item
	post contrast
	
	\begin{itemize}
		\item
		patent aneurysm: bright and uniform enhancement
		\item
		thrombosed aneurysm: rim enhancement due to filling defect
	\end{itemize}
\end{itemize}

\subparagraph{MRI}

On MRI also the patent and thrombosed aneurysm display different imaging features:

\begin{itemize}
	\item
	\textbf{T1}
	
	\begin{itemize}
		\item
		most of the patent aneurysms appear as flow void, or they may show heterogeneous signal intensity
		\item
		in thrombosed aneurysms, the appearance depends on the age of clot within the lumen
	\end{itemize}
	\item
	\textbf{T2}
	
	\begin{itemize}
		\item
		typically hypointense
		\item
		laminated thrombus may show a hyperintense rim
	\end{itemize}
\end{itemize}

\subparagraph{Digital subtraction angiography (DSA)}

It has been reported more sensitivity in 3D DSA over 2D DSA when regarding the detection of small aneurysms . Attention must be given when measuring the aneurysm neck size as it can be overestimated by 3D reconstructions.

\paragraph{Treatment and prognosis}

Treatment of large or symptomatic aneurysms should be considered, with either endovascular coiling or surgical clipping.

Management of small aneurysms is controversial. Less than 7 mm in maximal diameter aneurysms are statistically unlikely to rupture, however, due to their prevalence, anyone working in the area has seen numerous patients with small aneurysms which have ruptured resulting in subarachnoid hemorrhage, often with devastating consequences.

Five-year cumulative risk of rupture of anterior circulation aneurysms :

\begin{itemize}
	\item
	\textless7 mm: 0\%
	\item
	7-12 mm: 2.6\%
	\item
	13-24 mm: 14.5\%
	\item
	\textgreater25 mm: 40\%
\end{itemize}

Five-year cumulative risk of rupture of posterior circulation aneurysms :

\begin{itemize}
	\item
	\textless7 mm: 2.5\%
	\item
	7-12 mm: 14.5\%
	\item
	13-24 mm: 18.4\%
	\item
	\textgreater25 mm: 50\%
\end{itemize}

As such management will vary according to local experience, the location and appearance of an aneurysm, patient demographics, etc. Risk models include the PHASES risk prediction scoreand those based on the ISUA-II and UCAS trials. Management options include endovascular occlusion with coils, flow diversion devices,endosaccular flow disruption devices, or surgical clipping.

The risk-stratification schemes take into account size and other physiological parameters including blood pressure. However, there is some evidence  that the shape of the aneurysm is also predictive of future aneurysm rupture risk:

\begin{itemize}
	\item
	aspect ratio:≥1.6 (the ratio of the maximal height of the aneurysm and the width of the neck)
	\item
	size ratio:≥1.7 (the ratio of the maximal height of the aneurysm and the width of the vessel of origin)
	\item
	area ratio:≥1.5 (the ratio of the area of the aneurysm to the parent artery in the neck plane)
	
	\begin{itemize}
		\item
		area of the aneurysm: π x Hp x W
		\item
		area of parent artery within neck:π x Dv x N
		\item
		Hp: perpendicular height measured as the largest perpendicular distance from the plane between aneurysm neck and dome
		\item
		W: width of the aneurysm (longest diameter perpendicular to Hp)
		\item
		Dv: diameter of the vessel
	\end{itemize}
\end{itemize}

Endovascular coiling is graded with the Raymond--Roy Occlusion Classification (RROC) \textbf{} scheme.

\begin{tcolorbox}[colback=green!5!white,colframe=green!75!white,title=Differential diagnosis]
	When the abnormality has been confirmed to be vascular, the differential includes:
	
	\begin{itemize}
		\item
		fusiform aneurysm
		\item
		infundibulum: usually triangular dilatation with the vessel arising from the apex
		\item
		dissecting aneurysm
		\item
		mycotic aneurysm
		\item
		variant arterial anatomy (see imaging differential diagnosis case)
	\end{itemize}
\end{tcolorbox}

\begin{tcolorbox}[colback=purple!5!white,colframe=purple!75!white,title=Practical points]
	Regardless of the modality used, a number of features need to be assessed to allow a decision in relation to treatment to be made:
	
	\begin{itemize}
		\item
		size: ideally three axis maximum size measurements
		\item
		neck: maximal width of the neck of an aneurysm
		\item
		the shape and lobulation
		\item
		orientation: the direction in which the aneurysm points is often important in both endovascular and surgical planning
		\item
		any smaller branches in the vicinity of an aneurysm
		\item
		any branch taking off from the aneurysm
		\item
		the presence of other aneurysms or vascular malformations
		\item
		relevant arterial variant anatomy (that may complicate or exclude endovascular treatment)
	\end{itemize}
\end{tcolorbox}

\subsection{Ruptured saccular aneurysm}

\textbf{Ruptured saccular (berry) aneurysms} usually result in subarachnoid hemorrhage (SAH)but can, depending on the location of the rupture and presence of adhesions to the aneurysm, also result in cerebral hematoma, subdural hematoma, and/or intraventricular hemorrhage.

\paragraph{Epidemiology}

Saccuar aneurysms form 97\% of all aneurysms of the central nervous system. Up to 80\% of patients with a spontaneous subarachnoid hemorrhage have ruptured an aneurysm and 90\% of these aneurysms are located in the anterior circulation (carotid system), with 10\% found in the posterior circulation (vertebrobasilar system).

\paragraph{Clinical presentation}

Rupture of a saccular aneurysm with associated subarachnoid hemorrhage most frequently presents with a sudden, excruciating headache often described as "the worst I've ever had" or "thunderclap", resulting from blood being forced into the subarachnoid space under arterial pressure. Other features include:

\begin{itemize}
	\item
	visual changes
	\item
	facial pain
	\item
	seizures
	\item
	autonomic disturbances (nausea/vomiting, chills and palpitations)
	\item
	focal neurology (sensory loss, weakness, memory loss, language difficulties)
\end{itemize}

Examination findings include meningism(nuchal rigidity, fever, photophobia),altered consciousness, and other focal neurological signs such as ophthalmoplegia and pupillary abnormalities.

\paragraph{Pathology}

Although some of the details of the pathophysiology of the formation of a saccular aneurysm remain unknown, the vast majority of aneurysms arise at arterial branching points along the circle of Willis . It is likely that the difference in composition of intracranial arteries compared to similarly sized arteries in the rest of the body (e.g. reduced thickness of adventitia) plays a significant role in aneurysm formation and rupture. Additional deficiencies in arterial wall strength (e.g. connective tissue disease or infection) further increase the incidence of aneurysm formation.

The rupture of a saccular aneurysm is in most cases spontaneous with no clear precipitant. In approximately one-third of cases, associated increased intracranial arterial pressure can be surmised by history or examination at the time of presentation (e.g. coital rupture, recreational drugs, childbirth, etc.).

\subparagraph{Etiology}

Saccular aneurysms either form sporadically or secondary to a genetic predisposition:

\begin{itemize}
	\item
	sporadic (most common)
	
	\begin{itemize}
		\item
		a genetic component may also be implicated as there is an increased incidence in first-degree relatives of affected patients
	\end{itemize}
	\item
	genetic
	
	\begin{itemize}
		\item
		Ehlers-Danlos syndrome type IV
		\item
		neurofibromatosis type 1
		\item
		Marfan syndrome
		\item
		autosomal dominant polycystic kidney disease
	\end{itemize}
\end{itemize}

\subparagraph{Macroscopic appearance}

An unruptured aneurysm appears as a thin-walled, shiny red outpouching usually measuring a few milimeters to 3 cm in diameter. Rupture usually occurs at the apex of the sac .

\subparagraph{Microscopic appearance}

The arterial wall adjacent to the neck of the sac usually shows thickening of the intima and thinning out of the media as the neck is approached. The sac itself is usually made up of thickened intima with the adventitia of the parent artery surrounding the sac .

\paragraph{Radiographic features}

\subparagraph{Determining the site of rupture}

After rupture, the location of the blood or hematoma can help determine the site of the ruptured aneurysm in the majority of cases:

\begin{itemize}
	\item
	\textbf{ACOM:} (\textasciitilde35\%) septum pellucidum, interhemispheric fissure and intraventricular, inferior frontal lobe (intraparenchymal)
	\item
	\textbf{PCOM:} (\textasciitilde35\%) Sylvian fissure,medial temporal lobe (intraparenchymal)
	\item
	\textbf{MCA:} (\textasciitilde20\%)Sylvian fissure and intraventricular, anterior temporal lobe (intraparenchymal)
	\item
	\textbf{basilar artery:} (\textasciitilde5\%) prepontine cistern
	\item
	\textbf{ICA:} Sylvian fissure and intraventricular
	\item
	\textbf{pericallosal artery:} corpus callosum
	\item
	\textbf{PICA:} foramen magnum
\end{itemize}

An intracerebral hemorrhage adjacent to the ruptured aneurysm is known as a jet hematoma or flame hemorrhage, caused when an aneurysm abuts a lobe and at the time of rupture the pressure of the blood leaving the aneurysm dissects into the brain parenchyma. This often coexists with the presence of subdural hemorrhage from the aneurysmal rupture (which occurs in up to 8\% of cases), although subdural hemorrhages can also occur independently due to saccular aneurysm rupture .

\subparagraph{Aneurysmal characteristics suggestive of rupture}

In cases of subarachnoid hemorrhage with multiple aneurysms, it is often important to identify which aneurysm has bled, as not all aneurysms present can be treated simultaneously. The location of blood, particularly if there is a parenchymal hematoma, is very helpful in identifying the responsible aneurysm. If this is not present or if blood is diffusely within the subarachnoid space, then aneurysm morphology can be helpful:

\begin{itemize}
	\item
	largest aneurysm
	\item
	length-to-neck ratio \textgreater1.6 
	\item
	increased volume to surface area 
	\item
	aneurysm angulation 
	\item
	presence of a focal bleb/outpouching (known as Murphey's teat) 
\end{itemize}

\paragraph{Treatment and prognosis}

The rupture of an intracranial aneurysm is a medical emergency with a high mortality .Treatment focuses on managing both the aneurysm and complications of hemorrhage.

The aneurysm needs to be secured, either endovascularly by the introduction of coils and/or stents, or by surgery with clipping of the aneurysm neck.

Mortality from the first rupture is between 25-50\% with repeat bleeding a common complication in survivors. With each recurrent bleed, the prognosis is worsened. In the first few days following a subarachnoid hemorrhage, there is an increased risk of additional ischemic injury from the reactive vasospasm from surrounding vasculature.

\subparagraph{Complications}

Complications that require management include:

\begin{itemize}
	\item
	elevated intracranial pressure
	
	\begin{itemize}
		\item
		hydrocephalus
	\end{itemize}
	\item
	cerebral vasospasm causing delayed cerebral ischemia
	\item
	hyponatremia
	\item
	coronary spasm
	\item
	neurogenic pulmonary edema
	\item
	pulseless electrical activity (PEA)
\end{itemize}

\subsection{Endosaccular flow disruption devices}

\textbf{Endosaccular flow disruption devices}, also simply known as \textbf{flow disruptors}, are used for the treatment of either ruptured or unruptured saccular, wide-neck, usually bifurcation as well as side-wall intracranial aneurysms. Their primary function is to stop blood from flowing into the aneurysm, allowing time for the aneurysm to heal. They are usually tightly packed nitinol-based materials in different shapes.

Types of flow disruptors include:

\begin{itemize}
	\item
	Woven EndoBridge (WEB)Aneurysm Embolization System
	\item
	Artisse™ (formerly LUNA) Intrasaccular device
	\item
	Medina Embolic Device (MED)
	\item
	Contour Neurovascular System™:hybrid design of neck-bridging and endosaccular device that aims to close the aneurysm at its neck, without the need to cater to the entire aneurysm volume or shape, therefore expanding its indications
\end{itemize}

The major advantage of endosaccular flow disruptors to other stents i.e., flow diverters, etc. is the ability to treat both unruptured and acutely ruptured wide-neck or sidewall aneurysms without the necessity to prescribe pre-operative antiplatelet or anticoagulation.

Antiplatelet regimens differ between institutions ranging from none, to single, to dual antiplatelet therapy pre- and post-operatively depending on the ruptured or unruptured aneurysmal state, immediate angiographic aneurysm occlusion,or any procedure-related events such as device protrusion into its parent vessel. In a multicenter study (WWWeb Consortium), it was found that an antiplatelet regimen did not change treatment outcome .

\paragraph{Complications}

The most common complication is thromboembolism. Delayed ipsilateral parenchymal hemorrhage,a very serious but rare complication, was also reported.

\paragraph{Outcomes}

Among them, WEB is the most widely known and studied device. Large European multicenter trials (WEBCAST and WEBCAST 2) illustrated a very high success rate (complete occlusion or neck remnant at one-year follow-up) in 80\% with no device-related mortality and only 1.8-2\% morbidity .

Another multicenter study conducted in 22 centers within continental Europe, South, and North America showed an even better 85.7\% success rate .

Compared to other endovascular techniques i.e., stent or balloon-assisted coiling, endosaccular flow disruptor shows slightly better angiographic outcome, with no post-operative re-rupture.

If an adequate occlusion is not achieved after serial angiographic studies, retreatment with conventional coiling may be indicated.

\subparagraph{Follow-up imaging}

Serial imaging may assess the recurrence of the aneurysm which can infrequently occur due to device migration or device compaction, possibly requiring re-treatment. Inadequate occlusion usually leans toward the device-compaction group.

Flow disruptors do not produce significant MR artifact and are not contraindications for MRI. Time-of-flight MRA is a useful screening modality to assess WEB-treated aneurysms. DSA with VasoCT remains the gold standard for follow-up.

The WEB Occlusion Scale (WOS) which was adapted from Bicêtre Occlusion Scale Score (BOSS) can be employed to assess the adequacy of treatment.
\subsection{Non-ischemic cerebral enhancing (NICE) lesions}

\textbf{Non-ischemic cerebral enhancing (NICE) lesions} are an uncommon delayed complication of cerebrovascular procedures, including aneurysm coiling, thrombectomy and placement flow-diverter stent placement .

\paragraph{Epidemiology}

As NICE lesions are seen following endovascular procedures most commonly for aneurysm treatment, the epidemiology of cases mirrors that of cerebral aneurysms. Reported cases vary in age from 31 to 71 years .

The incidence of NICE lesions in patients who have undergone aneurysm coiling is reported as approximately 0.5\%  although, given the low number of reported cases and the potential for many cases being unrecognised, the true incidence is unknown.

There appears to be a female predilection .

\paragraph{Clinical presentation}

NICE lesions are a delayed phenomenon and typically present many days to weeks following the procedure; reported range: 2 weeks to 12 months . Symptoms are variable depending on the location and number of lesions .

CSF examination is usually bland or demonstrates mild elevation of protein and white cells .

\paragraph{Pathology}

The cause of NICE lesions remains somewhat controversial with most authors believing that they represent a granulomatous reaction to foreign body emboli occurring at the time of intervention . An alternative theory is that they are the result of nickel hypersensitivity, less favored as many endovascular catheters and devices are made of other alloys (e.g. cobalt-chromium) .

In some reported cases where biopsy was undertaken, histology demonstrated foreign-body granulomatous reaction surrounding sterile microabscess containing fragments of the hydrophilic coating of many catheters, such as polyvinylpyrrolidone (PVP) . Similar reactions have been identified in other parts of the body (e.g. the wrist from radial puncture in coronary angiograms) .

It has been postulated that the risk of developing NICE lesions is related to tight-fitting catheter/guidewire/device combinations, where the risk of sheering off the hydrophilic coating is theoretically higher . This may also explain why the use of flow-diverting stents, although the devices are nowadays very flexible and soft, are over-represented in reported cases as flow diverters have to be pushed out of a tight fitting sheath and microcatheter .

Of note, it has been theorized that the same mechanism may be responsible for delayed intraparenchymal hemorrhage (DIPH) following aneurysm repair  although other theories also exist.

\paragraph{Radiographic features}

NICE lesions are typically only seen in the territory corresponding to the vessel that was the target of endovascular therapy . Variations in circle of Willis anatomy can therefore influence the distribution of lesions.

\subparagraph{CT}

Non-contrast CT may be normal or demonstrate small areas of hypodensity .

\subparagraph{MRI}

NICE lesions appear as multiple punctate or small ring enhancing foci. They can involve the white matter, cortex or leptomeninges . They have the following signal characteristics :

\begin{itemize}
	\item
	\textbf{T1:} hypointense
	\item
	\textbf{T2}
	
	\begin{itemize}
		\item
		central iso- to hypointense
		\item
		peripheral hyperintensity due to vasogenic edema
	\end{itemize}
	\item
	\textbf{DWI/ADC:} variable
	\item
	\textbf{SWI/T2*:} variable, some normal other show signal loss
\end{itemize}

\subparagraph{Angiography (DSA)}

No angiographic features have been described .

\paragraph{Treatment and prognosis}

The natural history of NICE lesions remains to be established. Corticosteroids can reduce edema and enhancement, however, recurrence may occur and thus follow-up with MRI is recommended . Antiseizure medications may be necessary in patients with seizures .

\paragraph{Differential diagnosis}

\begin{itemize}
	\item
	subacute infarcts: may appear similar, however, would be expected to appear immediately following the procedure and will undergo expected evolution
	\item
	cerebral microabscesses: may appear similar on imaging but are usually encountered in a different clinical scenario and would not be expected to be confined to the vascular territory of prior treatment
\end{itemize}