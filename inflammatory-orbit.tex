\chapter{Inflammatory disease}

\section{Chronic relapsing inflammatory optic neuropathy}

\textbf{Chronic relapsing inflammatory optic neuropathy (CRION)} describes a rare, recurrent, corticosteroid-responsive optic neuropathy that should be considered as an important differential diagnosis in patients with multiple episodes of suspected optic neuritis. Although traditionally thought of as a seronegative condition, many cases of CRION may be part of myelin oligodendrocyte glycoprotein antibody-associated disease(MOGAD)}.

\paragraph{Epidemiology}

The exact incidence of CRION is unknown. Generally, patients are young female adults, with one study finding a median affected age of approximately 35 years .

\paragraph{Clinical presentation}

Chronic relapsing inflammatory optic neuropathy is characterized by subacute objective visual loss and pain with at least one relapse . Optic disc swelling may also be present but is not a cardinal clinical feature . Relapses can occur in the same eye (more common) or in both eyes sequentially or simultaneously .

\paragraph{Pathology}

The underlying pathophysiology of chronic relapsing inflammatory optic neuropathy is unknown (as of September 2018), however it is thought to be immune-mediated due to the positive response patients have to immunosuppressivetherapy .

\subparagraph{Markers}

Affected patients are negative for AQP4 antibodies, and thus, CRION is generally thought of as a diagnosis of exclusion . However, there is growing literature on the relationship between CRION and MOG antibody positivity, with evidence suggesting that many patients, if not most, with CRION are MOG antibody positive and may actually have recurrent or bilateral optic neuritis of myelin oligodendrocyte glycoprotein\hspace{0pt antibody-associated disease(MOGAD)} instead .

\paragraph{Radiographic features}

CT is often unremarkable , however, MRI is useful in the diagnosis of CRION, particularly in excluding differential diagnoses.

\subparagraph{MRI}

Dedicated orbital views are ideal, demonstrating a thickened optic nerve with signal characteristics similar to those seen in optic neuritis:

\begin{itemize}
\item
\textbf{T2/FLAIR:} high signal, more prominent with fat suppression
\item
\textbf{T1 C+ (Gd):} enhancement,more prominent with fat suppression
\end{itemize}

Unlike patients with demyelinating optic neuritis, such as in multiple sclerosis, accompanying parenchymal signal changes are usually absent, and have only been rarely reported in the literature .

\paragraph{Treatment and prognosis}

Acute management generally involves pulsed methylprednisolone, followed by a tapering regimen of oral prednisolone . In the long term and once clinically stable, steroid-sparing immunosuppressants may be employed, such as azathioprine or cyclophosphamide .

Unlike demyelinating optic neuritis, prompt corticosteroid therapy in CRION does alter visual outcome and ultimate prognosis .

\paragraph{Differential diagnosis}

\begin{itemize}
\item
neuromyelitis optica spectrum disorder
\item
myelin oligodendrocyte glycoprotein\hspace{0pt antibody-associated disease(MOGAD)}
\item
neurosarcoidosis
\item
ischemic optic neuropathies
\item
infectious optic neuropathies (e.g. neurosyphilis)
\item
metabolic optic neuropathies (e.g. methanol poisoning)
\item
Leber hereditary optic neuropathy
\end{itemize}