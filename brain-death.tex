\chapter{Brain death}

\section{Stasis filling}

\textbf{Stasis filling} describes persistent visualization of intravenous contrast within the proximal cerebral arteries but not within the cortical branches or venous outflow in suspected brain death patients, mimicking true cerebral blood flow (CBF).

\paragraph{Pathology}

\subparagraph{Etiology}

During brain death, raised intracranial pressures result in increased cerebral vascular resistance and eventual cessation of blood flow within the capillaries and deep veins . When this occurs, the proximal intracranial arteries may still be patent and transcranial Doppler sonography (TCD) will demonstrate oscillating, forward systolic and reversed diastolic flow . It is postulated that slow propagation of intravenous contrast can then occur. Despite apparent contrast within the proximal cerebral arteries, studies investigating this phenomenon with CT perfusion have demonstrated that no true flow occurs (i.e. CBF and CBV of 0) .

\paragraph{Radiographic features}

Stasis filling will appear as visualization of contrast within the proximal cerebral arteries, but not within the cortical branches . The visualized flow is usually significantly delayed and weaker than is expected in healthy individuals .

\subparagraph{CT}

CT angiography (CTA) will demonstrate opacification of the proximal cerebral arteries and not the cortical branches. It is more commonly seen with delayed phase imaging (usually 60 seconds) than in the early phase (5 seconds) .

In jurisdictions where CTA is legally recognized as an ancillary test for suspected brain death, the presence of stasis filling does not preclude the diagnosis of brain death . In some cases, the finding is equivocal and may require repeat imaging or the use of alternate imaging modalities .

\subparagraph{Angiography (DSA)}

Similar findings have been reported with DSA, but are less commonly seen due to the inherent higher contrast resolution of CT .

\begin{tcolorbox}[colback=green!5!white,colframe=green!75!white,title=Differential diagnosis]
	\begin{itemize}
		\item
		true cerebral blood flow
		\item
		severe hypoxic ischemic encephalopathy
	\end{itemize}
\end{tcolorbox}