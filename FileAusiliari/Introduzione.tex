		\titleformat{\chapter}
		[hang]
		{\Huge}
		{}
		{0em}
		{}
		[\Large {\begin{tikzpicture} [remember picture, overlay]
		\pgftext[right,x=14.75cm,y=0.2cm]{\HUGE\bfseries 
			Introduzione}
		\end{tikzpicture}}]
%%%%%%%%%%%%%%%%%%%%%%%%%%%%%%%%%%%%%%%%%%%%%%%%%%%%%%%%%%%%%%%%%%%%%%%%%%%%%%%%%
\chapter*{}\normalfont\addcontentsline{toc}{part}{Introduzione}
La neuroradiologia clinica è una disciplina straordinaria, in costante evoluzione, che richiede una comprensione approfondita e aggiornata per affrontare al meglio le sfide della pratica quotidiana. Questo libro nasce da un'idea semplice ma ambiziosa: creare una risorsa accessibile a tutti, gratuita e sempre aggiornata, che possa fungere sia da guida per chi si avvicina a questa materia, sia da riferimento affidabile per i professionisti più esperti.

Viviamo in un'epoca in cui la condivisione delle conoscenze è più importante che mai. In questo spirito, questo testo è pensato non solo come un manuale, ma come un progetto di comunità, in cui ogni lettore è invitato a contribuire con idee, suggerimenti e correzioni. Solo attraverso uno sforzo collaborativo possiamo garantire che questo libro rimanga rilevante e utile nel tempo.

Il pubblico a cui si rivolge è ampio e variegato: dagli studenti di medicina e specializzandi in radiologia, che troveranno qui un punto di partenza solido, ai neuroradiologi e clinici esperti, che potranno utilizzarlo come strumento di consultazione e aggiornamento.

Invito tutti voi, lettori e colleghi, a considerare questo libro non solo come un prodotto finito, ma come un cantiere aperto, dove ogni voce ha valore. Ogni contributo, piccolo o grande, arricchirà questa risorsa, rendendola sempre più completa e utile per la nostra comunità professionale.

Grazie per il vostro interesse e il vostro supporto in questo progetto. Insieme, possiamo costruire qualcosa di significativo per la neuroradiologia clinica.

\textit{Giorgio Maria Agazzi}