\chapter*{Preface}
Welcome to the essential guide to neuroradiology for residents and new specialists. The intricate world of neuroimaging, with its rapid advancements and vast information, can often feel overwhelming. This book cuts through that complexity, distilling the field into an accessible, practical resource designed to support you in your daily practice and board preparation.

Drawing heavily from the rich, peer-reviewed knowledge base of Radiopaedia, and supplemented by carefully selected open-access articles, this book ensures you have access to high-quality, up-to-date information. In line with the principles of open knowledge sharing, all text and images within this book are released under the Creative Commons Attribution-ShareAlike 4.0 International (CC BY-SA 4.0) license. This means you're free to share and adapt the material, provided you give appropriate credit and distribute any new contributions under the same license.

Our aim is to demystify challenging cases, highlight critical diagnostic pearls, and equip you with the confidence to interpret a wide range of neuroradiological cases, ensuring you're well-prepared for any clinical scenario and ready to excel in this dynamic specialty.

Throughout this book, you'll find specialized text boxes designed to enhance your learning and quick reference. These are color-coded for easy identification based on their content: green boxes are dedicated to differential diagnoses, helping you consider various possibilities for a given imaging finding. Purple boxes provide crucial practical points, offering tips and insights for real-world application in your daily work. Finally, blue boxes highlight essential radiological signs, drawing your attention to key imaging features that are critical for accurate interpretation. This visual system is designed to help you quickly identify and absorb the most pertinent information.

These are some examples of color coded blocks:

\begin{tcolorbox}[colback=blue!5!white,colframe=blue!75!white,title=Radiological sign]
	Radiological sign
\end{tcolorbox}

\begin{tcolorbox}[colback=green!5!white,colframe=green!75!white,title=Differential diagnosis]
	Differential diagnosis
\end{tcolorbox}

\begin{tcolorbox}[colback=purple!5!white,colframe=purple!75!white,title=Practical points]
	Practical points
\end{tcolorbox}

This book is meticulously structured to guide you through the breadth of neuroradiology, starting with foundational knowledge and progressing to complex pathologies. We begin with an Introductory Section covering essential neuroanatomy, fundamental neurological concepts, and the various radiological techniques indispensable to the field. Following this groundwork, Section 2: Trauma delves into the critical imaging aspects of head and spinal injuries. Section 3: Vascular then explores the wide spectrum of cerebrovascular diseases, from acute stroke to vascular malformations. Section 4: Infection, Inflammation, and Demyelinating Diseases provides a detailed approach to infectious and inflammatory conditions, including pyogenic, viral, fungal, and parasitic infections, as well as demyelinating and other inflammatory processes. The extensive Section 5: Neoplasms, Cysts, and Tumor-Like Lesions offers a systematic review of central nervous system tumors, encompassing everything from diffuse gliomas and neuronal tumors to meningiomas, metastases, and various non-neoplastic cysts. Moving on, Section 6: Toxic, Metabolic, Degenerative, and CSF Disorders addresses a range of conditions, including toxic encephalopathies, inherited and acquired metabolic disorders, dementias, and hydrocephalus. Finally, Section 7: Congenital Malformations and Genetic Tumor Syndromes concludes the book with an in-depth look at developmental anomalies of the brain and spine, along with associated genetic and neurocutaneous syndromes. This logical progression is designed to provide a cohesive and thorough learning experience across all major neuroradiological domains.

\textit{Giorgio Maria Agazzi}