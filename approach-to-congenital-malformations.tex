\chapter{Congenital malformations}

\subsection{Megalencephaly}

\textbf{Megalencephaly}encompasses a variety of disorders characterized by an abnormally large brain, typically due to either over-growth of the brain (anatomic megalencephaly) or accumulation of abnormal metabolites within the brain parenchyma (metabolic megalencephaly) .

It may involve all or part of the cerebral hemispheres and can be bilateral, unilateral (hemimegalencephaly) or focal (e.g. lobar or involving only the cerebellum) . It is often associated with other structural abnormalities and has numerous associations.

\paragraph{Terminology}

\subparagraph{Megalencephaly vs macrocephaly}

Megalencephaly should not be used interchangeably with macrocephaly, which means an increase in size of the cranial vault. The latter may be due to many other causes, and megalencephaly is an uncommon cause of macrocephaly, hydrocephalus, for example, being far more common.

Additionally, not all patients with megalencephaly will necessarily have macrocephaly.

\subparagraph{Definition}

Megalencephaly is defined as a brain whose weight or size exceeds two standard deviations above the mean for age and gender matched cohort . In practice, as volumetric assessment of the brain (or weighing the brain) is impractical, the size of the skull (macrocephaly) is used as a surrogate .

\paragraph{Epidemiology}

Megalencephaly is uncommon and is seen associated with a variety of underlying conditions.

\paragraph{Clinical presentation}

It may be apparent as abnormal head circumference measurements, especially noted in the first four months of life.

Intellectual disability, seizures, and other neurological abnormalities have been reported. It is important to emphasize that there is no classical pattern of symptoms .

\paragraph{Pathology}

Megalencephaly is a complex abnormal cell proliferation process representing excessive amount of normal brain constituents, cellular proliferation and/or inadequate physiologic apoptosis (anatomic/developmental megalencephaly) or the abnromal accumulation of metabolites (metabolic megalencephaly) .

\subparagraph{Anatomic megalencephaly}

Anatomic megalencephaly, also referred to as developmental megalencephaly, resulst sfrom enlargement of brain parenchyma due to increase number of normal cellular constituents . This results from a wide variety of usually genetic abnormalities. As a general rule, these causes are more likely to result in focal abnormalities (e.g. hemimegalencephaly, lobar megalencephaly, focal tubers) and are more likely due to denovo somatic mutations .

Causes include :

\begin{itemize}
	\item
	acrocallosal syndrome
	\item
	Bannayan-Riley-Ruvalcaba syndrome
	\item
	cardiofaciocutaneous syndrome
	\item
	CLOVES syndrome
	\item
	Costello syndrome
	\item
	Cowden syndrome
	\item
	epidermal nevus syndrome
	\item
	Gorlin-Goltz syndrome
	\item
	Klippel-Trenaunay-Weber syndrome
	\item
	Legius syndrome
	\item
	Lhermitte-Duclos syndrome
	\item
	macrocephaly capillary malformation syndrome
	\item
	megalencephaly polymicrogyria-polydactyly hydrocephalus syndrome
	\item
	neurofibromatosis type 1
	\item
	Noonan syndrome
	\item
	Noonan syndrome with multiple lentigines (LEOPARD syndrome)
	\item
	Opitz-Kaveggia syndrome
	\item
	Perlman syndrome
	\item
	Pretzel syndrome
	\item
	proteus syndrome
	\item
	Simpson-Golabi-Behmel syndrome
	\item
	Soto syndrome
	\item
	tuberous sclerosis
	\item
	Weaver syndrome
\end{itemize}

\subparagraph{Metabolic megalencephaly}

Metabolic causes of megalencephaly are numerous and heterogeneous and not all are due to merely the accumulation of abnormal metabolite. In some instances inflammatory changes may also contribute . In general, metabolic causes are more likely to involve the whole brain and are often due to inherited recessive mutations. They include :

\begin{itemize}
	\item
	leukodystrophies
	
	\begin{itemize}
		\item
		Alexander disease
		\item
		Canavan disease
		\item
		megalencephalic leukoencephalopathy with subcortical cysts (Van der Knaap disease)
		\item
		leukoencephalopathy with vanishing white matter
	\end{itemize}
	\item
	lysosomal storage diseases
	
	\begin{itemize}
		\item
		Hunter syndrome
		\item
		Hurler syndrome
		\item
		Krabbe disease
		\item
		Maroteaux-Lamy syndrome
		\item
		Sandhoff disease
		\item
		Sanfilippo syndrome
		\item
		Sly syndrome
		\item
		Tay-Sachs disease
	\end{itemize}
	\item
	organic acid disorders
	
	\begin{itemize}
		\item
		glutaric aciduria (type 1)
		\item
		D-2-Hydroxyglutaric aciduria
		\item
		L-2-Hydroxyglutaric aciduria
	\end{itemize}
\end{itemize}

\paragraph{Radiographic features}

\subparagraph{MRI}

In addition to the brain being enlarged, the brain often also demonstrates abnormal structure including, thickened cortex, abnormal gyration, hyperintense white matter, cerebellar overgrowth and polymicrogyria . The particular features will depend on the underlying abnormality.