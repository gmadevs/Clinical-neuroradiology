\chapter{General imaging patterns}

\section{Cerebellar hypoplasia}

\textbf{Cerebellar hypoplasia}is largely a descriptive term that encompasses a wide range of conditions, including congenital morphological cerebellar abnormalities and acquired changes that result in the cerebellum having reduced volume, stable over time . The pattern of volume loss may be regional (affecting only part of the cerebellum) or global.

\paragraph{Terminology}

Cerebellar hypoplasia is divided by many into primary (congenital) and secondary (acquired) forms, although distinguishing between the two is not always possible .

In some cases, global cerebellar hypoplasia can appear indistinguishable from diffuse cerebellar atrophy on a single study. It can only be distinguished from the latter by demonstrating or implying (clinically) that there has been no change over time .

\paragraph{Clinical presentation}

The clinical presentation is different, varying from normal to severe bilateral spastic cerebral palsy, intellectual disability, seizures, microcephaly and sensorineural hearing loss .

\paragraph{Pathology}

Given the heterogeneity of conditions included in the term cerebellar hypoplasia, it is unsurprising that the underlying pathology is also variable.

The primary causes of global cerebellar atrophy are chromosomal abnormalities (trisomy 13 and trisomy 18), metabolic disorders, genetic syndromes, and migrational disorders while congenital infections (cytomegalovirus followed by rubella and varicella viruses) are considered as secondary causes .

\paragraph{Radiographic features}

All imaging modalities show a reduction in size and volume of parts of, or the entire, cerebellum with variable degrees of enlargement of adjacent CSF spaces. The pattern will vary and imaging features will help identify the underlying cause/condition, which are discussed separately .

\begin{itemize}
	\item
	global cerebellar hypoplasia
	
	\begin{itemize}
		\item
		chromosomal abnormalities (e.g. trisomy 13 and trisomy 18)
		\item
		genetic disorders
		
		\begin{itemize}
			\item
			CHARGE syndrome
			\item
			Cri du chat syndrome
			\item
			neurofibromatosis type 1
			\item
			Ritscher-Schinzel syndrome
			\item
			and many more... 
		\end{itemize}
		\item
		metabolic disorders
		
		\begin{itemize}
			\item
			mitochondrial disorders (e.g. Leigh disease, pyruvate
			
			dehydrogenase deficiency)
			\item
			mucopolysaccharidoses (types I and II)
			\item
			Smith-Lemli-Opitz syndrome
			\item
			Zellweger syndrome
		\end{itemize}
	\end{itemize}
	\item
	unilateral cerebellar hypoplasia
	
	\begin{itemize}
		\item
		PHACES syndrome
		\item
		familial porencephaly
	\end{itemize}
	\item
	pontocerebellar hypoplasia
	
	\begin{itemize}
		\item
		acquired (e.g. cerebellar agenesis, severe prematurity)
		\item
		congenital muscular dystrophy (e.g. Fukuyama disease, muscle-eye-brain disease, Walker-Warburg syndrome)
		\item
		cortical malformations (e.g. lissencephaly, primary microcephaly, polymicrogyria)
		\item
		genetic disorders (e.g. \emph{CASK} mutation, cerebellofaciodental syndrome)
		\item
		metabolic diseases (e.g. congenital disorders of glycosylation)
		\item
		pontocerebellar hypoplasia of Barth
	\end{itemize}
	\item
	vermian hypoplasia
	
	\begin{itemize}
		\item
		genetic disorders
		
		\begin{itemize}
			\item
			acrocallosal syndrome
			\item
			Beckwith-Wiedemann syndrome
			\item
			Gillespie syndrome
		\end{itemize}
		\item
		malformations
		
		\begin{itemize}
			\item
			Dandy Walker malformation
			\item
			isolated inferior vermian hypoplasia
			\item
			pontine temental cap dysplasia
			\item
			vermian aplasia (Joubert syndrome)
			\item
			rhombencephalosynapsis
		\end{itemize}
	\end{itemize}
\end{itemize}

\paragraph{Differential diagnosis}

\begin{itemize}
	\item
	diffuse cerebellar atrophy: progressive loss of volume
	\item
	Blake pouch cyst
	\item
	mega cisterna magna
	\item
	arachnoid cyst in the posterior fossa
\end{itemize}