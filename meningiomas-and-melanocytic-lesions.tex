\chapter{Meningiomas and melanocytic lesions}

\subsection{Meningioma}

\textbf{Meningiomas} are extra-axial tumors and represent the most common tumor of the meninges. They are a non-glial neoplasm that originates from the meningocytes or arachnoid cap cells of the meninges and are located anywhere that meninges are found and in some places where only rest cells are presumed to be located.

Although they are usually easily diagnosed and are typically indolent with a low rate of recurrence following surgery, there are 15 subtypes with variable imaging features and, in some instances, more aggressive biological behavior and higher grades.

Typical meningiomas appear as dural-based masses isointense to grey matter on both T1 and T2 weighted imaging, enhancing vividly on both MRI and CT. Some of the subtypes can vary dramatically in their imaging appearance.

This article is a general discussion of meningioma, focusing on typical primary intradural meningiomas and the imaging findings of intracranial disease.

Spinal meningioma and primary extradural meningioma as well as some of the various subtypes are discussed separately.

\paragraph{Terminology}

When describing meningiomas, a variety of terms can be used to more accurately describe these common tumors.

Most commonly they are either classified according to the histological subtype (e.g. rhabdoid or papillary etc.),location (e.g. skull base, spinal, intraosseous,intraventricular, etc.), and by etiology (e.g. radiation-induced, etc.).

A broad division of meningiomas into primary intradural (which may or may not have a secondary extradural extension) and primary extradural is also used, although the latter is rare accounting for only 1-2\% of cases .Ectopic primary meningiomas include tumors residing in the head and neck, orbit, nose, paranasal sinus, oropharynx and even more remotely (e.g.lung).

\paragraph{Epidemiology}

Meningiomas are more common in women, with a ratio of 2:1 intracranially and 4:1 in the spine. Atypical and malignant meningiomas are slightly more common in males. They are uncommon in patients before the age of 40 and should raise suspicion of neurofibromatosis type 2when found in young patients.

\paragraph{Clinical presentation}

Many small meningiomas are found incidentally and are entirely asymptomatic. Often they cause concern as they are mistakenly deemed to be the cause of vague symptoms, most frequently headaches. Larger tumors or those with adjacent edema or abutting particularly sensitive structures can present with a variety of symptoms. Most common presentations include :

\begin{itemize}
	\item
	headache: 36\%
	\item
	paresis: 22\%
	\item
	change in mental status: 21\%
\end{itemize}

Meningiomas may also become clinically apparent due to mass effect depending on their location:

\begin{itemize}
	\item
	supratentorial:85-90\%
	
	\begin{itemize}
		\item
		parasagittal, convexities: 45\%
		
		\begin{itemize}
			\item
			seizures and hemiparesis
		\end{itemize}
		\item
		sphenoid ridge:15-20\%
		\item
		olfactory groove/planum sphenoidale: 10\%
		
		\begin{itemize}
			\item
			anosmia (usually not recognized)
			\item
			Foster Kennedy syndrome
		\end{itemize}
		\item
		juxtasellar:5-10\%
		
		\begin{itemize}
			\item
			visual field defects
			\item
			cranial nerve deficits
		\end{itemize}
	\end{itemize}
	\item
	infratentorial:5-10\%
	
	\begin{itemize}
		\item
		obstructive hydrocephalus
		\item
		cranial nerve deficits
	\end{itemize}
	\item
	miscellaneous intradural:\textless5\%
	
	\begin{itemize}
		\item
		intraventricular meningioma
		\item
		optic nerve meningioma
		\item
		pineal gland
		
		\begin{itemize}
			\item
			Parinaud syndrome
			\item
			obstructive hydrocephalus
		\end{itemize}
	\end{itemize}
\end{itemize}

Occasionally transosseous or intraosseous involvement with prominent hyperostosis may result in local mass effect (e.g.proptosis).

Although dural venous sinus invasion and occlusion does occur, it usually occurs very gradually. Therefore most cases of venous invasion are asymptomatic as collateral veins have had time to enlarge.


\paragraph{Pathology}

Meningiomas are thought to arise from meningocytes or arachnoid cap cells, which themselves arise from pluripotent mesenchymal progenitor cells, which accounts for the unusual location of primary extradural tumors .

Although the majority of tumors are sporadic, they are also seen in the setting of previous cranial irradiation and of course in patients with neurofibromatosis type 2(Merlin gene on Chromosome 22). Additionally, meningiomas demonstrate estrogen and progesterone sensitivity and may grow during pregnancy.


\subparagraph{Subtypes}

In the 5th Edition (2021)WHO classification of CNS tumorsa total of 15 subtypes of meningioma are recognized.

\begin{itemize}
	\item
	angiomatous meningioma
	\item
	atypical meningioma: grade 2
	\item
	anaplastic (malignant) meningioma: grade 3
	\item
	chordoid meningioma: grade 2
	\item
	clear cell meningioma: grade 2
	\item
	fibrous meningioma(7\%)
	\item
	lymphoplasmacytic-rich meningioma
	\item
	meningothelial meningioma(17\%)
	\item
	metaplastic meningioma
	\item
	microcystic meningioma
	\item
	papillary meningioma: usually more aggressive behavior
	\item
	psammomatous meningioma
	\item
	rhabdoid meningioma: usually more aggressive behavior
	\item
	secretory meningioma
	\item
	\hspace{0pt}transitional meningioma(40\%):mixed histology, typically containing meningothelial and fibrous components
\end{itemize}


\subparagraph{Grading}

Unlike other tumors, the term "atypical" and "anaplastic"/"malignant" have been retained as histological subtypes with grade 2 and grade 3 tumors respectively .

Otherwise, meningiomas are graded from grade 1 to 3 based on histological features (e.g. mitotic index) some histological subtypes (e.g. chordoid meningiomas and clear cell meningiomas) and molecular features (see below).

An important change in the 5th Edition (2021) WHO classification of CNS tumors is that the identification of some histological subtypes (e.g. papillary meningiomas and rhabdoid meningiomas) no longer is sufficient to denote a higher grade .

Grade 2 criteria

\begin{itemize}
	\item
	increased mitotic figures: 4 to 19 in 10 consecutive high power fields (HPF)
	\item
	brain invasion (see below)
	\item
	chordoid or clear cell histological subtype
	\item
	three or more of the following:
	
	\begin{itemize}
		\item
		increased cellularity
		\item
		prominent nucleoli
		\item
		necrosis
		\item
		sheet-like growth
		\item
		small cells with high nuclear to cytoplasmic ratio
	\end{itemize}
\end{itemize}

Grade 3 criteria

\begin{itemize}
	\item
	increased mitotic figures: ≥20 in 10 consecutive high power fields (HPF)
	\item
	homozygous deletion of CDKN2A/B
	\item
	sarcoma or carcinoma or melanoma-like appearance
	\item
	TERT promoter mutation
\end{itemize}

Brain invasion

Brain invasion as a stand-alone feature remains controversial. In prior editions of the WHO classification (e.g. 2016) if a meningioma (regardless of histology) demonstrated any brain invasion it was designated as grade 2 as it was believed to denote a poorer prognosis with a higher likelihood of recurrence . In many instances, growth is actually along perivascular spaces rather than truly into the brain parenchyma. The 5th Edition has backed away from this dogmatic recommendation recognizing the difficulty in assessing this in some instances .Nonetheless, overt brain invasion remains sufficient to denote a grade 2 tumor.

\subparagraph{Macroscopic features}

In general, there are two main macroscopic forms easily recognized in imaging studies:

\begin{itemize}
	\item
	globose: rounded, well defined dural masses, likened to the appearance of a fried egg seen in profile (the most common presentation)
	\item
	en plaque: extensive regions of dural thickening
\end{itemize}

The cut surface reflects the various histologies encountered, ranging from very soft to extremely firm in fibrous or calcified tumors. They are usually light tan in coloring, although again this will depend on histological subtypes.

\subparagraph{Molecular markers}

Increasingly molecular markers are being incorporated into the diagnosis and grading of meningioma subtypes .

\begin{itemize}
	\item
	SMARCE1 mutations: clear cell subtype
	\item
	BAP1 mutations: papillary and rhabdoid subtypes
	\item
	KLF4/TRAF7 mutations: secretory subtype
	\item
	TERT promoter mutation: grade 3
	\item
	homozygous deletion of CDKN2A/B: grade 3
	\item
	H3K27me3 loss of nuclear expression: worse prognosis
	\item
	methylome profiling: prognostic subtyping
\end{itemize}

\paragraph{Radiographic features}

Meningiomas are best imaged with MRI with contrast as this most accurately delineates the tumor, presence of intra- and trans-osseous extension and relationship to the underlying brain. CT, however, is useful if bony anatomy is required (e.g. at the base of skull), when patients cannot have MRI, and especially when the meningioma is entirely ossified/calcified (see burnt-out meningioma).

Note that in addition to histological variants, many of which have less-typical imaging appearances, a number of 'special examples' of meningiomas are best discussed separately. These include:

\begin{itemize}
	\item
	burnt-out meningioma
	\item
	cystic meningiomas
	\item
	intraosseous meningioma
	\item
	intraventricular meningioma
	\item
	optic nerve sheath meningioma
	\item
	radiation-induced meningioma
\end{itemize}

The remainder of this section focuses on more typical imaging appearances of run-of-the-mill meningiomas.


\subparagraph{Plain radiograph}

Plain films no longer have a role in the diagnosis or management of meningiomas. Historically a number of features were observed, including:

\begin{itemize}
	\item
	enlarged meningeal artery grooves
	\item
	hyperostosis or lytic regions
	\item
	calcification
	\item
	displacement of calcified pineal gland/choroid plexus due to mass effect
\end{itemize}


\subparagraph{CT}

CT is often the first modality employed to investigate neurological signs or symptoms, and often is the modality which detects an incidental lesion:

\begin{itemize}
	\item
	non-contrast CT
	
	\begin{itemize}
		\item
		60\% slightly hyperdense to normal brain, the rest are more isodense
		\item
		20-30\% have some calcification 
		\item
		\textgreater50\% demonstrate variable adjacent edema (see below) 
	\end{itemize}
	\item
	post-contrast CT
	
	\begin{itemize}
		\item
		72\% brightly and homogeneously contrast enhance 
		\item
		malignant or cystic variants demonstrate more heterogeneity/less intense enhancement
	\end{itemize}
	\item
	hyperostosis(5\%) 
	
	\begin{itemize}
		\item
		typical for meningiomas that abut the base of the skull
		\item
		need to distinguish reactive hyperostosis from:
		
		\begin{itemize}
			\item
			direct skull vault invasion by adjacent meningioma
			\item
			primary intraosseous meningioma
		\end{itemize}
	\end{itemize}
	\item
	enlargement of the paranasal sinuses (pneumosinus dilatans) has also been suggested to be associated with anterior cranial fossa meningiomas 
	\item
	lytic/destructive regions are seen particularly in higher grade tumors but should make one suspect alternative pathology (e.g. hemangiopericytoma or metastasis)
\end{itemize}


\subparagraph{MRI}

As is the case with most other intracranial pathology, MRI is the investigation of choice for the diagnosis and characterization of meningiomas. When appearance and location are typical, the diagnosis can be made with a very high degree of certainty. In some instances, however, the appearances are atypical and careful interpretation is needed to make a correct preoperative diagnosis.

Meningiomas typically appear as extra-axial masses with a broad dural base. They are usually homogeneous and well-circumscribed, although many variants are encountered.It seems that the signal intensity of meningiomas on T2-weighted images correlates with the histological subtypes .

Signal characteristics

Signal characteristics of typical meningiomas include:

\begin{itemize}
	\item
	\textbf{T1}
	
	\begin{itemize}
		\item
		usually isointense to grey matter (60-90\%)
		\item
		hypointense to grey matter (10-40\%): particularly fibrous, psammomatous variants
	\end{itemize}
	\item
	\textbf{T1 C+ (Gd)}: usually intense and homogeneous enhancement
	\item
	\textbf{T2}
	
	\begin{itemize}
		\item
		usually isointense to grey matter (\textasciitilde50\%)
		\item
		hyperintense to grey matter (35-40\%)
		
		\begin{itemize}
			\item
			usually correlates with a soft texture and hypervascular tumors 
			\item
			seen in microcystic,secretory, cartilaginous (metaplastic),chordoidand angiomatous variants 
		\end{itemize}
		\item
		hypointense to grey matter (10-15\%): compared to grey matter and usually correlates with harder texture and more fibrous and calcified contents
	\end{itemize}
	\item
	\textbf{DWI/ADC}: grade 2 and 3 tumors may show greater than expected restricted diffusion although this is not universally useful in prospectively predicting histological grade 
	\item
	\textbf{MR spectroscopy}: \textbf{} usually does not play a significant role in diagnosis but can help distinguish meningiomas from mimics. Features include:
	
	\begin{itemize}
		\item
		increase in alanine (1.3-1.5 ppm)
		\item
		increased glutamine/glutamate
		\item
		increased choline (Cho): cellular tumor
		\item
		absent or significantly reduced N-acetylaspartate (NAA): non-neuronal origin
		\item
		absent or significantly reduced creatine (Cr)
	\end{itemize}
	\item
	\textbf{MR perfusion}: \textbf{} good correlation between volume transfer constant (k-trans) and histological grade 
	\item
	\textbf{MR tractography}: allows the identification of white matter tracts adjacent to the meningioma
	
	\begin{itemize}
		\item
		this may aid in preoperative planning for meningioma resection by allowing planning of a safer access route that would result in less residual functional iatrogenic deficits\hspace{0pt}
	\end{itemize}
\end{itemize}

Helpful imaging signs

A number of helpful imaging signs have been described,including:

\begin{itemize}
	\item
	CSF cleft sign, which is not specific for meningioma, but helps establish the mass to be extra-axial; loss of this can be seen in grade II and grade III which may suggest brain parenchyma invasion
	\item
	dural tailis seen in 60-72\%(note that a dural tail is also seen in other processes)
	\item
	sunburstor spoke-wheelappearance of the vessels
	\item
	white matter buckling sign
	\item
	arterial narrowing
	
	\begin{itemize}
		\item
		typically seen in meningiomas which encase arteries
		\item
		useful sign in parasellar tumors, in distinguishing a meningioma from a pituitary macroadenoma; the latter typically does not narrow vessels
	\end{itemize}
	\item
	peripheral rim of enhancement between meningioma and brain parenchima in post-contrast 3D-FLAIR can help in distinguishing meningioma from other dural based tumor
\end{itemize}

Edema

More than half of the meningiomas demonstrate a variable amount of vasogenic edema in adjacent brain parenchyma . Correlation between age, gender, tumor size,rapid growth,location (convexity and parasagittal \textgreater{} elsewhere), histologic type, and invasion in the case of malignant meningiomashave been suggested in literature but not yet confirmed. Although in general, the presence of severe adjacent edema is considered more compatible with aggressive meningiomas, in some histologically benign types such as secretory type, edema can be disproportionately larger than the small tumor size.

The underlying mechanism is most likely multifactorial however it has been shown that there is a strong association between the presence and severity of the peritumoral vasogenic edema (i.e. edema index) and expression of the vascular endothelial growth factor (VEGF) or expression of CEA and CK .

List of some of the proposed underlying mechanisms are:

\begin{itemize}
	\item
	venous stasis/occlusion/thrombosis
	\item
	compressive ischemia
	\item
	aggressive growth/invasion
	\item
	parasitization of pial vessels
	\item
	histologic subtype: secretory meningioma
	\item
	vascular endothelial growth factor (VEGF): produced within the meningioma that enters the adjacent parenchyma
	\item
	expression of CEA and CK
\end{itemize}


\subparagraph{Angiography (DSA)}

Catheter angiography is rarely now of diagnostic use but rather is performed for preoperative embolization to reduce intraoperative blood loss and alleviate resection of a tumor. This is especially useful for skull base tumors, or those thought to be particularly vascular (e.g. microcystic variants or those with very large vessels). Particles are favored typically 7-9 days prior to surgery although they are not free of complication, particularly one study showed a high prevalence of complications associated with particles smaller than 45-150 μm, so risks and benefits should be thoroughly assessed .

Meningiomas can have a dual blood supply. The majority of tumors are predominantly supplied by meningeal vessels; these are responsible for the sunburstor spoke-wheelpattern observed on MRI/DSA. Some tumors also have a significant pial supply to the periphery of a tumor.

A well known angiographic sign of meningiomas is the mother-in-law sign, in which the tumor contrast blush "comes early, stays late, and is very dense".


\paragraph{Treatment and prognosis}

Treatment is usually with surgical excision. If only incomplete resection is possible (especially at the base of the skull) then external-beam radiation therapy (or even brachytherapy)can be used .Radiation has been shown to improve local control and prolongs overall survival .

No widespread chemotherapeutic/systemic therapy has been proven to be efficacious although some mTOR inhibitor and antiangiogenic treatments show promise .

The Simpson gradecorrelates the degree of surgical resection completeness with symptomatic recurrence rate which also varies with grade and length of follow-up . Metastatic disease is rare but has been reported .

\paragraph{Differential diagnosis}

The differential diagnosis generally includes other dural masses as well as some location-specific entities.

The main dural masses to consider include:

\begin{itemize}
	\item
	solitary fibrous tumors of the dura
	
	\begin{itemize}
		\item
		more aggressive often destroying bone
		\item
		extensive peripheral vascularity
		\item
		more microlobulation
	\end{itemize}
	\item
	dural metastases(e.g. breast cancer)
	\item
	for other less common differentials see dural masses
\end{itemize}

Specific location differentials include:

\begin{itemize}
	\item
	cerebellopontine angle
	
	\begin{itemize}
		\item
		acoustic schwannoma
	\end{itemize}
	\item
	pituitary region
	
	\begin{itemize}
		\item
		pituitary macroadenoma
		\item
		craniopharyngioma
	\end{itemize}
	\item
	base of the skull
	
	\begin{itemize}
		\item
		hypertrophic pachymeningitis
		\item
		extramedullary hematopoiesis
		\item
		chondrosarcoma
		\item
		chordoma
	\end{itemize}
\end{itemize}

In the setting of hyperostosis consider:

\begin{itemize}
	\item
	Paget's disease
	\item
	fibrous dysplasia
	\item
	sclerotic metastases (e.g. prostate and breast carcinoma)
\end{itemize}
\subsection{Intraventricular meningioma}

\textbf{Intraventricular meningiomas} are rare tumors usually encountered in adults and are somewhat distinct from the far more common extra-ventricular meningioma.

On imaging, they classically present as vividly enhancing solid mass at the trigone of the lateral ventricles.

\paragraph{Epidemiology}

Intraventricular meningiomas are rare, accounting for only 0.5-3\% of all meningiomas . Nonetheless, because of the overall rarity of intraventricular tumors after childhood, they account for 10-15\% of all intraventricular neoplasmin adults .

Most intraventricular meningiomas present between the 3and 6 decades with a recognized female predilection (M:F ratio of 1:2). They are rare in childhood .

\paragraph{Clinical presentation}

Intraventricular meningiomas present usually due to mass effect, either by direct compression of the adjacent brain or from obstruction to normal CSF drainage with resultant hydrocephalus.

\paragraph{Pathology}

Intraventricular meningiomas are thought to arise from mengingothelial inclusion bodies located in the tela choroidea and/or mesenchymal stroma of the choroid plexus . In general, these meningiomas are most commonly of the fibrous meningiomas.

\subparagraph{Location}

\begin{itemize}
	\tightlist
	\item
	80\% trigone of the lateral ventricle
	\item
	15\%third ventricle
	\item
	5\%fourth ventricle
\end{itemize}

\paragraph{Radiographic features}

Their signal and attenuation characteristics are the same as other meningiomas, demonstrating essentially isodensity and intensity to grey matter precontrast and vivid, usually homogeneous enhancement following administration of contrast. Compared to extra-axial meningiomas, a greater proportion is calcified (50\% compared to 20\% for standard meningioma).

The vascular supply depends on the location but generally is from the arterial supply of the adjacent choroid plexus .

For further discussion of the radiographic appearances of these tumors, refer to the general article: meningioma.

\paragraph{Treatment and prognosis}

As is the case with other meningiomas, provided complete excision is possible, surgical excision is curative and therefore the treatment of choice. Recurrence rate following resection is fairly low, \textasciitilde5\% .


\paragraph{Differential diagnosis}

The differential somewhat depends on the location of the tumor and age of the patient, however, in general considerations should include:

\begin{itemize}
	\tightlist
	\item
	glial tumor
	
	\begin{itemize}
		\tightlist
		\item
		ependymoma
		\item
		astrocytoma
	\end{itemize}
	\item
	choroid plexus metastases
	
	\begin{itemize}
		\tightlist
		\item
		renal cell carcinoma
		\item
		melanoma
	\end{itemize}
	\item
	choroid plexus papilloma (particularly in children)
	\item
	CNS lymphoma
	\item
	central neurocytoma
\end{itemize}