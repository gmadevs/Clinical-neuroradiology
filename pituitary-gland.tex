\chapter{Pituitary gland}
\subsection{Lymphocytic hypophysitis}

\textbf{Lymphocytic hypophysitis} is an uncommon non-neoplastic inflammatory condition that affects the pituitary gland. It is closely related to other inflammatory conditions in the region, namely orbital pseudotumor and Tolosa-Hunt syndrome.

\paragraph{Epidemiology}

Lymphocytic hypophysitis is seen most frequently in women, with a F:M of \textasciitilde9:1, and often in the postpartum period or the third trimester of pregnancy.

\subparagraph{Associations}

\begin{itemize}
	\item
	autoimmune conditions such as
	
	\begin{itemize}
		\item
		autoimmune thyroiditis, e.g. Hashimoto thyroiditis, Graves disease and subacute thyroiditis
		\item
		autoimmune adrenalitis
		\item
		pernicious anemia
		\item
		type 1 diabetes mellitus
		\item
		vitiligo
		\item
		focal lymphocytic parathyroiditis
		\item
		rheumatoid disease
		\item
		systemic lupus erythematosus (SLE)
	\end{itemize}
	\item
	immune checkpoint inhibitors
	
	\begin{itemize}
		\item
		more common with CTLA4 inhibitors (e.g. ipilimumab) than PD-1 or PD-L1 inhibitors 
	\end{itemize}
\end{itemize}

\paragraph{Clinical presentation}

Clinical presentation is varied depending on the part of the pituitary affected and the size of the lesion. Lymphocytic hypophysitis can thus be classified as:

\begin{itemize}
	\item
	anterior pituitary: lymphocytic adenohypophysitis
	
	\begin{itemize}
		\item
		most common
		\item
		mimics a pituitary adenoma
		\item
		endocrine hormone deficits are common, including hypopituitarism
		\item
		mass effects on adjacent structures (e.g. optic chiasm)
	\end{itemize}
	\item
	posterior pituitary:lymphocytic infundibular neurohypophysitis
	
	\begin{itemize}
		\item
		rare
		\item
		diabetes insipidus
	\end{itemize}
	\item
	both anterior and posterior pituitary:lymphocytic infundibular panhypophysitis
\end{itemize}

\paragraph{Pathology}

It is characterized by infiltration of the pituitary stalkwith lymphocytes, as the name would suggest. Importantly, there is a paucity of plasma cells or granulomas, differentiating it from IgG4-related hypophysitis and granulomatous hypophysitis(e.g. due to neurosarcoidosis), respectively.

\paragraph{Radiographic features}


\subparagraph{CT}

Coronal CT and multiplanar reconstructions can visualize the pituitary region reasonably well. Lymphocytic hypophysitis appears as an enhancing soft tissue mass involving the pituitary and extending into the suprasellar region.


\subparagraph{MRI}

MRI, as is the case with other pituitary lesions, is the best modality for assessing this condition which appears as a pituitary region mass.

\begin{itemize}
	\item
	\textbf{T1}
	
	\begin{itemize}
		\item
		affected area is isointense with slight signal heterogeneity
		\item
		normal posterior pituitary bright spot may be absent 
	\end{itemize}
	\item
	\textbf{T1 C+ (Gd)}
	
	\begin{itemize}
		\item
		can variably enhance, usually homogeneously 
		\item
		dural enhancement may be present 
		\item
		infundibulum may be thickened 
	\end{itemize}
	\item
	\textbf{T2}
	
	\begin{itemize}
		\item
		hypointensity in parasellar region (parasellar dark T2 sign) can be present and may be useful in differentiating from a pituitary adenoma 
	\end{itemize}
\end{itemize}


\paragraph{Treatment and prognosis}

Lymphocytic hypophysitis is usually self-limiting and spontaneous recovery can occur. Corticosteroids are sometimes given and deficient hormones can be replaced . In patients on immune checkpoint inhibitors, consideration should be given towards stopping this therapy, either temporarily or permanently .


\paragraph{Differential diagnosis}

The differential diagnosis is primarily that of other pituitary region masses. Considerations include:

\begin{itemize}
	\item
	pituitary adenoma
	
	\begin{itemize}
		\item
		macroadenomas are expected to enlarge the sella turcica
	\end{itemize}
	\item
	craniopharyngioma (papillary type)
	\item
	suprasellar meningioma
	
	\begin{itemize}
		\item
		dural-based
		\item
		usually follows the cerebral cortex intensity
	\end{itemize}
	\item
	pituitary metastasis
	\item
	Langerhans cell histiocytosis (LCH)
	\item
	IgG4-related hypophysitis
	\item
	granulomatous hypophysitis(idiopathic or secondary to systemic illness e.g.sarcoidosis, syphilis, and tuberculosis)
	\item
	xanthomatous hypophysitis
	\item
	necrotizing hypophysitis
\end{itemize}