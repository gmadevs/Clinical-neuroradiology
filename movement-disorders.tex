\chapter{Movement disorders}

\section{Gluten ataxia}

\subsubsection{Field}

\textbf{Gluten ataxia} is a relatively common central nervous system manifestation of celiac disease and is usually encountered in individuals who do not have overt gastrointestinal symptoms.

\paragraph{Epidemiology}

Gluten ataxia is encountered in both pediatric and adult celiac populations. It is a fairly common cause of all sporadic ataxias .

\paragraph{Clinical presentation}

Gluten ataxia presents with cerebellar ataxia, sometimes with concurrent sensory ataxia, primarily affecting the lower limbs and gait . A less common manifestation is ataxia with myoclonus .

Gastrointestinal symptoms are usually absent, found in only 10\% of individuals, and on duodenal biopsy, a diagnosis of celiac disease can be made in on half of patients . As such, the diagnosis of gluten ataxia requires screening for antibodies associated with celiac disease:anti-gliadin, anti-EMA, anti-TG2, and anti-TG6 antibodies .

\paragraph{Pathology}

There is some evidence that gluten-dependent transglutaminase 6 (TG6) autoantibodies react against cells within the cerebellum .

\paragraph{Radiographic features}

The main imaging finding is that of cerebellar atrophy that is usually gradual but in some cases can be rapid .

\paragraph{Treatment and prognosis}

Treatment is primarily with strict adherence to a gluten-free diet. In many individuals, objective and subjective improvement in ataxia can be demonstrated.

\begin{tcolorbox}[colback=green!5!white,colframe=green!75!white,title=Differential diagnosis]
\begin{itemize}
	\tightlist
	\item
	spinocerebellar ataxias
	\item
	multisystem atrophy - cerebellum (MSA-C)
	
	\begin{itemize}
		\tightlist
		\item
		also has extrapyramidal and autonomic features 
	\end{itemize}
\end{itemize}
\end{tcolorbox}

\section{Cerebellar ataxia with neuropathy and vestibular areflexia syndrome (CANVAS)}

\textbf{Cerebellar ataxia with neuropathy and vestibular areflexia syndrome (CANVAS)} is a rare neurodegenerative balance disorder and \emph{RFC1}-related disease characterized by cerebellar ataxia, sensory neuronopathy (ganglionopathy), and bilateral vestibular hypofunction.

\paragraph{Epidemiology}

The epidemiology is yet to be defined, but CANVAS is thought to be rare. It often first manifests in middle-aged adults .

\paragraph{Clinical presentation}

The diagnosis of CANVAS is challenging because patients will typically have a nonspecific presentation, most commonly with gait imbalance and falls, often worse in the dark . Less commonly, they may present with dysesthesias or oscillopsia .

On physical examination, each of the three hallmark features of CANVAS can be demonstrated :

\begin{itemize}
	\item
	cerebellar ataxia: saccadic smooth pursuit, nystagmus, limb ataxia, cerebellar dysarthria
	\item
	non-length-dependent sensory neuronopathy (ganglionopathy): deficits of pin-prick sensation (most common) or other sensory modalities, absent ankle jerk reflexes, abnormal corneal reflex, abnormal jaw jerk reflex
	
	\begin{itemize}
		\item
		nerve conduction studies are more sensitive to subtle abnormalities
	\end{itemize}
	\item
	bilateral vestibulopathy: bilateral abnormal tests of the vestibulo-ocular reflex (e.g. bidirectionally abnormal head impulse test, abnormal dynamic visual acuity)
	
	\begin{itemize}
		\item
		the video head impulse test is more sensitive to subtle abnormalities
		\item
		hearing is unaffected
	\end{itemize}
\end{itemize}

A particularly useful physical sign in CANVAS is that of an abnormal visually enhanced vestibulo-ocular reflex (also known as the doll's eye reflex or oculo-cephalic reflex), which can only occur if both cerebellar ataxia and bilateral vestibulopathy are present , two of the hallmark features of CANVAS. The abnormality can be diagnosed clinically at the bedside or using objective measures such as video-oculography .

In addition to the classic triad, other commonly associated signs and symptoms include:

\begin{itemize}
	\item
	postural hypotension 
	\item
	dysphagia 
	\item
	chronic cough 
\end{itemize}

\paragraph{Pathology}

The pathophysiology of CANVAS is yet to be fully elucidated.There may be a genetic component implicated in CANVAS, with a pentanucleotide repeat in replication factor complex subunit 1 (\emph{RFC1}) having been recognized in some patients with the disease . However, the pathogenic mechanism of this expansion is unclear .

Histopathologically and electrophysiologically, there is evidence that CANVAS is characterized by a ganglionopathy resulting in bilateral vestibulopathy (i.e. Scarpa's ganglion involvement) and sensory neuronopathy (i.e. dorsal root ganglion involvement), as well as subclinical involvement of other ganglia such as the geniculate and trigeminal ganglia . In the cerebellum, there is pathological evidence of vermian atrophy, with notable loss of vermian Purkinje cells .

\subparagraph{Genetics}

Familial and sporadic cases are usually caused by biallelic intronic AAGGG repeat expansions in the gene \emph{RFC1}(replication factor complex subunit 1) . Additionally, another pentanucleotide repeat expansion in \emph{RFC1}, ACAGG, has also been described in Asian-Pacific patients with CANVAS , but this is considered rare.

Other patients carrying this biallelic repeat expansion may have incomplete CANVAS phenotypes, such as just a sensory neuronopathy or pure cerebellar ataxia . Thus, it is likely that CANVAS exists on a spectrum known as \textbf{\emph{RFC1}-related disease} .

\paragraph{Radiographic features}

\subparagraph{MRI}

MRI brain is the imaging investigation of choice . Evidence of MRI changes in CANVAS can be nonspecific but are usually present . Typically, there will be focal cerebellar atrophy, with particular involvement of the vermian lobules VI, VIIA, and VIIB, as well as hemispheric cerebellar atrophy of crus I (vermian lobule VII) . Additionally, but less commonly, spinal cord atrophy may also be present .

\subparagraph{Nuclear medicine}

PET-CT

FDG-PET brain shows cerebellar hypometabolism .

SPECT

I-123 ioflupane SPECTshows decreased striatal uptake of radiotracer, reflecting the involvement of dopaminergic neurons .

Other

I-123 metaiodobenzylguanidinescintigraphy shows decreased cardiac uptake of radiotracer, reflecting the involvement of cardiac sympathetic nerves .

\paragraph{Treatment and prognosis}

There is no disease-modifying therapy available for CANVAS. Management focuses on symptom-specific management such as vestibular rehabilitation, speech pathology monitoring and management of dysphagia, and neuropathic pain management (e.g. pregabalin) . The prognosis varies, but CANVAS is generally considered to be a slowly progressive condition over decades .

\paragraph{Differential diagnosis}

Clinical differential diagnoses include various causes of adult-onset ataxia :

\begin{itemize}
	\item
	adult-onset Friedreich ataxia
	\item
	spinocerebellar ataxias (especially spinocerebellar ataxia type 3)
	\item
	multiple system atrophy (cerebellar type)
	\item
	Wernicke encephalopathy
	\item
	superficial siderosis of the CNS
	\item
	neurosarcoidosis
\end{itemize}