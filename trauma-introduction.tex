\chapter{Trauma introduction}
\minitoc

Head trauma represents a significant and often devastating cause of morbidity and mortality worldwide, impacting individuals across all age groups. As neuroradiologists and radiologists-in-training, a thorough understanding of the imaging manifestations of head injuries is paramount. Timely and accurate interpretation of neuroimaging is critical for patient triage, guiding immediate management decisions, and informing prognosis.

This chapter will provide a comprehensive overview of head trauma from an imaging perspective, focusing on the essential knowledge required for residents and new specialists. We will explore the various mechanisms of injury, the spectrum of primary and secondary brain insults, and the characteristic imaging features across different modalities. While Computed Tomography (CT) remains the cornerstone for acute evaluation dueating its speed and ability to detect emergent conditions like hemorrhage and fractures, we will also discuss the evolving role of Magnetic Resonance Imaging (MRI) for identifying more subtle injuries and assessing long-term sequelae.

By systematically reviewing skull fractures, extra-axial hemorrhages (epidural, subdural, subarachnoid, intraventricular), intra-axial injuries (contusions, diffuse axonal injury), and secondary complications such as cerebral edema and herniation, this chapter aims to equip you with the diagnostic confidence needed to navigate the complexities of head trauma imaging in your clinical practice.