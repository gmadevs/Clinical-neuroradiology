\chapter{Introduzione}
\section{Mielinizzazione}
È un processo dinamico che inizia nella vita fetale e continua dopo la nascita fino al terzo anno.

Gli oligodendrociti formano la mielina, causando un aumento dei lipidi cerebrali e una riduzione dell'acqua.

Questo causa cambiamenti nel segnale RM, con riduzione dei tempi T1 e T2. I lipidi depositati aumentano il segnale T1, mentre la riduzione dell'acqua diminuisce il segnale T2.

Nel neonato, la sostanza bianca non mielinizzata ha segnale più basso in T1 e più alto in T2 rispetto alla sostanza grigia. Con la maturazione, questo pattern si inverte fino a raggiungere l'aspetto adulto (sostanza bianca: alto T1, basso T2; sostanza grigia: basso T1, alto T2). Il pattern adulto si completa in T1 a 12-15 mesi e in T2 a 3 anni.

L'aumento di volume è particolarmente evidente nel corpo calloso, il principale sistema commissurale del cervello. Lo spessore adulto viene raggiunto intorno ai 15 mesi di età, sebbene un incremento molto lieve possa persistere fino ai 20-25 anni.

Le immagini T2-pesate fluid attenuated inversion recovery (FLAIR) sono meno sensibili ai cambiamenti mielinici e andrebbero evitate nella valutazione della mielina prima del completamento della mielinizzazione

A 36 settimane, la mielina diventa evidente nelle regioni tipiche del neonato a termine (braccio posteriore della capsula interna, corona radiata e tratti corticospinali). Dopo 37 settimane di età gestazionale, il tronco encefalico dorsale appare diffusamente mielinizzato, probabilmente come combinazione di nuclei e tratti di sostanza bianca, anche se non è più possibile delineare strutture specifiche. Durante il primo anno postnatale, la mielina si diffonde in tutto il cervello secondo uno schema preordinato di sequenze cronologiche e topografiche. La mielinizzazione procede in modo centrifugo, dal basso verso l'alto e dal posteriore all'anteriore. Nel tronco encefalico, avanza dalle aree dorsali a quelle ventrali. Negli emisferi cerebrali, si sviluppa dal solco centrale verso il polo e dai lobi occipitale e parietale verso i lobi frontale e temporale. Le fibre sensitive si mielinizzano prima delle fibre motorie, e le vie di proiezione precedono le vie associative; i tratti sensitivi, visivi e uditivi sono già mielinizzati alla nascita. A termine, la mielinizzazione è evidente nei flocculi cerebellari, nei peduncoli cerebellari inferiori e superiori, nel verme cerebellare, nei nuclei dentati, nel tronco encefalico dorsale (nuclei dei nervi cranici e tratti sensitivi), nella decussazione dei peduncoli cerebellari superiori, nei nuclei ventroposterolaterali del talamo, nei globi pallidi, nel putamen posteriore, nella porzione posteriore del braccio posteriore delle capsule interne, nella corona radiata centrale e nei giri pre- e postcentrali

A 3 mesi di età, la mielinizzazione è visibile sia nelle immagini T1 che T2-pesate nei peduncoli cerebellari, nella sostanza bianca cerebellare profonda e nella radiazione ottica. Il braccio anteriore della capsula interna, lo splenio del corpo calloso e la sostanza bianca centrale sottocorticale appaiono mielinizzati solo nelle immagini T1-pesate. A questa età diventa evidente la diversa tempistica della comparsa della mielinizzazione nelle sequenze T1 e T2-pesate. Le ultime aree a mielinizzarsi (cosiddette zone terminali) sono comunemente considerate le aree sottocorticali dei lobi frontale e temporale.