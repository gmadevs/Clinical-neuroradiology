\section*{Lista abbreviazioni}
\begin{description}
	\item[DII] Danno ipossico-ischemico
\end{description}

\section{Danno ipossico-ischemico}

\subsection{Definizione}
Il danno ipossico-ischemico (DII) si riferisce a una combinazione di lesioni ipossiche e ipoperfusione cerebrale. Questa condizione è una delle cause più comuni di paralisi cerebrale infantile e di gravi deficit neurologici nei bambini, con un'incidenza di 2-9 su 1000 nati vivi.

\subsection{Eziologia}
L'ipossia-ischemia, una delle principali cause di encefalopatia neonatale, si verifica quando il flusso sanguigno al cervello è ridotto (ischemia) e l'ossigenazione del sangue è compromessa (ipossiemia). Questa condizione può portare a gravi conseguenze neurologiche fino alla morte.
Le cause alla base del danno ipossico-ischemico sono varie, e possono essere raggruppate in base al momento in cui si verificano:

\begin{itemize}
	\tightlist
	\item
	\textbf{Fattori antepartum:} includono condizioni materne come ipotensione, trattamenti per l'infertilità, gravidanze multiple, infezioni prenatali e malattie della tiroide.
	\item
	\textbf{Fattori intrapartum:} si riferiscono a complicanze durante il travaglio, come parto con forcipe, estrazione podalica, prolasso del cordone ombelicale, distacco della placenta, cordone ombelicale stretto attorno al collo e febbre materna.
	\item
	\textbf{Fattori postpartum:} comprendono complicanze che si manifestano dopo la nascita, come distress respiratorio grave, sepsi o shock.
\end{itemize}

Indipendentemente dalla causa specifica, i processi fisiopatologici sottostanti al danno ipossico-ischemico sono simili e includono una cascata di eventi:

\begin{itemize}
	\tightlist
	\item
	\textbf{Riduzione del flusso sanguigno cerebrale (ischemia):} porta a un passaggio dal metabolismo aerobico a quello anaerobico, causando una rapida deplezione di ATP e un accumulo di lattato nelle cellule.
	\item
	\textbf{Rilascio di neurotrasmettitori eccitatori:} come il glutammato, che attiva i recettori NMDA, provocando un afflusso di calcio nei neuroni postsinaptici.
	\item
	\textbf{Formazione di radicali liberi:} che danneggiano i costituenti cellulari, come i mitocondri, peggiorando la produzione di ATP e portando alla morte cellulare.
	\item
	\textbf{Edema citotossico:} causato dallo stress osmotico, con conseguente ridistribuzione dell'acqua dallo spazio extracellulare a quello intracellulare.
\end{itemize}

Inoltre, è importante considerare che la maturità del cervello neonatale al momento dell'insulto influisce sul tipo e sulla gravità del danno.

\subsection{Epidemiologia}

L'encefalopatia ipossico-ischemica colpisce dall'1 al 6 ogni 1000 nati vivi. In particolare, nei paesi sviluppati, si stima un'incidenza di 1-3 casi ogni 1000 nati vivi. Nei neonati a termine, il DII è responsabile del 15-20\% della mortalità neonatale. Nei neonati pretermine nati prima della 32esima settimana di gestazione, la mortalità può arrivare fino al 50\%. Circa il 50\% dei casi di paralisi cerebrale sono osservati in neonati prematuri. Nonostante un calo di mortalità perinatale e materna, l'incidenza di paralisi cerebrale non è cambiata negli ultimi 40 anni. Tra il 15\% e il 20\% dei neonati che soffrono di HII muoiono nel periodo neonatale e un ulteriore 25\% sviluppa deficit neurologici permanenti.

\subsection{Presentazione clinica}
La presentazione clinica del danno ipossico-ischemico nei neonati può variare a seconda della gravità e della durata dell'insulto, nonché dell'età gestazionale del neonato al momento dell'evento. I segni e i sintomi possono essere aspecifici nelle prime ore di vita e tendono a evolvere nel corso dei giorni.
Nell'anamnesi perinatale si riscontra spesso un evento stressante, come un parto complicato, che porta a sospettare un DII. I neonati possono nascere con grave acidemia metabolica o mista, manifestata da un basso pH nel sangue del cordone ombelicale (inferiore a 7) e bassi punteggi di Apgar (inferiori a 3) a 0 e 5 minuti dopo la nascita. I segni neurologici includono depressione della coscienza, con neonati che mostrano un livello di coscienza ridotto; ipotonia, frequente soprattutto in caso di lesione delle regioni corticali, più raramente ipertono; riflessi vivaci in alcuni casi; e convulsioni, che si verificano nel 50-70\% dei neonati asfittici generalmente entro le prime 24 ore. Si possono anche osservare apnea e respiro periodico nei primi giorni dopo un grave insulto, oltre ad alterazioni dello stato di vigilanza come ipereccitabilità o prolungata veglia. 
I neonati possono inoltre presentare anomalie autonomiche, come bradicardia o tachicardia, e altri sintomi tra cui alterazioni metaboliche come ipoglicemia, ipocalcemia e squilibri idro-elettrolitici, oltre a possibili alterazioni renali, cardiache, polmonari e intestinali. L'elettroencefalogramma può risultare anomalo ed essere utile per prevedere l'esito clinico, inclusa la probabilità di morte e gravi sequele neurologiche a lungo termine. È importante notare che i sintomi possono comparire dopo un periodo iniziale in cui il neonato sembra relativamente normale. 
Va sottolineato che la gravità della presentazione clinica è correlata all'estensione e alla gravità del danno cerebrale, e che l'ipotermia terapeutica, se iniziata entro 6 ore dall'asfissia perinatale, può ridurre significativamente la mortalità e la morbilità, diminuendo l'incidenza di disabilità, paralisi cerebrale e ritardi dello sviluppo.
La forma lieve-moderata acuta (meno di 10 minuti) comporta asfissia parziale acuta o ipoperfusione è una forma comune durante il parto, solitamente senza significative conseguenze cliniche o di imaging; nella forma prolungata (15-25 minuti) si ha un'ipossia-ischemia parziale prolungata dove lo shunt mantiene il flusso sanguigno alle strutture vitali, con un pattern di lesione periferico o watershed; nella forma severa (profonda) acuta (meno di 10 minuti) si verifica un'ipossia-ischemia profonda acuta con shunt inadeguato, dove le regioni metabolicamente attive sono le più suscettibili, con un pattern di lesione dei gangli della base-talamo; infine, nella forma severa prolungata (15-25 minuti) si ha un'ipossia-ischemia profonda prolungata, catastrofica, che porta a una lesione cerebrale totale.

\subsection{Approccio diagnostico}

La risonanza magnetica andrebbe effettuata 1-5 giorni dopo l'evento ischemico.

\subsection{Anatomia patologica}

\subsection{Imaging}

\subsubsection{TC}

L'imaging tramite TC ha un ruolo limitato nella valutazione dell'encefalopatia ipossico-ischemica neonatale, soprattutto se confrontato con l'ecografia e la RM.
La TC è generalmente considerata poco sensibile nella valutazione del DII neonatale. In particolare, la TC ha diverse limitazioni: ha una bassa sensibilità nel rilevare danni alla sostanza bianca. La normale bassa attenuazione della sostanza bianca neonatale, dovuta all'elevato contenuto d'acqua, può mascherare l'edema o altri segni precoci di danno ipossico. Inoltre, la TC può non evidenziare lesioni di piccole dimensioni o sfumate, come quelle che si manifestano nella fase acuta o subacuta del'DII. Un altro limite importante è l'esposizione a radiazioni ionizzanti.

\subsubsection{RM}
Nelle immagini pesate in T1 ottenute in neonati sani a termine, un focus di iperintensità dovrebbe essere sempre visibile nel braccio posteriore della capsula interna e dovrebbe avere un'intensità di segnale relativamente più elevata rispetto al putamen posterolaterale. I talami ventrolaterali e i globi pallidi, in particolare negli aspetti mediali, dovrebbero avere un'intensità di segnale più elevata rispetto agli altri nuclei della sostanza grigia profonda. Nelle immagini pesate in T2  si dovrebbe sempre osservare un focus di ipointensità nel braccio posteriore delle capsule interne e nei talami ventrolaterali.

\paragraph{Pattern di danno} I pattern di danno ipossico-ischemico nel cervello neonatale variano significativamente a seconda dell'età gestazionale e della maturazione cerebrale al momento dell'insulto. La maturità del cervello influenza la sua vulnerabilità alle lesioni, determinando quali aree saranno maggiormente colpite. In generale, il DII non colpisce tutte le aree cerebrali uniformemente, ma mostra una "vulnerabilità selettiva", con alcune regioni più suscettibili al danno rispetto ad altre. I pattern di danno possono essere suddivisi in base all'età gestazionale del neonato:

\textbf{Neonati pretermine (meno di 36 settimane di gestazione)}:

\begin{itemize}
	\tightlist
	\item
	\textbf{Ipoperfusione lieve-moderata}: il danno si manifesta principalmente nella sostanza bianca periventricolare (PVL). Ciò è dovuto alla vascolarizzazione ventricolopeta del cervello immaturo, in cui i vasi sanguigni penetrano dalla superficie del cervello verso le regioni periventricolari. La PVL può evolvere con un quadro di necrosi, cavitazione e sviluppo di cisti, che con il tempo possono collassare con conseguente gliosi e riduzione del volume della sostanza bianca. La PVL si osserva più frequentemente in prossimità dei trigoni dei ventricoli laterali e dei forami di Monro.
	\item
	\textbf{Ipoperfusione grave}: le lesioni coinvolgono principalmente le strutture della sostanza grigia profonda, come il talamo e il tronco encefalico. In particolare, il talamo, il verme anteriore e il tronco encefalico dorsale sono le aree più frequentemente colpite. È stato osservato che, nei neonati nati prima delle 32 settimane, il coinvolgimento dei gangli della base è meno grave rispetto al talamo. I gangli della base tendono a cavitare e ridursi in dimensione senza segni di gliosi. In questi casi si possono osservare anche emorragie della matrice germinale.
\end{itemize}

\textbf{Neonati a termine (36 settimane di gestazione o più)}:

\begin{itemize}
	\tightlist
	\item
	\textbf{Ipoperfusione lieve-moderata}: si osserva un pattern di danno "periferico" o "a spartiacque" (watershed). Le lesioni interessano principalmente la corteccia cerebrale e la sostanza bianca sottocorticale nelle zone di confine tra i territori vascolari delle arterie cerebrali anteriore, media e posteriore. Le zone occipitali e i lobi temporali posteriori sono solitamente quelli più coinvolti con risparmio delle regioni anteriori. Queste zone sono più vulnerabili a causa della ridistribuzione del flusso sanguigno a favore delle aree metabolicamente più attive.
	\item
	\textbf{Ipoperfusione grave}: il pattern di danno è definito "dei gangli basali-talamo". Questo tipo di danno è associato ad eventi ipossici o ischemici più gravi e di breve durata. Le aree più colpite sono il talamo laterale, il putamen posteriore, l'ippocampo, i tratti cortico-spinali e la corteccia sensomotoria. Inoltre, si può osservare una perdita del normale segnale iperintenso nel braccio posteriore della capsula interna sulle immagini T1-pesate (segno del braccio posteriore assente).
\end{itemize}

\textbf{Fattori che influenzano il pattern di danno}:

\begin{itemize}
	\tightlist
	\item
	\textbf{Maturità cerebrale:} La maturità del cervello al momento dell'insulto è uno dei principali fattori che determinano il pattern di danno. Il cervello pretermine è più vulnerabile alla PVL a causa della predominanza di pre-oligodendrociti, mentre il cervello a termine è più suscettibile al danno corticale e sottocorticale. Le aree con maggiore mielinizzazione e attività metabolica tendono ad essere più vulnerabili al danno.
	\item
	\textbf{Gravità dell'insulto:} episodi di ipossia-ischemia grave tendono a causare modelli di lesione più diffusi, con coinvolgimento delle strutture della sostanza grigia profonda. In particolare, nei neonati pretermine, un insulto grave può coinvolgere il talamo, il tronco cerebrale e il cervelletto, mentre nei neonati a termine un insulto severo coinvolge il talamo laterale, il putamen posteriore, l'ippocampo e la corteccia sensomotoria. Insulti meno severi interessano maggiormente la sostanza bianca periventricolare nei pretermine e le aree di spartiacque nei neonati a termine.
	\item
	\textbf{Durata dell'insulto:} insulti di breve durata spesso non causano danni cerebrali, mentre eventi prolungati portano a lesioni più estese.
\end{itemize}

Inoltre, i modelli di lesione possono essere influenzati dai meccanismi di ridistribuzione del flusso ematico cerebrale. In caso di ipossia, il flusso sanguigno viene reindirizzato verso le aree più vitali e metabolicamente attive come il tronco cerebrale e il talamo, a scapito di altre aree.

Possono verificarsi infarti venosi periventricolari. Gli infarti venosi periventricolari sono secondari alla trombosi delle vene midollari che drenano il parenchima cerebrale periventricolare e sono solitamente identificati come lesioni triangolari unilaterali con emorragia interna sulle immagini coronali. I neonati prematuri con infarti venosi periventricolari che coinvolgono la regione peritrigonale con associata assenza di mielinizzazione negli arti posteriori della capsula interna hanno una maggiore probabilità di sviluppare emiplegia congenita.

Nel caso di danno anossico prolungato e severo in quadro di morte cerebrale si può osservare una diffusa iperintensità cerebrale con normale segnale del cervelletto, segno conosciuto come "white cerebrum sign".

\paragraph{DWI} Le lesioni in DWI possono evolvere nel tempo. Nella fase acuta, le lesioni possono essere caratterizzate da edema e restrizione della diffusione all'imaging DWI. Con il passare del tempo, può verificarsi un danno cellulare, con gliosi, atrofia e deposizione di minerali.

La \textbf{pseudonormalizzazione} nella risonanza magnetica con imaging DWI è un fenomeno che si verifica in seguito a un danno ipossico-ischemico nel neonato e si riferisce a un \textbf{apparente miglioramento o normalizzazione del segnale DWI}, nonostante la presenza sottostante di danno cerebrale. Questo fenomeno può portare a un'interpretazione errata dell'esame, se non compreso nel contesto dell'evoluzione temporale del danno.
La pseudonormalizzazione si manifesta tipicamente \textbf{circa una settimana dopo l'evento ipossico-ischemico}. Inizialmente, la DWI mostra un aumento del segnale nelle aree colpite a causa della restrizione della diffusione dell'acqua causata dall'edema citotossico. Successivamente, a distanza di circa una settimana, questo segnale elevato diminuisce o scompare, dando l'impressione di un miglioramento. La pseudonormalizzazione non indica una guarigione del tessuto cerebrale, bensì un \textbf{cambiamento nelle caratteristiche del danno}. L'edema citotossico iniziale, che causa la restrizione della diffusione, può risolversi o evolvere in altri tipi di edema (come l'edema vasogenico), o in processi di morte cellulare, portando a una normalizzazione del segnale DWI. Pertanto, mentre il segnale DWI può apparire normale, il danno cellulare sottostante persiste o si evolve. La pseudonormalizzazione può portare a \textbf{sottostimare l'estensione del danno} se si considera solo la DWI eseguita dopo una settimana dall'insulto. È importante interpretare le immagini DWI nel contesto della storia clinica e in correlazione con le altre sequenze di risonanza magnetica. Anche se le immagini DWI mostrano una pseudonormalizzazione, studi hanno evidenziato che i danni cerebrali possono persistere e portare a esiti negativi nello sviluppo neurologico. I valori di ADC (coefficiente di diffusione apparente) possono persistere diminuiti nelle aree danneggiate anche nella seconda settimana. In questi casi, la DTI può essere più utile per evidenziare il danno che persiste o si evolve nel tempo. L'analisi congiunta dei parametri di imaging derivati da DTI, come FA e MD, può aiutare a comprendere meglio l'evoluzione del danno anche in presenza di pseudonormalizzazione della DWI. Ad esempio, mentre i valori di ADC possono pseudonormalizzare, i valori di FA possono rimanere ridotti per un periodo di tempo più lungo, indicando la persistenza del danno strutturale. Una riduzione di FA è stata osservata nei casi di HIE moderata-grave nelle prime 3 settimane di vita, mentre una riduzione della MD si osserva nella sostanza bianca nei casi più gravi. A causa del fenomeno della pseudonormalizzazione, è \textbf{raccomandabile eseguire studi di follow-up con RM} per monitorare l'evoluzione del danno e la presenza di possibili alterazioni strutturali. Il follow-up, eseguito dopo la pseudonormalizzazione, può rivelare atrofia corticale, degenerazione cistica o altre conseguenze tardive.

\paragraph{SWI} SWI e T2*GE sono particolarmente utili nell'imaging del cervello neonatale per identificare microemorragie e per migliorare la valutazione del danno ipossico-ischemico dal punto di vista della tempistica e l'estensione del danno. Si possono evidenziare \textbf{emorragie della matrice germinale}, che sono più frequenti nei neonati pretermine. Le SWI rispetto alle sequenze T2*GE sono superiori nel rilevare la componente emorragica di tali lesioni. L'emorragia della matrice germinale può essere associata a emorragie intraventricolari e può complicare ulteriormente il decorso clinico del neonato. Sono utili nell'identificare la \textbf{trombosi venosa durale}, condizione che può associarsi a ischemia e danno cerebrale. La trombosi venosa si manifesta tipicamente con assenza di flusso nei seni venosi e, a volte, con emorragie correlate. In caso di emorragie pregresse, le sequenze GRE e SWI possono evidenziare la presenza di \textbf{depositi di emosiderina}, un prodotto di degradazione dell'emoglobina, che può persistere a lungo dopo l'evento emorragico. Le sequenze SWI sono utili nell'identificare il coinvolgimento cerebellare, che è un reperto che può essere presente in casi di emorragia della matrice germinativa.

\paragraph{Spettroscopia} La spettroscopia di risonanza magnetica è una tecnica di imaging avanzata che fornisce informazioni sul metabolismo cerebrale e può essere uno strumento prezioso nella valutazione del danno ipossico-ischemico (HII).

\begin{itemize}
	\tightlist
	\item
	\textbf{Rilevamento precoce del danno:} La MRS è particolarmente utile nelle prime 24 ore dopo un evento ipossico-ischemico, quando le tecniche di imaging convenzionali possono non mostrare anomalie significative. La MRS può rivelare cambiamenti metabolici, come l'aumento del lattato, anche quando l'imaging strutturale è normale.
	\item
	\textbf{Valutazione della gravità del danno}: La MRS può essere più sensibile rispetto alla RM convenzionale nel valutare la gravità del danno cerebrale nelle prime ore dopo l'insulto. L'entità delle alterazioni metaboliche riscontrate con la MRS può essere correlata alla severità della lesione.
	\item
	\textbf{Misurazione del lattato}: La MRS è in grado di rilevare l'aumento del lattato (Lac) nelle aree cerebrali colpite. Il lattato è un indicatore di metabolismo anaerobico, che si verifica quando il tessuto cerebrale non riceve sufficiente ossigeno. Un picco di lattato a 1,3 ppm (a 1,5 T) è un reperto tipico in caso di lesione ipossica-ischemica acuta. Il lattato può essere presente normalmente nei neonati prematuri e nel liquor cefalo-rachidiano, quindi bisogna porre attenzione al posizionamento del voxel. Un'altra insidia nell'interpretazione degli spettri di risonanza magnetica è che il lattato può tornare a livelli normali circa 24 ore dopo la nascita, un fenomeno noto come pseudonormalizzazione del picco del lattato. Un aumento secondario dei livelli di lattato, noto come crollo energetico secondario, si verifica dopo 24-48 ore, con un picco del livello di lattato a circa il quinto giorno dopo un episodio di HII.
	\item
	\textbf{Identificazione di altri metaboliti}: La MRS può identificare anche altri metaboliti che cambiano a seguito di danno ipossico-ischemico come glutammina-glutammato, e può mostrare una diminuzione di N-acetil-aspartato (NAA), un marker di integrità neuronale. In alcuni casi si possono identificare alterazioni specifiche di alcuni disturbi metabolici. Per esempio, un picco a 0,9 ppm è tipico della malattia delle urine a sciroppo d'acero, e un picco di glicina a 3,56 ppm può essere presente nell'iperglicinemia non chetotica.
	\item
	\textbf{Valutazione prognostica}: Le alterazioni metaboliche rilevate con la MRS, come l'aumento del lattato e la diminuzione del NAA, sono state correlate con gli esiti neuroevolutivi nei neonati con HII. La combinazione di MRS e di altri esami di imaging può essere utile per predire il decorso neurologico a breve termine.
	\item
	\textbf{Distinzione tra lesioni ipossiche e metaboliche:} La MRS può essere utile nella diagnosi differenziale, ad esempio tra HII e patologie metaboliche, permettendo di identificare pattern metabolici specifici di alcune malattie. In caso di sospetti disturbi metabolici, una sequenza MRS appropriata (ad esempio, con un tempo di eco breve, TE 30-35ms) può consentire l'identificazione di specifici picchi di metaboliti caratteristici di tale disturbo.
	\item
	\textbf{Limitazioni}:
	
	\begin{itemize}
		\tightlist
		\item
		La MRS richiede più tempo rispetto all'imaging RM convenzionale.
		\item
		La MRS può studiare solo regioni di interesse specifiche.
		\item
		Le alterazioni dei metaboliti possono evolvere rapidamente, per questo sono consigliabili studi ripetuti nel tempo.
		\item
		L'interpretazione dei risultati può essere influenzata dall'ipotermia terapeutica, che attualmente rappresenta il trattamento di elezione per questa patologia.
	\end{itemize}
\end{itemize}

\paragraph{DTI} L'imaging con tensore di diffusione (DTI) è una tecnica avanzata di risonanza magnetica (RM) che ha un valore significativo nella valutazione del danno ipossico-ischemico (HIE) nei neonati. Il DTI fornisce informazioni dettagliate sulla microstruttura cerebrale che non sono rilevabili con le tecniche di RM convenzionali.

\begin{itemize}
	\item
	\textbf{Valutazione della microstruttura cerebrale:} La DTI quantifica il movimento delle molecole d'acqua nei tessuti cerebrali utilizzando modelli matematici, consentendo di caratterizzare le strutture anatomiche microscopiche. Questa tecnica fornisce \textbf{misure quantitative} come l'\textbf{anisotropia frazionaria (FA)} e la \textbf{diffusività media (MD)}, che sono sensibili ai cambiamenti cellulari e extracellulari causati dall'HIE. Le alterazioni di questi parametri possono indicare edema citotossico, edema vascolare, infiammazione, morte cellulare e degenerazione walleriana.
	\item
	\textbf{Rilevazione precoce del danno:} La DTI può rilevare alterazioni del tessuto cerebrale nelle \textbf{prime fasi dell'HIE}, quando le tecniche di RM convenzionali possono risultare normali o solo leggermente alterate. In particolare, la DTI è più sensibile rispetto alla RM convenzionale nel rilevare \textbf{lesioni acute e subacute}. Le alterazioni di FA e MD possono indicare lesioni precoci, permettendo un intervento tempestivo.
	\item
	\textbf{Predizione della prognosi:} La DTI è uno strumento potente per \textbf{predire la prognosi} dell'HIE neonatale. Studi hanno dimostrato che alterazioni delle misure di DTI in regioni specifiche come il corpo calloso, il talamo, i gangli della base, il tratto corticospinale e la sostanza bianca frontale sono altamente predittive di esiti neurologici gravi.
	\item
	\textbf{Correlazione con la gravità del danno:} La DTI è utile per quantificare l'entità del danno cerebrale e per differenziare tra \textbf{HIE lieve, moderata e grave}. In generale, i casi di HIE grave sono associati a cambiamenti più estesi e severi nei valori di FA e MD rispetto ai casi lievi o moderati. \textbf{La gravità dei cambiamenti di FA e MD si correla con la gravità del danno cerebrale e della prognosi}.
	
	\begin{itemize}
		\tightlist
		\item
		Ad esempio, una riduzione di FA è stata osservata nei casi di HIE moderata-grave nelle prime 3 settimane di vita, mentre una riduzione della MD si osserva nella sostanza bianca nei casi più gravi.
		\item
		Valori bassi di FA nel tratto corticospinale e nel peduncolo cerebrale, e bassi valori di ADC nel tratto corticospinale e nei gangli della base sono stati correlati con una prognosi neurologica sfavorevole.
		\item
		Al contrario, elevati valori di ADC nella parte posteriore del braccio della capsula interna, si correlano con una migliore sopravvivenza e prognosi a due anni nei neonati con HIE.
	\end{itemize}
	\item
	\textbf{Monitoraggio dell'evoluzione del danno:} La DTI può essere utilizzata per monitorare l'evoluzione del danno nel tempo. È stato osservato che i valori di ADC tendono ad aumentare nella fase cronica, mentre i valori di FA diminuiscono. Le alterazioni di DTI cambiano nel tempo e possono differire a seconda della gravità e della tempistica dell'esame.
	\item
	\textbf{Identificazione di aree cerebrali specifiche coinvolte:} La DTI permette di identificare le aree cerebrali specifiche che sono state danneggiate dall'HIE. L'analisi basata su regioni di interesse (ROI) e le analisi basate su dati (come l'analisi voxel-by-voxel utilizzando la statistica spaziale basata sul tratto (TBSS) e i metodi basati su atlante (ABA) consentono di quantificare i valori di DTI per ogni regione anatomica. Studi hanno dimostrato che diverse regioni cerebrali sono coinvolte in modo differenziale nell'HIE.
	
	\begin{itemize}
		\tightlist
		\item
		Riduzioni di MD sono state osservate nel putamen, talamo, braccio anteriore e posteriore della capsula interna, sostanza bianca occipitale e tratto corticospinale.
	\end{itemize}
	\item
	Riduzioni di FA sono state osservate nel cervelletto in casi di HIE grave, insieme alla riduzione di MD nel peduncolo cerebellare superiore e riduzione di FA nel peduncolo cerebellare medio. * Aumenti di ADC accompagnati da riduzioni di FA sono stati osservati nei gangli della base, talamo, parte posteriore della capsula interna, peduncolo cerebrale e sostanza bianca periferica.
	\item
	\textbf{Studio della mielinizzazione:} La DTI è sensibile ai cambiamenti nella mielinizzazione della sostanza bianca. L'evoluzione dei parametri di DTI può fornire informazioni sullo sviluppo normale e patologico della mielina nei neonati con HIE. La DTI è in grado di studiare la traiettoria di sviluppo normale del cervello.
	\item
	\textbf{Approcci di analisi:}
	
	\begin{itemize}
		\tightlist
		\item
		\textbf{Analisi basata su ROI:} Prevede la selezione di regioni specifiche del cervello, con l'obiettivo di quantificare la DTI per queste regioni.
		\item
		\textbf{TBSS:} È un metodo per quantificare i valori scalari su base voxel per intere regioni della sostanza bianca, basato sulla scheletrizzazione delle mappe FA.
		\item
		\textbf{ABA:} Utilizza atlanti cerebrali per la quantificazione dell'immagine.
	\end{itemize}
	\item
	\textbf{Prognosi a lungo termine:} La DTI può fornire una predizione degli esiti neurologici a lungo termine, come le disabilità motorie e cognitive. Studi hanno dimostrato che alterazioni di FA e MD si correlano con esiti negativi nel neurosviluppo.
	\item
	\textbf{Limiti:} La DTI è sensibile alle differenze tra i tipi di scanner e i parametri di scansione, il che può rendere difficile il confronto tra studi diversi. Pertanto, è preferibile utilizzare un singolo scanner e un protocollo di scansione fisso negli studi scientifici.
\end{itemize}


\subsubsection{Ecografia}

L'ecografia transcranica (US) è uno strumento di screening potente e ampiamente disponibile per la valutazione dei neonati con sospetta encefalopatia ipossico-ischemica (HIE). Nonostante la risonanza magnetica (RM) sia considerata il "gold standard'' per l'imaging cerebrale neonatale, l'US ha diversi vantaggi che la rendono utile nella pratica clinica.

\begin{itemize}
	\item
	\textbf{Screening e diagnosi precoce:} L'ecografia è una tecnica \textbf{economica, portatile e facilmente accessibile}, che può essere eseguita al letto del paziente senza la necessità di sedazione. Questo la rende particolarmente utile per lo screening iniziale e la valutazione precoce dei neonati a rischio di HIE, soprattutto nei centri non accademici e nei paesi a basso reddito, o quando il paziente è troppo instabile per essere trasferito per una RM. L'US può identificare precocemente \textbf{segni di danno cerebrale} nei neonati con HIE, consentendo di iniziare tempestivamente le cure necessarie.
	\item
	\textbf{Valutazione del pattern e dell'estensione del danno:} L'US può contribuire a determinare il \textbf{pattern, il timing e l'estensione del danno} in caso di HIE. Questo è fondamentale per le implicazioni terapeutiche e per la previsione degli esiti neuroevolutivi. L'US può distinguere tra diversi tipi di lesioni, come l'\textbf{emorragia intraventricolare}, la \textbf{leucomalacia periventricolare} (PVL) e l'\textbf{infarto cerebrale}.
	\item
	\textbf{Monitoraggio nel tempo:} L'US può essere ripetuta nel tempo per definire l'\textbf{evoluzione delle lesioni} e per monitorare la risposta al trattamento. Studi ecografici seriali consentono di valutare la progressione o la regressione delle lesioni, così come la comparsa di eventuali complicanze come idrocefalo.
	\item
	\textbf{Valutazione dell'integrità vascolare:} L'ecografia Doppler può essere utilizzata per valutare le \textbf{dinamiche vascolari cerebrali} e l'\textbf{integrità dell'autoregolazione cerebrale}, tramite il calcolo dell'indice di resistenza (RI). Un RI anomalo (inferiore o uguale a 0.55) nelle prime 72 ore di vita è associato a una prognosi sfavorevole (morte o disabilità grave). L'indice resistivo diminuisce con l'aumentare dell'età gestazionale e deve essere correlato all'età gestazionale per ottenere risultati accurati. Tuttavia, il valore predittivo di un RI basso diminuisce durante l'ipotermia terapeutica e ritorna una volta che il neonato viene riscaldato.
	\item
	\textbf{Identificazione di lesioni specifiche:}
	
	\begin{itemize}
		\tightlist
		\item
		L'US può rilevare \textbf{aree di iperecogenicità} nella sostanza bianca periventricolare in caso di PVL. Nelle fasi iniziali della PVL (2-10 giorni di vita), si osserva una maggiore ecogenicità della sostanza bianca periventricolare, successivamente si possono formare delle cisti. Può essere presenta edema cerebrale e aspetto assottigliato dei ventricoli laterali.
		\item
		In caso di danno corticale, l'US può rivelare \textbf{regioni iperecogene a forma di cuneo} nelle zone di confine.
		\item
		L'US può essere utilizzata anche per lo screening di \textbf{emorragie intracraniche}.
	\end{itemize}
	\item
	\textbf{Limiti dell'ecografia:} Nonostante i suoi vantaggi, l'US presenta dei limiti rispetto alla RM. In particolare, \textbf{l'US ha una sensibilità inferiore rispetto alla RM} per la rilevazione di lesioni sottili, come le lesioni della sostanza bianca nel caso di danno ipossico-ischemico lieve o moderato e \textbf{non è in grado di visualizzare le strutture cerebrali con la stessa chiarezza} della RM. La RM è più accurata nell'individuazione di lesioni corticali e nel distinguere tra diversi pattern di danno.
\end{itemize}

\subsection{Trattamento e prognosi}

\subsubsection{Sequele neurologiche}

\paragraph{Leucomalacia periventricolare} La leucomalacia periventricolare (PVL) è una lesione della sostanza bianca cerebrale che si verifica comunemente nei neonati pretermine, ma può essere osservata anche nei neonati a termine. La patogenesi della PVL è complessa e multifattoriale, e coinvolge sia l'ischemia che la vulnerabilità delle cellule oligodendrogliali.

\begin{itemize}
	\tightlist
	\item
	\textbf{Vulnerabilità degli oligodendrociti:} La PVL è causata dalla \textbf{vulnerabilità selettiva} delle cellule della linea degli oligodendrociti a lesioni ipossiche-ischemiche. In particolare, i \textbf{preoligodendrociti}, precursori degli oligodendrociti maturi, sono particolarmente suscettibili ai danni da stress ossidativo e da eccitotossicità causati dall'ipossia. Questo è significativo perché, prima dell'inizio della mielinizzazione, la sostanza bianca è popolata principalmente da questi precursori. La PVL si manifesta più frequentemente nelle aree adiacenti ai trigoni dei ventricoli laterali e ai forami di Monro.
	\item
	\textbf{Ischemia:} In passato si credeva che la PVL fosse causata principalmente da \textbf{ischemia nelle zone di confine} della sostanza bianca periventricolare, in particolare nelle zone dove la vascolarizzazione è limitata. Si ipotizzava che la sostanza bianca nel cervello fetale immaturo fosse rifornita da arterie che penetravano dall'esterno verso i ventricoli, rendendo la sostanza bianca profonda più vulnerabile a diminuzioni della perfusione. Questa teoria è stata messa in discussione, tuttavia studi anatomici hanno evidenziato una bassa vascolarizzazione della sostanza bianca cerebrale fino a 32 settimane di gestazione, suggerendo un ruolo dell'ipovascolarizzazione nello sviluppo della PVL.
	\item
	\textbf{Eventi ipossico-ischemici:} Gli eventi ipossico-ischemici portano a una serie di eventi a cascata che contribuiscono allo sviluppo della PVL. Questi eventi causano un \textbf{danno alle cellule endoteliali} dei capillari della matrice germinativa, con conseguente perdita dell'integrità capillare. La riperfusione cerebrale successiva all'evento ipossico-ischemico può portare a \textbf{emorragie}, che possono essere rilevate con l'ecografia o la RM.
	\item
	\textbf{Infiammazione:} L'infezione o l'infiammazione possono peggiorare la PVL. Il rilascio di \textbf{citochine infiammatorie} può disturbare l'autoregolazione vascolare cerebrale, aumentando la suscettibilità al danno ipossico-ischemico.
\end{itemize}

\textbf{Evoluzione della PVL:}

\begin{itemize}
	\tightlist
	\item
	\textbf{Fase iniziale:} Inizialmente, si osserva \textbf{necrosi}, con aree di aumentata ecogenicità all'ecografia nelle regioni periventricolari entro le prime 48 ore.
	\item
	\textbf{Fase di normalizzazione:} Segue un periodo transitorio di relativa normalizzazione, generalmente dalla seconda alla quarta settimana di vita.
	\item
	\textbf{Formazione di cisti:} Successivamente, si sviluppano \textbf{cisti periventricolari} tra le 3 e le 6 settimane di vita. Queste cisti possono evolvere in pori encefalici.
	\item
	\textbf{Fase tardiva:} Con il tempo, le cisti si riducono e si sviluppa \textbf{gliosi}, con una significativa riduzione del volume della sostanza bianca nelle regioni periventricolari. La sostanza bianca periventricolare assume un aspetto iperintenso in T2 nelle immagini RM, e le pareti dei ventricoli possono apparire irregolari.
	\item
	\textbf{Stadio finale:} La fase finale della PVL è caratterizzata da \textbf{ventricolomegalia} con dilatazione dei trigoni, e un contorno irregolare dei ventricoli, con la perdita di volume della sostanza bianca periventricolare. Può esserci assottigliamento del corpo calloso, in particolare nella porzione posteriore e nello splenio.
\end{itemize}

\textbf{Reperti di imaging:}

\begin{itemize}
	\tightlist
	\item
	\textbf{Ecografia:} L'ecografia può rilevare inizialmente un'aumentata ecogenicità periventricolare, seguita dalla comparsa di cisti.
	\item
	\textbf{RM:} La RM consente una migliore visualizzazione delle anomalie della sostanza bianca periventricolare rispetto all'ecografia, specialmente quando ci sono aree non cistiche. La RM può evidenziare focolai di ipersignal T1, circondate da zone di ipersignal T2, che rappresentano l'astrogliosi reattiva e possibili depositi di mineralizzazione. La diffusione e la spettroscopia RM possono aiutare nella valutazione della lesione.
	\item
	\textbf{TC:} La TC non è utile nelle prime fasi della PVL, ma può essere utile per confermare i reperti di PVL allo stadio finale, come la riduzione di volume della sostanza bianca e la ventricolomegalia.
\end{itemize}

La gravità della PVL e l'estensione delle lesioni sono associate a un aumentato rischio di deficit neurologici a lungo termine, come la paralisi cerebrale. I deficit più comuni sono quelli motori e visivi, a causa del coinvolgimento delle vie motorie e visive che attraversano le regioni più colpite della sostanza bianca.

\paragraph{Altre sequele}
Le sequele a lungo termine delle lesioni profonde includono l'atrofia e la mineralizzazione cronica del talamo, del braccio posteriore della capsula interna e dei gangli della base. Può verificarsi un'associata perdita generalizzata di volume nella materia grigia corticale e nella materia bianca sottocorticale, dovuta all'interruzione del normale sviluppo dei neuroni e delle vie assonali.

\subsection{Checklist di refertazione}

\subsection{Bibliografia}
\tiny{
\noindent
1.  Onda K, Chavez-Valdez R, Graham EM, et al. Quantification of Diffusion Magnetic Resonance Imaging for Prognostic Prediction of Neonatal Hypoxic-Ischemic Encephalopathy. Dev Neurosci 2024;46:55–68.

\noindent
2.  Midiri F, La Spina C, Alongi A, et al. Ischemic hypoxic encephalopathy: The role of MRI of neonatal injury and medico-legal implication. Forensic Science International 2021;327:110968.

\noindent
3.  Salas J, Tekes A, Hwang M, et al. Head Ultrasound in Neonatal Hypoxic-Ischemic Injury and Its Mimickers for Clinicians: A Review of the Patterns of Injury and the Evolution of Findings Over Time. Neonatology 2018;114:185–97.

\noindent
4.  Krishnan P, Shroff M. Neuroimaging in Neonatal Hypoxic Ischemic Encephalopathy. Indian J Pediatr 2016;83:995–1002.

\noindent
5.  Ghei SK, Zan E, Nathan JE, et al. MR Imaging of Hypoxic-Ischemic Injury in Term Neonates: Pearls and Pitfalls. RadioGraphics 2014;34:1047–61.

\noindent
6.  Liauw L, Van Der Grond J, Van Den Berg-Huysmans AA, et al. Hypoxic-Ischemic Encephalopathy: Diagnostic Value of Conventional MR Imaging Pulse Sequences in Term-born Neonates. Radiology 2008;247:204–12.

\noindent
7.  Huang BY, Castillo M. Hypoxic-Ischemic Brain Injury: Imaging Findings from Birth to Adulthood. RadioGraphics 2008;28:417–39.

\noindent
8.  Chao CP, Zaleski CG, Patton AC. Neonatal Hypoxic-Ischemic Encephalopathy: Multimodality Imaging Findings. RadioGraphics 2006;26:S159–72.

}

\note{Nota a margine}
\expl{Nota a margine colorata}

\begin{itemize}[label=$\square$] % Riquadro vuoto come simbolo
	\item Primo elemento
	\item Secondo elemento
	\item Terzo elemento
\end{itemize}