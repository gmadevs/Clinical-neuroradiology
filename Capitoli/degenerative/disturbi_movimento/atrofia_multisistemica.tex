\section{Atrofia Multi-Sistemica}

\subsection{Definizione}
L'atrofia multisistemica (AMS) è una malattia neurodegenerativa progressiva che si caratterizza per la degenerazione neuronale in diverse aree del sistema nervoso centrale. L'AMS è considerata una sinucleinopatia, una categoria di malattie neurodegenerative caratterizzate dall'accumulo anomalo di alfa-sinucleina

\subsection{Eziologia}

\subsection{Epidemiologia}
L'atrofia multisistemica rappresenta una patologia neurodegenerativa di rara incidenza, gli studi epidemiologici hanno documentato un'incidenza annuale di 0,6 casi per 100.000 persone-anno nella popolazione generale, con un incremento significativo a 3 casi per 100.000 persone-anno nella popolazione ultracinquantenne. La prevalenza della patologia si attesta tra 2 e 5 casi per 100.000 abitanti, evidenziando una frequenza approssimativamente decimale rispetto alla malattia di Parkinson. L'esordio clinico si colloca tipicamente tra i 54 e i 60 anni d'età, sebbene siano stati documentati casi di insorgenza precoce. Dal punto di vista demografico, la distribuzione della MSA presenta un pattern ubiquitario a livello globale, senza evidenti predilezioni etniche, razziali o di genere. L'eziologia della patologia risulta prevalentemente sporadica, con rare manifestazioni familiari documentate che suggeriscono una potenziale componente genetica in specifici casi.

\subsection{Presentazione  clinica}
L'atrofia multisistemica presenta un quadro clinico eterogeneo caratterizzato dalla degenerazione progressiva di molteplici sistemi neurologici, manifestandosi attraverso una costellazione di sintomi autonomici, parkinsoniani, cerebellari e piramidali. La dicotomia fenotipica principale si articola nelle varianti MSA-P e MSA-C: la forma parkinsoniana (MSA-P) si caratterizza per bradicinesia, rigidità e tremore con risposta subottimale alla levodopa, mentre la variante cerebellare (MSA-C) manifesta prominente atassia, disturbi dell'equilibrio e disartria. La disfunzione autonomica, elemento patognomonico della patologia, si esprime attraverso ipotensione ortostatica, alterazioni vescico-sfinteriche, disturbi della sudorazione e disfunzione erettile nel sesso maschile. Il coinvolgimento piramidale si evidenzia mediante iperreflessia, segno di Babinski e spasticità. Il quadro clinico si arricchisce ulteriormente per la presenza di disturbi del sonno REM, stridor respiratorio - indicatore prognostico sfavorevole - camptocormia, antécollis, disfagia, disfonia, nistagmo e compromissione cognitiva di grado variabile. La progressione accelerata della patologia, confrontata con altre sindromi neurodegenerative, determina un rapido declino delle funzioni motorie, autonomiche e respiratorie. La complessità diagnostica deriva dalla sostanziale sovrapposizione sintomatologica con altre sinucleinopatie, particolarmente la malattia di Parkinson e la demenza a corpi di Lewy. L'identificazione precoce dell'MSA richiede il riconoscimento di sintomi cardine, tra cui il parkinsonismo levodopa-resistente, l'atassia cerebellare, la disfunzione autonomica e i segni piramidali, supportati da neuroimaging e valutazione clinica sistematica per l'esclusione di condizioni fenotipicamente simili.

\subsection{Approccio diagnostico}

\subsection{Anatomia patologica}

\subsection{Imaging}

\subsection{Trattamento e prognosi}

\subsection{Checklist di refertazione}

\subsection{Bibliografia}
\small{
	
	
}

\note{Nota a margine}
\expl{Nota a margine colorata}

\begin{itemize}[label=$\square$] % Riquadro vuoto come simbolo
	\item Primo elemento
	\item Secondo elemento
	\item Terzo elemento
\end{itemize}