\section {Sindrome tremorigeno-atassica associata a X-Fragile (FXTAS)}

\subsection{Definizione}
La sindrome tremorigena/atassica associata all'X fragile (FXTAS) è una malattia neurodegenerativa che si manifesta in età adulta.

\subsection{Eziologia}
La sindrome tremore/atassia associata all'X fragile (FXTAS) è causata da un'espansione della ripetizione CGG nel gene FMR1 (fragile X mental retardation 1). Questa espansione, che rientra nel range di premutazione (da 55 a 200 ripetizioni CGG), è la principale causa della patologia. La premutazione si distingue dalla mutazione completa (>200 ripetizioni CGG), che è responsabile della sindrome dell'X fragile (FXS).
L'espansione della ripetizione CGG nel gene FMR1 porta a un aumento dei livelli di mRNA FMR1. Questo aumento dell'mRNA è ritenuto un fattore chiave nella patogenesi della FXTAS, in quanto l'accumulo di mRNA tossico interferisce con la normale funzione cellulare e causa neurodegenerazione. In particolare, l'mRNA FMR1 elevato è stato collegato alla formazione di inclusioni intranucleari che si trovano nei neuroni e negli astrociti di tutto il cervello, sono una caratteristica patologica distintiva della FXTAS e sono positivi all'ubiquitina all'immunoistochimica. Le inclusioni contengono materiale granulofilamentoso, oltre a oltre 20 proteine e mRNA FMR1. La presenza di queste inclusioni contribuisce alla disfunzione cellulare e alla neurodegenerazione osservata nella FXTAS.
Diversi altri meccanismi fisiopatologici sono stati identificati come fattori che contribuiscono all'eziologia della FXTAS, tra cui la disfunzione mitocondriale, l'accumulo di ferro in specifiche regioni del cervello come il putamen, il plesso coroideo e il nucleo dentato del cervelletto, il dismetabolismo del calcio, le anomalie vascolari che includono la malattia dei piccoli vasi cerebrali, micro sanguinamenti e spazi perivascolari allargati, e la tossicità dell'RNA causata dagli elevati livelli di mRNA FMR1.
È importante notare che non tutti i portatori di premutazione sviluppano la FXTAS. La penetranza della FXTAS aumenta con l'età e la lunghezza della ripetizione CGG nel gene FMR1. Altri fattori, come il sesso (i maschi sono più frequentemente colpiti rispetto alle femmine) e fattori genetici non ancora completamente identificati, possono anche influenzare il rischio di sviluppare la FXTAS. In sintesi, l'eziologia della FXTAS è legata all'espansione della ripetizione CGG nel gene FMR1 che porta a livelli elevati di mRNA tossico, con conseguente neurodegenerazione e patologia vascolare.

\subsection{Epidemiologia}
La FXTAS presenta un quadro epidemiologico ben definito, caratterizzato dalla frequenza della premutazione del gene FMR1 nella popolazione generale, con una frequenza stimata di 1 donna su 209 e 1 uomo su 430 portatori della premutazione (55-200 ripetizioni CGG), con alcune fonti che riportano frequenze leggermente diverse di 1 su 250-300 femmine e 1 su 750-850 maschi. La prevalenza della premutazione risulta maggiore nelle donne rispetto agli uomini.
Per quanto riguarda l'incidenza della FXTAS, non tutti i portatori di premutazione sviluppano la sindrome. Nei maschi portatori, circa il 20-33\% sviluppa la FXTAS, con una probabilità che aumenta significativamente con l'età: circa il 40\% dopo i 50 anni, fino a raggiungere il 75\% dopo i 70 anni. Le donne portatrici sono invece meno colpite, con una percentuale che varia tra l'8 e il 16\%, e alcune fonti indicano una penetranza inferiore al 10\% nelle donne oltre i 50 anni, con manifestazioni cliniche generalmente meno gravi rispetto agli uomini.
L'età media di insorgenza dei sintomi si colloca intorno ai 60 anni, con la maggior parte dei casi che si manifesta nella settima decade di vita. La manifestazione può essere preceduta da sintomi iniziali spesso sottovalutati, come lievi disfunzioni cognitive. La maggiore protezione delle donne dalla malattia è attribuibile alla presenza di un cromosoma X normale che esprime il gene FMR1. Le donne che sviluppano la FXTAS tendono ad avere un'inattivazione del cromosoma X non casuale, con una maggiore frazione di cellule che esprimono una premutazione attiva.
Nella popolazione generale, la prevalenza della premutazione è stimata in circa 1 su 813 maschi, con una prevalenza a vita per lo sviluppo di FXTAS tra i maschi di circa 1 su 8000, rendendo questa patologia comparabile ad altre malattie neurodegenerative come le atassie ereditarie. I principali fattori di rischio includono l'età avanzata e il numero elevato di ripetizioni CGG nella premutazione, mentre l'allele APOE$\epsilon$4 potrebbe rappresentare un fattore genetico predisponente allo sviluppo della sindrome.

\subsection{Presentazione clinica}
I sintomi motori della sindrome rappresentano il quadro clinico primario, con il tremore come manifestazione iniziale predominante. Tale tremore si caratterizza per la sua natura intenzionale, posturale o cinetica. La progressione della patologia comporta lo sviluppo di atassia cerebellare, manifestantesi attraverso compromissioni dell'equilibrio, della coordinazione e del pattern deambulatorio. Il quadro può includere manifestazioni parkinsoniane, quali bradicinesia e rigidità, seppur generalmente di entità contenuta.
La sfera cognitiva presenta alterazioni significative, con deficit mnesici, attentivi e delle funzioni esecutive. Il deterioramento cognitivo manifesta un decorso progressivo, potendo evolvere verso quadri di severità variabile, talora sovrapponibili a sindromi demenziali. Le alterazioni cognitive precoci tipicamente coinvolgono le capacità di pianificazione, organizzazione e decision-making.
Il quadro sintomatologico si completa con manifestazioni neurologiche periferiche, quali parestesie e algie agli arti, disfunzioni autonomiche caratterizzate da ipotensione postprandiale, e alterazioni neuropsichiatriche comprendenti depressione, ansia, irritabilità e labilità emotiva. Sono frequentemente riscontrabili disturbi del sonno, includenti insonnia e movimenti periodici degli arti inferiori.
La presentazione clinica evidenzia una significativa eterogeneità fenotipica, con predominanza variabile di tremore, atassia, declino cognitivo, neuropatia o sintomatologia psichiatrica. La sequenza temporale di insorgenza sintomatologica presenta considerevole variabilità interindividuale.
L'esordio insidioso e progressivo della sindrome, caratterizzato da sintomi inizialmente sfumati e aspecifici, può ostacolare la diagnosi precoce. La presenza di anamnesi familiare positiva per FXTAS o disturbi neurologici analoghi, particolarmente nella linea maschile, costituisce un elemento di rilevanza diagnostica significativa.

\subsection{Approccio diagnostico}
L'approccio diagnostico per la FXTAS si basa su una combinazione di valutazioni cliniche, esami radiologici e test genetici. L'obiettivo è identificare i segni caratteristici della malattia e distinguerla da altre patologie con sintomi simili.
L'anamnesi deve porre particolare attenzione alla storia familiare di disturbi neurologici, tremore, atassia, problemi cognitivi o disturbi psichiatrici. LA valutazione clinica neurologica deve verificare la presenza di tremore, osservando il tipo (intenzionale, posturale, cinetico) e la localizzazione. La presenza di un'eventuale atassia, valutando l'equilibrio, la coordinazione e l'andatura. Si utilizzano scale di valutazione standardizzate come la Scale for the Assessment and Rating of Ataxia (SARA).
La valutazione neurologica approfondita comprende l'identificazione di manifestazioni parkinsoniane, con particolare attenzione alla presenza di bradicinesia e rigidità, unitamente all'analisi delle alterazioni neurologiche periferiche evidenziate da modificazioni della sensibilità e dei riflessi osteotendinei. Il protocollo diagnostico include la valutazione della funzionalità autonomica, con focus specifico sui fenomeni di ipotensione ortostatica, mentre il profilo cognitivo viene esaminato mediante una batteria standardizzata di test neuropsicologici comprendente il Mini-Mental State Examination, la Frontal Assessment Battery per le funzioni esecutive, il Symbol Digit Modalities Test per la velocità di elaborazione dell'informazione e il Behavioral Dyscontrol Scale per l'attenzione e l'inibizione della risposta. L'assessment psichiatrico si concentra sulla rilevazione di manifestazioni depressive, ansiose e alterazioni comportamentali, mentre la disabilità fisica viene quantificata attraverso scale validate specifiche per la valutazione dell'impatto funzionale del tremore e dei deficit dell'equilibrio.
L'analisi molecolare del gene FMR1 costituisce un elemento diagnostico imprescindibile per la conferma della FXTAS, mediante la determinazione quantitativa delle ripetizioni CGG nella regione promotrice. La metodologia diagnostica prevede l'identificazione della premutazione, caratterizzata da un'espansione compresa tra 55 e 200 ripetizioni CGG, attraverso l'impiego sinergico di tecniche di PCR e Southern blot, con particolare riferimento a protocolli PCR ottimizzati per la rilevazione di espansioni a bassa abbondanza. Il medesimo approccio analitico consente la discriminazione dalle mutazioni complete del gene FMR1, definite da espansioni superiori a 200 ripetizioni CGG, associate alla sindrome dell'X fragile (FXS).
La biopsia cutanea, eseguita con microscopia elettronica, può rivelare inclusioni intranucleari filamentose, un reperto che si riscontra anche nella malattia da inclusione intranucleare neuronale (NIID); per questo motivo è essenziale integrare il dato con il test genetico per il gene FMR1. Altri esami utili includono test neuropsicologici per valutare il declino cognitivo, una valutazione psichiatrica per rilevare sintomi come depressione e ansia, l'elettromiografia (EMG) per identificare la presenza di neuropatia periferica e una valutazione del sonno per indagare problematiche come i movimenti periodici delle gambe.
\begin{TitoloIntro}[colbacktitle=red]{Criteri diagnostici}
{
La diagnosi di FXTAS si basa sulla presenza di specifici criteri clinici e radiologici, in combinazione con la dimostrazione della premutazione del gene FMR1.
\begin{description}[style=unboxed,leftmargin=0cm]
\item[FXTAS probabile]{presenza di due criteri clinici maggiori, oppure un criterio clinico maggiore e uno minore (parkinsonismo o deficit cognitivo), oppure un criterio clinico minore e un criterio radiologico maggiore}
\item[FXTAS possibile]{presenza di un criterio clinico maggiore e un criterio radiologico minore (lesioni della sostanza bianca cerebrale o atrofia cerebrale)}
\end{description}
}
\end{TitoloIntro}
È fondamentale considerare che i sintomi della FXTAS possono sovrapporsi con quelli di altre malattie neurodegenerative, come la malattia di Parkinson, l'atrofia multisistemica (MSA) e le atassie spinocerebellari. Pertanto, è necessario eseguire un'attenta diagnosi differenziale. Inoltre, in caso di sospetto di FXTAS, è importante fornire consulenza genetica al paziente e alla famiglia.

\subsection{Anatomia patologica}

\subsection{Imaging}

\begin{figure*}[h]
	\centering
	\includegraphics[width=0.7\linewidth]{FileAusiliari/Immagini/degenerative/tre-02-56-352-1-g001}
	\caption[fxtas-flair]{Risonanza magnetica assiale FLAIR (Fluid Attenuated Inversion Recovery) in un paziente FXTAS.(A) Iperintensità nel peduncolo cerebellare medio; (B,C) perdita di volume globale con iperintensità sparse della sostanza bianca. FXTAS, Fragile X-associated tremor ataxia syndrome. Da Tremor Other Hyperkinet Mov (N Y). 2012 May 11;2:tre-02-56-352-1. doi: 10.7916/D8HD7TDS}
	\label{fig:tre-02-56-352-1-g001}
\end{figure*}

\subsubsection{RM}
Il segno RM più frequente nei pazienti con FXTAS è l'iperintensità della sostanza bianca nel peduncolo cerebellare medio (MCP), noto come segno del peduncolo cerebellare medio (MCP sign), rappresenta un criterio diagnostico rilevante, anche se non sempre presente, essendo osservato nel 60\% dei maschi affetti. Altri reperti includono iperintensità della sostanza bianca nel cervelletto e nel tronco encefalico, iperintensità dello splenium del corpo calloso (CCS) come criterio diagnostico maggiore aggiuntivo, lesioni della sostanza bianca in aree periventricolari, subcorticali, pontine e insulari, atrofia cerebrale e cerebellare, e atrofia della sostanza grigia subcorticale, che coinvolge il talamo, il nucleo caudato, il putamen e il globo pallido. Altri segni radiologici includono un aumento degli spazi perivascolari, soprattutto nei gangli della base, micro sanguinamenti cerebrali più frequenti nei pazienti con FXTAS e segnali iperintensi visibili sulla *Diffusion Weighted Imaging* (DWI) lungo la giunzione corticomidollare. Inoltre, la quantificazione delle immagini MRI consente di misurare i volumi cerebrali e le lesioni della sostanza bianca. In alcuni casi selezionati, l'utilizzo del Magnetic Resonance Parkinson Index (MRPI) può aiutare a distinguere la FXTAS da altre malattie neurodegenerative.
*Specificità peduncolo cerebellare medio

\subsection{Trattamento e prognosi}

\subsection{Checklist di refertazione}

\subsection{Bibliografia}
\small{
	
	
}

{\tiny Ultima modifica: \filemodprintdate{\jobname}}

\note{Nota a margine}
\expl{Nota a margine colorata}

\begin{itemize}[label=$\square$] % Riquadro vuoto come simbolo
	\item Primo elemento
	\item Secondo elemento
	\item Terzo elemento
\end{itemize}