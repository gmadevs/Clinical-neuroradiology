\section {Sindrome tremorigeno-atassica associata a X-Fragile (FXTAS)}

\subsection{Definizione}
La sindrome tremorigena/atassica associata all'X fragile (FXTAS) è una malattia neurodegenerativa che si manifesta in età adulta.

\subsection{Eziologia}
La sindrome tremore/atassia associata all'X fragile (FXTAS) è causata da un'espansione della ripetizione CGG nel gene FMR1 (fragile X mental retardation 1). Questa espansione, che rientra nel range di premutazione (da 55 a 200 ripetizioni CGG), è la principale causa della patologia. La premutazione si distingue dalla mutazione completa (>200 ripetizioni CGG), che è responsabile della sindrome dell'X fragile (FXS).
L'espansione della ripetizione CGG nel gene FMR1 porta a un aumento dei livelli di mRNA FMR1. Questo aumento dell'mRNA è ritenuto un fattore chiave nella patogenesi della FXTAS, in quanto l'accumulo di mRNA tossico interferisce con la normale funzione cellulare e causa neurodegenerazione. In particolare, l'mRNA FMR1 elevato è stato collegato alla formazione di inclusioni intranucleari che si trovano nei neuroni e negli astrociti di tutto il cervello, sono una caratteristica patologica distintiva della FXTAS e sono positivi all'ubiquitina all'immunoistochimica. Le inclusioni contengono materiale granulofilamentoso, oltre a oltre 20 proteine e mRNA FMR1. La presenza di queste inclusioni contribuisce alla disfunzione cellulare e alla neurodegenerazione osservata nella FXTAS.
Diversi altri meccanismi fisiopatologici sono stati identificati come fattori che contribuiscono all'eziologia della FXTAS, tra cui la disfunzione mitocondriale, l'accumulo di ferro in specifiche regioni del cervello come il putamen, il plesso coroideo e il nucleo dentato del cervelletto, il dismetabolismo del calcio, le anomalie vascolari che includono la malattia dei piccoli vasi cerebrali, micro sanguinamenti e spazi perivascolari allargati, e la tossicità dell'RNA causata dagli elevati livelli di mRNA FMR1.
È importante notare che non tutti i portatori di premutazione sviluppano la FXTAS. La penetranza della FXTAS aumenta con l'età e la lunghezza della ripetizione CGG nel gene FMR1. Altri fattori, come il sesso (i maschi sono più frequentemente colpiti rispetto alle femmine) e fattori genetici non ancora completamente identificati, possono anche influenzare il rischio di sviluppare la FXTAS. In sintesi, l'eziologia della FXTAS è legata all'espansione della ripetizione CGG nel gene FMR1 che porta a livelli elevati di mRNA tossico, con conseguente neurodegenerazione e patologia vascolare.

\subsection{Epidemiologia}
La FXTAS presenta un quadro epidemiologico ben definito, caratterizzato dalla frequenza della premutazione del gene FMR1 nella popolazione generale, con una frequenza stimata di 1 donna su 209 e 1 uomo su 430 portatori della premutazione (55-200 ripetizioni CGG), con alcune fonti che riportano frequenze leggermente diverse di 1 su 250-300 femmine e 1 su 750-850 maschi. La prevalenza della premutazione risulta maggiore nelle donne rispetto agli uomini.
Per quanto riguarda l'incidenza della FXTAS, non tutti i portatori di premutazione sviluppano la sindrome. Nei maschi portatori, circa il 20-33\% sviluppa la FXTAS, con una probabilità che aumenta significativamente con l'età: circa il 40\% dopo i 50 anni, fino a raggiungere il 75\% dopo i 70 anni. Le donne portatrici sono invece meno colpite, con una percentuale che varia tra l'8 e il 16\%, e alcune fonti indicano una penetranza inferiore al 10\% nelle donne oltre i 50 anni, con manifestazioni cliniche generalmente meno gravi rispetto agli uomini.
L'età media di insorgenza dei sintomi si colloca intorno ai 60 anni, con la maggior parte dei casi che si manifesta nella settima decade di vita. La manifestazione può essere preceduta da sintomi iniziali spesso sottovalutati, come lievi disfunzioni cognitive. La maggiore protezione delle donne dalla malattia è attribuibile alla presenza di un cromosoma X normale che esprime il gene FMR1. Le donne che sviluppano la FXTAS tendono ad avere un'inattivazione del cromosoma X non casuale, con una maggiore frazione di cellule che esprimono una premutazione attiva.
Nella popolazione generale, la prevalenza della premutazione è stimata in circa 1 su 813 maschi, con una prevalenza a vita per lo sviluppo di FXTAS tra i maschi di circa 1 su 8000, rendendo questa patologia comparabile ad altre malattie neurodegenerative come le atassie ereditarie. I principali fattori di rischio includono l'età avanzata e il numero elevato di ripetizioni CGG nella premutazione, mentre l'allele APOE$\epsilon$4 potrebbe rappresentare un fattore genetico predisponente allo sviluppo della sindrome.

\subsection{Presentazione clinica}
I sintomi motori della sindrome rappresentano il quadro clinico primario, con il tremore come manifestazione iniziale predominante. Tale tremore si caratterizza per la sua natura intenzionale, posturale o cinetica. La progressione della patologia comporta lo sviluppo di atassia cerebellare, manifestantesi attraverso compromissioni dell'equilibrio, della coordinazione e del pattern deambulatorio. Il quadro può includere manifestazioni parkinsoniane, quali bradicinesia e rigidità, seppur generalmente di entità contenuta.
La sfera cognitiva presenta alterazioni significative, con deficit mnesici, attentivi e delle funzioni esecutive. Il deterioramento cognitivo manifesta un decorso progressivo, potendo evolvere verso quadri di severità variabile, talora sovrapponibili a sindromi demenziali. Le alterazioni cognitive precoci tipicamente coinvolgono le capacità di pianificazione, organizzazione e decision-making.
Il quadro sintomatologico si completa con manifestazioni neurologiche periferiche, quali parestesie e algie agli arti, disfunzioni autonomiche caratterizzate da ipotensione postprandiale, e alterazioni neuropsichiatriche comprendenti depressione, ansia, irritabilità e labilità emotiva. Sono frequentemente riscontrabili disturbi del sonno, includenti insonnia e movimenti periodici degli arti inferiori.
La presentazione clinica evidenzia una significativa eterogeneità fenotipica, con predominanza variabile di tremore, atassia, declino cognitivo, neuropatia o sintomatologia psichiatrica. La sequenza temporale di insorgenza sintomatologica presenta considerevole variabilità interindividuale.
L'esordio insidioso e progressivo della sindrome, caratterizzato da sintomi inizialmente sfumati e aspecifici, può ostacolare la diagnosi precoce. La presenza di anamnesi familiare positiva per FXTAS o disturbi neurologici analoghi, particolarmente nella linea maschile, costituisce un elemento di rilevanza diagnostica significativa.

\subsection{Approccio diagnostico}


\subsection{Anatomia patologica}

\subsection{Imaging}
\subsubsection{RM}
*Specificità peduncolo cerebellare medio

\subsection{Trattamento e prognosi}

\subsection{Checklist di refertazione}

\subsection{Bibliografia}
\small{
	
	
}

\note{Nota a margine}
\expl{Nota a margine colorata}

\begin{itemize}[label=$\square$] % Riquadro vuoto come simbolo
	\item Primo elemento
	\item Secondo elemento
	\item Terzo elemento
\end{itemize}