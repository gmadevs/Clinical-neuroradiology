\section{Morbo di Parkinson}

\subsection{Definizione}

\subsection{Eziologia}

\subsection{Epidemiologia}
Il MP costituisce una delle principali cause di disabilità e mortalità nell'ambito delle patologie neurologiche, con una prevalenza particolarmente elevata negli Stati Uniti e in Canada (160-180 casi/100.000 abitanti). L'incidenza annuale in Nord America oscilla tra 108 e 212 casi ogni 100.000 individui di età $\geq$65 anni, con una prevalenza dello 0,3\% nella popolazione adulta $\geq$40 anni e dell'1,6\% nei soggetti ultrasessantacinquenni.
L'esordio della patologia mostra una significativa correlazione con l'età, manifestandosi tipicamente dalla quinta decade di vita, con un'età media alla diagnosi di 70,5 anni e una finestra di esordio prevalente tra i 45 e i 70 anni. Una variante giovanile può presentarsi tra i 20 e i 40 anni, sebbene l'esordio prima dei 30 anni risulti infrequente. La distribuzione per sesso evidenzia una predominanza maschile, particolarmente accentuata nella fascia d'età 50-60 anni.
L'eziologia del MP comprende fattori di rischio genetici, con particolare rilevanza nelle forme ad esordio precoce, e ambientali, tra cui l'esposizione a pesticidi e inquinanti atmosferici. Sono stati identificati fattori protettivi, inclusi il consumo di caffè, l'attività fisica e il fumo di sigaretta. La patologia si presenta prevalentemente in forma sporadica (85-90\% dei casi), mentre una minoranza dei casi (10-15\%) presenta familiarità positiva.

\subsubsection{Fattori di rischio}
L'eziopatogenesi del Morbo di Parkinson (MP) presenta una complessa interazione di fattori di rischio genetici, ambientali e non modificabili. L'anamnesi familiare positiva per MP in consanguinei di primo grado comporta un incremento del rischio relativo di 2-3 volte. Le forme monogeniche, rappresentanti meno del 10\% della casistica totale, manifestano pattern di ereditarietà autosomica dominante, recessiva o X-linked, caratterizzandosi per un esordio precoce rispetto alle forme sporadiche.
Le mutazioni eterozigoti del gene GBA1 costituiscono un significativo fattore di rischio genetico, unitamente ad alterazioni di altri geni codificanti per enzimi lisosomiali. Il coinvolgimento di geni quali SNCA, LRRK2, VPS35, Parkin, PINK1 e DJ-1 è stato ampiamente documentato. Particolare rilevanza assumono le mutazioni del gene Nurr1, determinante per l'identità neuronale dopaminergica, e del gene DJ-1, cruciale nella risposta allo stress ossidativo. Le alterazioni del gene PINK1, codificante per una chinasi mitocondriale, e del gene Park2, responsabile della sintesi della proteina parkina, sono associate a forme ad esordio precoce.
L'esposizione a neurotossine ambientali, inclusi mercurio, manganese, disolfuro di carbonio, solventi organici, MPTP e monossido di carbonio, può indurre degenerazione nigrostriatale e parkinsonismo. L'uso di neurolettici e l'abuso endovenoso di efedrone possono causare sindromi parkinsoniane potenzialmente irreversibili. Traumi cranici ripetuti, pesticidi, solventi e inquinamento atmosferico rappresentano ulteriori fattori di rischio ambientale documentati.
Tra i fattori non modificabili, l'età avanzata e il sesso maschile emergono come significativi predittori di rischio, con predominanza nella sesta decade di vita. Comorbidità quali depressione, ansia, stipsi, diabete mellito tipo 2, obesità e alterazioni del metabolismo del ferro sono state correlate a un incrementato rischio di MP.
Il consumo di tabacco e caffè, unitamente all'attività fisica regolare, ha mostrato effetti protettivi, sebbene di modesta entità. È fondamentale sottolineare che la maggioranza dei casi di MP rimane idiopatica, suggerendo un'eziologia multifattoriale.

\subsection{Presentazione  clinica}
La sintomatologia del Morbo di Parkinson manifesta un quadro clinico caratterizzato da manifestazioni motorie cardinali e sintomatologia non motoria associata. Il complesso sintomatologico motorio comprende tremore a riposo spesso asimmetrico con frequenza di 4-6 Hz, tipicamente descritto come "pill-rolling", bradicinesia manifestantesi con rallentamento motorio, ipomimia e ridotta oscillazione pendolare degli arti superiori durante la deambulazione, rigidità muscolare ("lead-pipe" o fenomeno della ruota dentata), e instabilità posturale documentabile attraverso il test della retropulsione. La deambulazione risulta caratterizzata da una progressione a piccoli passi con tendenza allo strascicamento e ridotta oscillazione degli arti superiori.
La sintomatologia accessoria include disartria con eloquio esplosivo secondario a incoordinazione linguo-diaframmatica, movimenti involontari della lingua con conseguente difficoltà protrusiva, e incremento della frequenza di ammiccamento palpebrale, quest'ultimo in contrasto con quanto osservato nella corea di Huntington. La disfunzione autonomica, i disturbi olfattivi, la sintomatologia algica, le alterazioni sensitive e i disturbi timici costituiscono il corredo sintomatologico non motorio. Il deterioramento cognitivo, con particolare coinvolgimento delle funzioni attentive, può manifestarsi e progredire nel decorso della patologia.
La progressione temporale della malattia evidenzia un esordio tipicamente unilaterale con successiva bilateralizzazione, manifestandosi prevalentemente nella sesta decade di vita. La responsività alla terapia dopaminergica, in particolare alla levodopa, rappresenta un elemento caratteristico, sebbene il tremore possa risultare farmacoresistente, in contrasto con la significativa risposta della bradicinesia e della rigidità. La variabilità fenotipica interindividuale costituisce un elemento distintivo della patologia.

\subsection{Approccio diagnostico}
L'iter diagnostico della malattia di Parkinson si fonda primariamente sulla valutazione clinica, data l'assenza di biomarcatori patognomonici. La diagnosi richiede la documentazione di bradicinesia associata ad almeno un sintomo cardine tra tremore a riposo o rigidità, valutati mediante la scala MDS-UPDRS standardizzata.
L'approccio diagnostico contempla un'accurata anamnesi ed esame obiettivo neurologico, focalizzati sull'identificazione dei sintomi cardinali: bradicinesia, tremore a riposo (4-6 Hz) tipicamente asimmetrico, rigidità e instabilità posturale. La responsività alla terapia dopaminergica, particolarmente evidente per bradicinesia e rigidità, costituisce un elemento diagnostico supportivo significativo, mentre una mancata risposta a dosaggi adeguati di levodopa suggerisce diagnosi alternative.
L'esclusione di parkinsonismi secondari richiede particolare attenzione all'insorgenza temporale dei sintomi e alla distribuzione topografica del coinvolgimento motorio. La diagnostica per immagini, sebbene non necessaria nelle presentazioni cliniche tipiche con adeguata risposta alla levodopa, può includere RM cerebrale, particolarmente utile mediante sequenze SWI per la valutazione del "swallow tail sign" nigrostriatale. La SPECT con 123I-FP-CIT (DaTscan) documenta la disfunzione dopaminergica presinaptica, mentre la PET con FDG consente la differenziazione metabolica tra PD e sindromi parkinsoniane atipiche.
L'analisi genetica, indicata in casi selezionati (esordio precoce, familiarità positiva, specifiche etnie), e la valutazione autonomica mediante scintigrafia miocardica con MIBG, che evidenzia la denervazione simpatica caratteristica, completano l'iter diagnostico. L'ecografia transcranica può evidenziare l'iperecogenicità della sostanza nera, supportando la diagnosi differenziale.
I criteri MDS stratificano la diagnosi in PD clinicamente stabilita e probabile, bilanciando specificità e sensibilità diagnostica nella pratica clinica.

\begin{Oss}
	La scala MDS-UPDRS (Movement Disorder Society-Unified Parkinson's Disease Rating Scale) è uno strumento di valutazione clinica ampiamente utilizzato per quantificare la gravità dei sintomi motori e non motori della malattia di Parkinson. Questa scala è stata sviluppata per migliorare la consistenza nella valutazione dei sintomi e per integrare meglio gli aspetti non motori della PD.
	Struttura: La scala MDS-UPDRS è composta da quattro sezioni:
	\begin{description}
		\item[Sezione I]{Esperienze non motorie della vita quotidiana. Questa sezione valuta aspetti come le capacità cognitive, i disturbi comportamentali e dell'umore}
		\item [Sezione II]{Esperienze motorie della vita quotidiana. Questa sezione valuta l'impatto dei sintomi motori sulle attività quotidiane}
		\item [Sezione III]{Esame motorio. Questa sezione valuta i segni motori della PD attraverso un esame clinico, come il tremore, la rigidità e la bradicinesia}
		\item[Sezione IV]{Complicanze della terapia. Questa sezione valuta le complicanze associate al trattamento farmacologico}
	\end{description}
	Il punteggio totale per le sezioni I-IV varia da 0 (nessuna disabilità) a 199 (disabilità totale). La sezione III, che valuta i sintomi motori, ha un punteggio che varia da 0 a 132.
	Oltre alla scala MDS-UPDRS, esistono altre scale di valutazione utilizzate nella PD, come la scala di Hoehn e Yahr e la scala di Schwab e England. La scala di Hoehn e Yahr valuta la gravità della malattia da 0 (nessuna malattia) a 5 (paziente costretto su sedia a rotelle o allettato senza assistenza).
\end{Oss}

\subsection{Anatomia patologica}

\subsection{Imaging}

\subsection{Trattamento e prognosi}

\subsection{Checklist di refertazione}

\subsection{Bibliografia}
\small{
	
	
}

\note{Nota a margine}
\expl{Nota a margine colorata}