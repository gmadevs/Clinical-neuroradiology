\section{Morbo di Parkinson}

\subsection{Definizione}

\subsection{Eziologia}

\subsection{Epidemiologia}
Il MP costituisce una delle principali cause di disabilità e mortalità nell'ambito delle patologie neurologiche, con una prevalenza particolarmente elevata negli Stati Uniti e in Canada (160-180 casi/100.000 abitanti). L'incidenza annuale in Nord America oscilla tra 108 e 212 casi ogni 100.000 individui di età $\geq$65 anni, con una prevalenza dello 0,3\% nella popolazione adulta $\geq$40 anni e dell'1,6\% nei soggetti ultrasessantacinquenni.
L'esordio della patologia mostra una significativa correlazione con l'età, manifestandosi tipicamente dalla quinta decade di vita, con un'età media alla diagnosi di 70,5 anni e una finestra di esordio prevalente tra i 45 e i 70 anni. Una variante giovanile può presentarsi tra i 20 e i 40 anni, sebbene l'esordio prima dei 30 anni risulti infrequente. La distribuzione per sesso evidenzia una predominanza maschile, particolarmente accentuata nella fascia d'età 50-60 anni.
L'eziologia del MP comprende fattori di rischio genetici, con particolare rilevanza nelle forme ad esordio precoce, e ambientali, tra cui l'esposizione a pesticidi e inquinanti atmosferici. Sono stati identificati fattori protettivi, inclusi il consumo di caffè, l'attività fisica e il fumo di sigaretta. La patologia si presenta prevalentemente in forma sporadica (85-90\% dei casi), mentre una minoranza dei casi (10-15\%) presenta familiarità positiva.

\subsubsection{Fattori di rischio}
L'eziopatogenesi del Morbo di Parkinson (MP) presenta una complessa interazione di fattori di rischio genetici, ambientali e non modificabili. L'anamnesi familiare positiva per MP in consanguinei di primo grado comporta un incremento del rischio relativo di 2-3 volte. Le forme monogeniche, rappresentanti meno del 10\% della casistica totale, manifestano pattern di ereditarietà autosomica dominante, recessiva o X-linked, caratterizzandosi per un esordio precoce rispetto alle forme sporadiche.
Le mutazioni eterozigoti del gene GBA1 costituiscono un significativo fattore di rischio genetico, unitamente ad alterazioni di altri geni codificanti per enzimi lisosomiali. Il coinvolgimento di geni quali SNCA, LRRK2, VPS35, Parkin, PINK1 e DJ-1 è stato ampiamente documentato. Particolare rilevanza assumono le mutazioni del gene Nurr1, determinante per l'identità neuronale dopaminergica, e del gene DJ-1, cruciale nella risposta allo stress ossidativo. Le alterazioni del gene PINK1, codificante per una chinasi mitocondriale, e del gene Park2, responsabile della sintesi della proteina parkina, sono associate a forme ad esordio precoce.
L'esposizione a neurotossine ambientali, inclusi mercurio, manganese, disolfuro di carbonio, solventi organici, MPTP e monossido di carbonio, può indurre degenerazione nigrostriatale e parkinsonismo. L'uso di neurolettici e l'abuso endovenoso di efedrone possono causare sindromi parkinsoniane potenzialmente irreversibili. Traumi cranici ripetuti, pesticidi, solventi e inquinamento atmosferico rappresentano ulteriori fattori di rischio ambientale documentati.
Tra i fattori non modificabili, l'età avanzata e il sesso maschile emergono come significativi predittori di rischio, con predominanza nella sesta decade di vita. Comorbidità quali depressione, ansia, stipsi, diabete mellito tipo 2, obesità e alterazioni del metabolismo del ferro sono state correlate a un incrementato rischio di MP.
Il consumo di tabacco e caffè, unitamente all'attività fisica regolare, ha mostrato effetti protettivi, sebbene di modesta entità. È fondamentale sottolineare che la maggioranza dei casi di MP rimane idiopatica, suggerendo un'eziologia multifattoriale.

\subsection{Presentazione  clinica}

\subsection{Approccio diagnostico}

\subsection{Anatomia patologica}

\subsection{Imaging}

\subsection{Trattamento e prognosi}

\subsection{Checklist di refertazione}

\subsection{Bibliografia}
\small{
	
	
}

\note{Nota a margine}
\expl{Nota a margine colorata}