\chapter{Temporal lobe epilepsy}

\subsection{Mesial temporal sclerosis}

\textbf{Mesial temporal sclerosis}, also commonly referred to as \textbf{hippocampal sclerosis}, is the most common association with intractable temporal lobe epilepsy . It is seen in up to 65\% of autopsy studies, although significantly less in imaging.

\paragraph{Clinical presentation}

Most patients present with temporal lobe epilepsy.

\subparagraph{Febrile seizures}

The relationship, if any, of mesial temporal sclerosis with febrile seizures is controversial, and made more difficult due to the relative insensitivity of imaging and the difficulty in establishing whether a particular seizure was truly febrile. Up to a third of patients with established refractory temporal lobe epilepsy have a history of seizures in childhood at the time of fever . Follow-up of children with febrile seizures does not demonstrate a significantly increased incidence of temporal lobe epilepsy .

\paragraph{Pathology}

The hippocampal formation is not uniformly affected, with the dentate gyrus, and the CA1, CA4, and to a lesser degree CA3 sections of the hippocampus being primarily involved . Histologically there is neuronal cell loss, gliosis, and sclerosis.

\subparagraph{Etiology}

Controversy exists as to the causative mechanism: is mesial temporal sclerosis a result of temporal lobe epilepsy or vice versa ? In children with newly diagnosed epilepsy, only \textasciitilde{} 1\% have evidence of mesial temporal sclerosis on imaging . Furthermore, in adults 3-10\% of cases of mesial temporal sclerosis demonstrate bilateral changes  even though symptoms may be unilateral.

\paragraph{Radiographic features}

\subparagraph{MRI}

MRI is the modality of choice to evaluate the hippocampus, however dedicated temporal lobe epilepsy protocol needs to be performed if good sensitivity and specificity is to be achieved . Thin section angled coronal sequences at right angles to the longitudinal axis of the hippocampus are required, to minimize volume averaging.

Coronal volume and coronal high resolution T2WI/FLAIR are best to diagnose mesial temporal sclerosis.

Findings include :

\begin{itemize}
	\item
	reduced hippocampal volume: hippocampal atrophy
	\item
	increased T2 signal
	\item
	abnormal morphology: loss of internal architecture (interdigitations of hippocampus),stratum radiata, a thin layer of white matter separates the dentate gyrus and Ammon horn
\end{itemize}

Although comparing left to right side is easiest, it must be remembered that up to 10\% of cases are bilateral, and thus if symmetry is the only feature being evaluated, many cases may be misinterpreted as normal.

Often mentioned, but probably one of the least specific findings, is enlargement of the temporal horn of the lateral ventricle . If anything, care must be taken not to allow an enlarged horn to trick you into thinking the hippocampus is reduced in size.

When severe and long standing, additional associated findings include :

\begin{itemize}
	\item
	atrophy of the ipsilateral fornix and mammillary body
	\item
	increased signal and or atrophy of the anterior thalamic nucleus
	\item
	atrophy of the cingulate gyrus
	\item
	increased signal and/or reduction in the volume of the amygdala
	\item
	reduction in the volume of the subiculum
	\item
	dilatation of temporal horn and temporal lobe atrophy
	\item
	collateral white matter and entorhinal cortex atrophy
	\item
	thalamic and caudate atrophy
	\item
	ipsilateral cerebral atrophy
	\item
	contralateral cerebellar hemiatrophy
	\item
	loss of grey-white matter interface in the anterior temporal lobe 
	\item
	reduced white matter volume in the parahippocampal gyrus 
\end{itemize}

Additional 3D volumetric studies can be performed, and although time consuming to post-process may be more sensitive to subtle hippocampal volume loss. Gadolinium is not required .

T2 relaxometry may also be useful in detecting cases of hippocampal sclerosis .

Diffusion MRI

As a result of neuronal loss, the extracellular space is enlarged and thus diffusion of water molecules is greater on the affected side, resulting in increased values on the affected side (higher signal on ADC).

Conversely, due to neuronal dysfunction and swelling, diffusion is restricted following a seizure, and thus values are lower .

MR spectroscopy

MR spectroscopy findings typically represent neuronal dysfunction :

\begin{itemize}
	\item
	decreased NAA and decreased NAA/Cho and NAA/Cr ratios
	\item
	decreased MI in ipsilateral temporal lobe
	\item
	increased lipid and lactate soon after a seizure
\end{itemize}

MR perfusion

MR perfusion demonstrates similar changes to SPECT (see below) with blood perfusion depending on when the scan is obtained.

During the peri-ictal phases, perfusion is increased, not only in the mesial temporal lobe but often in large parts of temporal lobe and hemisphere. In interictal periods, conversely,perfusion is reduced .

\subparagraph{Nuclear medicine}

SPECT(Tc-99m HMPAO or ECD) and PET(F18-FDG) imaging are also useful adjuncts, with both ictal and interictal scans demonstrating abnormalities:

\begin{itemize}
	\item
	\textbf{ictal scan:}hyperperfusion
	\item
	\textbf{interictal scan:}hypoperfusion
\end{itemize}

Other causes of temporal lobe epilepsy (TLE) should be considered, especially as small temporal lobe cortical tumors can have similar appearances.

\paragraph{Treatment and prognosis}

Temporal lobe epilepsy is initially managed medically with antiseizure medications. In patients who are refractory to medical management temporal lobectomy or selective amygdalohippocampectomy may be performed. Anterior temporal lobectomy is successful in 75-90\% of patients with mesial temporal sclerosis.