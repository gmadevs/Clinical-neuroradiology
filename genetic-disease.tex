\chapter{Genetic disease}
\subsection{Leigh syndrome}

\textbf{Leigh syndrome}, also known as \textbf{subacute necrotizing encephalomyelopathy (SNEM)}, is a mitochondrial disorder characterized by progressive neurodegeneration, mitochondrial dysfunction, and bilateral central nervous system lesions, that invariably leads to death, usually in childhood.

\paragraph{Epidemiology}

Leigh syndrome is rare, encountered in approximately 1 in 40,000 births, although some populations have much higher incidence (e.g. in Quebec in Canada) . There is no known gender or racial predilection .

\paragraph{Clinical presentation}

Typically, symptoms become evident before the age of 2 years, with a presentation in later childhood (juvenile form) or adulthood (adult form) being very rare . Later onset forms tend to progress more slowly .

The clinical presentation is incredibly heterogenous and often triggered by a metabolic challenge (e.g. infection), but potential features include (in approximate order of most to least common) :

\begin{itemize}
	\item
	psychomotor or developmental delay/regression
	
	\begin{itemize}
		\item
		overall the most common clinical feature
		\item
		more common in early onset disease
	\end{itemize}
	\item
	hypotonia
	\item
	respiratory dysfunction
	\item
	poor feeding and dysphagia
	\item
	seizures (focal and/or generalized types) and epilepsy
	\item
	neuromuscular weakness
	
	\begin{itemize}
		\item
		more common in later onset disease
	\end{itemize}
	\item
	ocular signs: nystagmus, ptosis, ophthalmoplegia, optic atrophy, retinopathy
	\item
	ataxia
	
	\begin{itemize}
		\item
		more common in later onset disease
	\end{itemize}
	\item
	dystonia
	\item
	sensorineural hearing loss
\end{itemize}

Less commonly, there is involvement of organs beyond the central nervous system, including cardiac, gastrointestinal and renal manifestations . Hypertrichosis has been described in some patients specifically with a \emph{SURF1} mutation .

\paragraph{Pathology}

\subparagraph{Genetics}

Leigh syndrome is one of many mitochondrial disorders. It is caused by dysfunction or deficiency of one or multiple of the complexes comprising the electron transport chain in mitochondria .

These complexes are encoded by either nuclear DNA (nDNA) or mitochondrial DNA (mtDNA), and thus, mutations in various genes from each genome can result in Leigh syndrome . Later onset phenotypes are more often associated with mtDNA mutations .

Commonly implicated genes in Leigh syndrome include :

\begin{itemize}
	\item
	mtDNA:
	
	\begin{itemize}
		\item
		\emph{MT-ATP6} gene which encodes for complex V
		\item
		\emph{MT-ND} genes (e.g. \emph{MT-ND5}) which encode for complex I
	\end{itemize}
	\item
	nDNA: \emph{SURF1} gene which encodes for complex IV
\end{itemize}

Given the wide range of genes, there is a range of inheritance patterns observed in Leigh syndrome . This includes mitochondrial (i.e. maternal) or autosomal recessive patterns of inheritance, and very rarely, also X-linked recessive and autosomal dominant patterns of inheritance .

\subparagraph{Microscopic appearance}

Brain biopsy is rarely performed but shows features of chronic energy deprivation leading to histological features such as :

\begin{itemize}
	\item
	spongiform degeneration
	\item
	capillary proliferation
	\item
	demyelination
	\item
	neuronal loss
	\item
	gliosis
\end{itemize}

These findings are similar to those seen in infarction .

\subparagraph{Markers}

CSF and serum lactate levels are usually elevated .

\paragraph{Radiographic features}

\subparagraph{CT}

CT demonstrates regions of low-density matching areas of the abnormal T2 signal on MRI (see below) . Occasionally some of these areas can show contrast enhancement .

\subparagraph{MRI}

MRI abnormalities are heterogeneous, although no definite differences have been consistently noted between different mutations . Generally, the distribution tends to be symmetrical and some lesions may regress with time, while others may be permanent and evolve .

\begin{itemize}
	\item
	\textbf{T2:}high signal 
	
	\begin{itemize}
		\item
		basal ganglia (most common), in particular the putamen
		\item
		brainstem, can affect any level but in particular the midbrain (e.g. periaqueductal grey matter) and medulla oblongata
		
		\begin{itemize}
			\item
			the double panda sign has been rarely reported 
		\end{itemize}
		\item
		thalami
		\item
		subthalamic nuclei and dentate nuclei 
		\item
		involvement of cerebral white matter, cerebellar white matter, or spinal cord is uncommon\hspace{0pt} 
	\end{itemize}
	\item
	\textbf{T1:}usually demonstrates reduced signal in regions of T2 high signal, although some areas of hyperintensity can be seen, as can some enhancement post-gadolinium administration
	\item
	\textbf{DWI:}in the acute setting high diffusion signal may be evident (so-called 'stroke-like' lesions)
	\item
	\textbf{MR spectroscopy}
	
	\begin{itemize}
		\item
		elevated choline
		\item
		occasionally elevated lactate
		\item
		reduced NAA
	\end{itemize}
\end{itemize}

\paragraph{Treatment and prognosis}

There is no disease-modifying therapy available , although a 'mitochondrial cocktail' is often prescribed, variably consisting of components such as thiamine (vitamin B1), coenzyme Q-10, riboflavin (vitamin B2) and vitamin C . Generally, medications which are mitochondrial toxins (e.g. sodium valproate) should be avoided .

Prognosis is poor, with death usually occurring in childhood. The later the onset, the slower the deterioration. Death is most frequently due to respiratory failure .

The factors associated with a worse outcome are :

\begin{itemize}
	\item
	disease onset before 6 months of age
	\item
	admission to an intensive care
	\item
	brainstem lesions
	\item
	MRS lactate peak
\end{itemize}

\paragraph{Differential diagnosis}

\begin{itemize}
	\item
	Wernicke encephalopathy
	
	\begin{itemize}
		\item
		similar appearance but typically presents in a very different patient demographic
		\item
		mammillary bodies not typically involved in Leigh syndrome
		\item
		enhancement more common in Wernicke encephalopathy
		\item
		hemorrhagic change more common in Wernicke encephalopathy 
	\end{itemize}
	\item
	other mitochondrial disorders (e.g. MEGDEL syndrome)
	
	\begin{itemize}
		\item
		brainstem and basal ganglia involvement less pronounced
	\end{itemize}
	\item
	acute necrotizing encephalitis of childhood
	
	\begin{itemize}
		\item
		lactate levels are usually normal
	\end{itemize}
	\item
	biotin-thiamine-responsive basal ganglia disease
	\item
	toxic encephalopathies (e.g. carbon monoxide poisoning)
\end{itemize}
\section{Branchio-oculo-facial syndrome}

\textbf{Branchio-oculo-facial syndrome}(\textbf{BOFS}) is a very rare autosomal dominant genetic disorder that is characterized clinically by abnormalities affecting the eyes, craniofacial structures, and branchial sinuses.

\paragraph{Epidemiology}

More than 80 cases have been reported in the global literature since its first description as a distinct entity in 1987 .

\paragraph{Clinical presentation}

Like many syndromes of this kind, the spectrum of phenotypic variability is fairly marked. Nevertheless, frequently observed features include:

Ocular anomalies

\begin{itemize}
	\tightlist
	\item
	microphthalmia
	\item
	stenotic nasolacrimal ducts
	\item
	coloboma
	\item
	premature cataracts
\end{itemize}

Craniofacial dysmorphism

\begin{itemize}
	\tightlist
	\item
	cleft lip and/or palate
	\item
	pseudocleft of the upper lip: highly specific for this syndrome
	\item
	wide nasal bridge
	\item
	high forehead
	\item
	early greying of the hair
\end{itemize}

Branchial sinuses

\begin{itemize}
	\tightlist
	\item
	skin clefts
	\item
	scarring of the skin
\end{itemize}

Renal anomalies

\begin{itemize}
	\tightlist
	\item
	agenesis of the kidneys
	\item
	dysplastic/hypoplastic kidneys
\end{itemize}

\paragraph{Pathology}

Phenotypically it bears similarities to branchio-oto-renal syndromeand branchio-otic syndrome, and at one time the three syndromes were felt to be allelic variants of an identical underlying condition, however, it is now known that the genotypes are different and the three conditions are considered to be distinct entities .

\subparagraph{Genetics}

The gene responsible for the branchio-oculo-facial syndrome is \emph{TFAP2A}, which encodes for activating enhancer-binding protein 2 alpha (AP-2α), a transcription factor important in the development of several major craniofacial structures, with a crucial role in normal embryogenesis of the eyes. Multiple mutations of \emph{TFAP2A} have now been described .