\chapter{Genetic disease}
\section{Branchio-oculo-facial syndrome}

\textbf{Branchio-oculo-facial syndrome}(\textbf{BOFS}) is a very rare autosomal dominant genetic disorder that is characterized clinically by abnormalities affecting the eyes, craniofacial structures, and branchial sinuses.

\paragraph{Epidemiology}

More than 80 cases have been reported in the global literature since its first description as a distinct entity in 1987 .

\paragraph{Clinical presentation}

Like many syndromes of this kind, the spectrum of phenotypic variability is fairly marked. Nevertheless, frequently observed features include:

Ocular anomalies

\begin{itemize}
	\tightlist
	\item
	microphthalmia
	\item
	stenotic nasolacrimal ducts
	\item
	coloboma
	\item
	premature cataracts
\end{itemize}

Craniofacial dysmorphism

\begin{itemize}
	\tightlist
	\item
	cleft lip and/or palate
	\item
	pseudocleft of the upper lip: highly specific for this syndrome
	\item
	wide nasal bridge
	\item
	high forehead
	\item
	early greying of the hair
\end{itemize}

Branchial sinuses

\begin{itemize}
	\tightlist
	\item
	skin clefts
	\item
	scarring of the skin
\end{itemize}

Renal anomalies

\begin{itemize}
	\tightlist
	\item
	agenesis of the kidneys
	\item
	dysplastic/hypoplastic kidneys
\end{itemize}

\paragraph{Pathology}

Phenotypically it bears similarities to branchio-oto-renal syndromeand branchio-otic syndrome, and at one time the three syndromes were felt to be allelic variants of an identical underlying condition, however, it is now known that the genotypes are different and the three conditions are considered to be distinct entities .

\subparagraph{Genetics}

The gene responsible for the branchio-oculo-facial syndrome is \emph{TFAP2A}, which encodes for activating enhancer-binding protein 2 alpha (AP-2α), a transcription factor important in the development of several major craniofacial structures, with a crucial role in normal embryogenesis of the eyes. Multiple mutations of \emph{TFAP2A} have now been described .