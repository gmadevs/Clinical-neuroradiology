\chapter{Miscellaneous tumors or tumor-like conditions}

\subsection{Juvenile nasopharyngeal angiofibroma}

\textbf{Juvenile nasopharyngeal angiofibromas} are a rare benign, but locally aggressive, vascular tumors that occur almost exclusively in young men; usually between the ages of 10 and 18.

On imaging, they present as vividly enhancing soft-tissue masses centered on the sphenopalatine foramen. Given its vascularity, prominent flow voids are seen on MRI leading to a salt and pepper appearance.

\paragraph{Epidemiology}

Juvenile nasopharyngeal angiofibromas occur almost exclusively in males and usually in adolescence (\textasciitilde15 years) . They account for only 0.5\% of all head and neck tumors .

\subparagraph{Associations}

\begin{itemize}
	\item
	red hair and fair skin in White patients 
	\item
	familial adenomatous polyposis (rare) 
\end{itemize}

\paragraph{Clinical presentation}

The presentation is typical with obstructive symptoms, epistaxis, and chronic otomastoiditis due to obstruction of the Eustachian tube. Patients may present with life-threatening epistaxis.On examination, it may be seen as a pale reddish-blue mass. It is, as the name suggests, very vascular and a biopsy can sometimes be fatal.

\paragraph{Pathology}

Juvenile nasopharyngeal angiofibromas are benign but highly vascular tumors which may be locally aggressive. The tumors express androgen receptors , which may explain their growth during puberty, regression after puberty or with estrogen administration, and occurring almost exclusively in males .

The exact site of origin is contentious as these masses usually present when they have reached a considerable size. However, most authors agree that they arise from the posterior choanal tissues in the region of the sphenopalatine foramen.

\subparagraph{Staging}

See staging of juvenile nasopharyngeal angiofibromas.

\paragraph{Radiographic features}

Imaging plays an important role in diagnosis, as well as staging,as biopsies should be avoided due to the risk of brisk hemorrhage,due to the tumor's vascular nature.

Although these masses are thought to arise from the region of the sphenopalatine foramen, they are usually sizable at diagnosis, frequently with extension medially into the nasopharynx, laterally into the pterygopalatine fossa and, over time, beyond into the orbit, paranasal sinuses, intracranial cavity and infratemporal fossa.

\subparagraph{Plain radiograph}

Plain radiographs no longer play a role in the workup of a suspected juvenile nasopharyngeal angiofibroma;however, they may still be obtained in some instances during the assessment of nasal obstruction or symptoms of sinus obstructions. Findings include :

\begin{itemize}
	\item
	visualization of a nasopharyngeal mass
	\item
	opacification of the sphenoid sinus
	\item
	anterior bowing of the posterior wall of the maxillary antrum(Holman-Miller sign)
	\item
	widening of the pterygomaxillary fissure and pterygopalatine fossa
	\item
	erosion of the medial pterygoid plate
\end{itemize}

\subparagraph{CT}

CT is particularly useful at delineating bony changes. Findings are similar to those described above. Typically a lobulated non-encapsulated soft tissue mass is demonstrated centered on the sphenopalatine foramen(which is often widened) and usually bowing the posterior wall of the maxillary antrum anteriorly. There is marked contrast enhancement following administration of contrast, reflecting the prominent vascularity.

Extensive bony destruction is usually not a feature; bone is rather remodeled or resorbed. This feature may be helpful in differentiation from other, more aggressive, lesions. Intracranial extension can, however, occur.

\subparagraph{Angiography (DSA)}

Angiography, although not essential, is often useful in both defining the feeding vessels and in preoperative embolization.Supply of these tumors is usually via :

\begin{itemize}
	\item
	external carotid artery: the majority
	
	\begin{itemize}
		\item
		internal maxillary artery
		\item
		ascending pharyngeal artery
		\item
		palatine arteries
	\end{itemize}
	\item
	internal carotid artery: less common, usually in larger tumors
	
	\begin{itemize}
		\item
		sphenoidal branches
		\item
		ophthalmic artery
	\end{itemize}
\end{itemize}

Of note, enlargement of feeding vessels is not a common finding .


\subparagraph{MRI}

MRI is excellent at evaluating tumor extension into the orbit and intracranial compartments.

\begin{itemize}
	\item
	\textbf{T1:}intermediate signal
	\item
	\textbf{T2:}heterogeneous signal: flow voids appear dark
	\item
	\textbf{T1 C+ (Gd):}shows prominent enhancement
\end{itemize}

The presence of prominent flow voids leads to a salt and pepper appearance on most sequences and are characteristic .


\paragraph{Treatment and prognosis}

Surgical resection (either open or, increasingly, endoscopic) is the treatment of choice, usually performed after preoperative embolization to help with hemostasis. The embolization may be performed up to five days prior to surgery. Irradiation may be an option if surgery is not possible or only incomplete resection has been achieved .

In cases where there is skull base involvement, a high recurrence rate (up to 50\%) has been reported .


\paragraph{Differential diagnosis}

Imaging differential considerations include:

\begin{itemize}
	\item
	angiomatous polyp:variant of a sinonasal polyp, located toward ostium/hardly extend to the nasopharynx, elderly age, less vascularity
	\item
	rhabdomyosarcoma (head and neck)
	\item
	nasopharyngeal carcinoma (NPC)
	\item
	nasopharyngeal teratoma
	\item
	nasopharyngeal lymphoma
	\item
	lymphangioma: no contrast enhancement
	\item
	encephalocele: no contrast enhancement
	\item
	esthesioneuroblastoma
\end{itemize}
\subsection{Langerhans cell histiocytosis}

\textbf{Langerhans cell histiocytosis (LCH)} is a rare multisystem disease with a wide and heterogeneous clinical spectrum and variable extent of involvement.

\paragraph{Terminology}

Langerhans cell histiocytosis was previously known as histiocytosis X. The newer term is preferred as it is more descriptive of its cellular background, and removes the ambiguity of the connotation "X".

Historically, the condition was also subdivided into three distinct entities,Letterer-Siwe disease,Hand-Schüller-Christian diseaseand eosinophilic granuloma, see below.

\paragraph{Epidemiology}

The disease is more common in the pediatric population, with a peak incidence between one and three years of age . Incidence is estimated at \textasciitilde5 per million children, and 1-2 cases per million adults .There is also a male predilection (M:F\textasciitilde1.5:1) .

\paragraph{Clinical presentation}

Essentially any part of the body can be affected and as such,clinical presentation will depend on specific involvement. The course of the disease ranges from those that spontaneously regress to those that have a rapidly progressive course (the latter is especially common in young children with multisystem disease).

Historically, three forms (two with eponymous names) have been recognized, although there is some confusion as to their definition :

\begin{itemize}
	\item
	Letterer-Siwe disease
	
	\begin{itemize}
		\item
		disseminated multiorgan disease
		\item
		typically young children/infants less than one year old
		\item
		fulminant course with poor prognosis
	\end{itemize}
	\item
	Hand-Schüller-Christian disease
	
	\begin{itemize}
		\item
		multiple lesions
		
		\begin{itemize}
			\item
			some authors confine the term to patients with solitary organ involvement 
			\item
			other authors accept multiorgan involvement (e.g. bone and spleen)
		\end{itemize}
		\item
		confined to the one location (usually bone)
		\item
		typically children
		\item
		intermediate prognosis
	\end{itemize}
	\item
	eosinophilic granuloma (EG)
	
	\begin{itemize}
		\item
		lesions are confined to one organ system
		
		\begin{itemize}
			\item
			some authors confine the term to patients with a solitary lesion 
			\item
			other authors accept multiple lesions 
		\end{itemize}
		\item
		70\% of cases affect bone
		\item
		typically children
		\item
		best prognosis
	\end{itemize}
\end{itemize}

A more useful and less controversial classification, which roughly correlates to the eponymous diseases above, is as follows:

\begin{itemize}
	\item
	multiple organ systems, multiple sites involved
	\item
	single organ system, multiple sites involved
	\item
	single lesion
\end{itemize}

Additionally, in 2008 the WHO recommended distinguishing Langerhans cell histiocytosis from a more pleomorphic variant known as Langerhans cell sarcoma.

As well as systemic disease, individual organ systems may be involved, which will be discussed separately:

\begin{itemize}
	\item
	skeletal manifestations of LCH
	\item
	central nervous system manifestations of LCH
	\item
	pulmonary manifestations of LCH
	\item
	salivary gland manifestations of LCH
	\item
	hepatobiliary manifestations of LCH
	\item
	gastrointestinal manifestations of LCH
\end{itemize}

The remainder of this article concerns a general overview of Langerhans cell histiocytosis.

\paragraph{Pathology}

Langerhans cell histiocytosis is due to uncontrolled monoclonal proliferation of Langerhans cells (distinctive cells of monocyte-macrophage lineage) and should be considered a malignancy although its biological behavior is very variable .An immune-mediated mechanism has been postulated. This proliferation is accompanied by inflammation and granuloma formation. Electron microscopy may reveal characteristic Birbeck granules.Immunohistochemistry reveals expression of the following antigens:

\begin{itemize}
	\item
	HLA-DR
	\item
	CD1a
	\item
	CD207 (langerin)
	\item
	S100
\end{itemize}

\paragraph{Radiographic features}

Imaging features are often not pathognomonic and tissue diagnosis is usually required for definitive diagnosis. As Langerhans cell histiocytosis can affect most organ systems, radiographic appearances are discussed separately (see above).

\paragraph{Treatment and prognosis}

The prognosis can be extremely variable with eosinophilic granuloma limited to bone carrying the best and Letterer-Siwe disease carrying the worst prognosis, respectively. The prognosis is more closely related to the disease burden rather than histological features, although frankly malignant features (Langerhans cell sarcoma) do also have an impact on survival :

\begin{itemize}
	\item
	unifocal disease (eosinophilic granuloma): \textgreater95\% survival
	\item
	two organ involvement: 75\% survival
	\item
	Langerhans cell sarcoma: 50\% survival
\end{itemize}