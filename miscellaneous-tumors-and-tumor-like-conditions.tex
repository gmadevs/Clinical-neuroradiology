\chapter{Miscellaneous tumors or tumor-like conditions}

\subsection{Langerhans cell histiocytosis}

\textbf{Langerhans cell histiocytosis (LCH)} is a rare multisystem disease with a wide and heterogeneous clinical spectrum and variable extent of involvement.

\paragraph{Terminology}

Langerhans cell histiocytosis was previously known as histiocytosis X. The newer term is preferred as it is more descriptive of its cellular background, and removes the ambiguity of the connotation "X".

Historically, the condition was also subdivided into three distinct entities,Letterer-Siwe disease,Hand-Schüller-Christian diseaseand eosinophilic granuloma, see below.

\paragraph{Epidemiology}

The disease is more common in the pediatric population, with a peak incidence between one and three years of age . Incidence is estimated at \textasciitilde5 per million children, and 1-2 cases per million adults .There is also a male predilection (M:F\textasciitilde1.5:1) .

\paragraph{Clinical presentation}

Essentially any part of the body can be affected and as such,clinical presentation will depend on specific involvement. The course of the disease ranges from those that spontaneously regress to those that have a rapidly progressive course (the latter is especially common in young children with multisystem disease).

Historically, three forms (two with eponymous names) have been recognized, although there is some confusion as to their definition :

\begin{itemize}
	\item
	Letterer-Siwe disease
	
	\begin{itemize}
		\item
		disseminated multiorgan disease
		\item
		typically young children/infants less than one year old
		\item
		fulminant course with poor prognosis
	\end{itemize}
	\item
	Hand-Schüller-Christian disease
	
	\begin{itemize}
		\item
		multiple lesions
		
		\begin{itemize}
			\item
			some authors confine the term to patients with solitary organ involvement 
			\item
			other authors accept multiorgan involvement (e.g. bone and spleen)
		\end{itemize}
		\item
		confined to the one location (usually bone)
		\item
		typically children
		\item
		intermediate prognosis
	\end{itemize}
	\item
	eosinophilic granuloma (EG)
	
	\begin{itemize}
		\item
		lesions are confined to one organ system
		
		\begin{itemize}
			\item
			some authors confine the term to patients with a solitary lesion 
			\item
			other authors accept multiple lesions 
		\end{itemize}
		\item
		70\% of cases affect bone
		\item
		typically children
		\item
		best prognosis
	\end{itemize}
\end{itemize}

A more useful and less controversial classification, which roughly correlates to the eponymous diseases above, is as follows:

\begin{itemize}
	\item
	multiple organ systems, multiple sites involved
	\item
	single organ system, multiple sites involved
	\item
	single lesion
\end{itemize}

Additionally, in 2008 the WHO recommended distinguishing Langerhans cell histiocytosis from a more pleomorphic variant known as Langerhans cell sarcoma.

As well as systemic disease, individual organ systems may be involved, which will be discussed separately:

\begin{itemize}
	\item
	skeletal manifestations of LCH
	\item
	central nervous system manifestations of LCH
	\item
	pulmonary manifestations of LCH
	\item
	salivary gland manifestations of LCH
	\item
	hepatobiliary manifestations of LCH
	\item
	gastrointestinal manifestations of LCH
\end{itemize}

The remainder of this article concerns a general overview of Langerhans cell histiocytosis.

\paragraph{Pathology}

Langerhans cell histiocytosis is due to uncontrolled monoclonal proliferation of Langerhans cells (distinctive cells of monocyte-macrophage lineage) and should be considered a malignancy although its biological behavior is very variable .An immune-mediated mechanism has been postulated. This proliferation is accompanied by inflammation and granuloma formation. Electron microscopy may reveal characteristic Birbeck granules.Immunohistochemistry reveals expression of the following antigens:

\begin{itemize}
	\item
	HLA-DR
	\item
	CD1a
	\item
	CD207 (langerin)
	\item
	S100
\end{itemize}

\paragraph{Radiographic features}

Imaging features are often not pathognomonic and tissue diagnosis is usually required for definitive diagnosis. As Langerhans cell histiocytosis can affect most organ systems, radiographic appearances are discussed separately (see above).

\paragraph{Treatment and prognosis}

The prognosis can be extremely variable with eosinophilic granuloma limited to bone carrying the best and Letterer-Siwe disease carrying the worst prognosis, respectively. The prognosis is more closely related to the disease burden rather than histological features, although frankly malignant features (Langerhans cell sarcoma) do also have an impact on survival :

\begin{itemize}
	\item
	unifocal disease (eosinophilic granuloma): \textgreater95\% survival
	\item
	two organ involvement: 75\% survival
	\item
	Langerhans cell sarcoma: 50\% survival
\end{itemize}