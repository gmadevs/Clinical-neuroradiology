\chapter{General neuroanatomy}

\subsection{Neuroanatomy}

\textbf{Neuroanatomy} encompasses the anatomy of all structures of the:

\begin{itemize}
	\item
	central nervous system (CNS), includes brain and spinal cord
	\item
	peripheral nervous system (PNS)
	\item
	supporting tissues and structures
\end{itemize}

The functional description of neuroanatomy divides the nervous system into:

\begin{itemize}
	\item
	somatic nervous system
	\item
	autonomic nervous system
\end{itemize}

This anatomy section promotes the use of the Terminologia Anatomica,the international standard of anatomical nomenclature .

\subsection{Functional neuroanatomy}

\textbf{Functional neuroanatomy} is the study of the functional connections in the brain and spinal cord, distinct but interconnected with the structural or "more conventional"anatomic descriptions of the central nervous system. It focuses on the relationship between structure and function and hence it is vital in our understanding of behavior and emotion. Functional neuroanatomy has a strong basis on the embryological development of the neuroaxisand clinical neuroanatomy is heavily reliant on the functional descriptions of the central nervous system.

\subsection{Central nervous system}

The \textbf{central nervous system (CNS)}is the part of the nervous systemthat includes the brain and spinal cord.

\paragraph{Gross anatomy}

The main components of the CNS are the brain and spinal cord. In addition, the CNS includes the optic nerves (cranial nerve II), retinas, olfactory nerves (cranial nerve I), and olfactory epithelium.

\paragraph{Histology}

Macroscopically, the CNS consists of grey matter and white matter.

\subsection{Intracranial compartments}

Intracranial lesions or processes can be described according to which \textbf{intracranial compartment} they occur within.

The simplest division is into 3 compartments:

\begin{enumerate}
	\tightlist
	\item
	extra-axial: external to the brain parenchyma
	\item
	intra-axial: within the brain parenchyma
	\item
	intraventricular: within the ventricular system
\end{enumerate}

\subsection{Intra-axial}

\textbf{Intra-axial} is a term that denotes lesions that are within the brain parenchyma, in contrast to extra-axial, which describes lesions outside the brain, and intraventricular, which denotes lesions within the ventricular system. Some authors include intraventricular lesions in the intra-axial group as most are lesions that arise from the brain parenchyma and grow exophytically into the ventricular system.

Examples of intra-axial lesions include:

\begin{itemize}
	\tightlist
	\item
	neoplasm
	
	\begin{itemize}
		\tightlist
		\item
		primary
		
		\begin{itemize}
			\tightlist
			\item
			glioblastoma (GBM)
			\item
			astrocytoma
			\item
			primary CNS lymphoma
			\item
			ganglioglioma
			\item
			oligodendroglioma
		\end{itemize}
		\item
		cerebral metastases
	\end{itemize}
	\item
	infection
	
	\begin{itemize}
		\tightlist
		\item
		cerebral abscess
		\item
		neurocysticercosis
	\end{itemize}
	\item
	intracerebral hemorrhage (ICH)
\end{itemize}


\subsection{Extra-axial}

\textbf{Extra-axial} is a descriptive term to denote lesions that are external to the brain parenchyma, in contrast to intra-axial which describes lesions within the brain substance.

\paragraph{Radiographic features}

Often it is trivially easy to distinguish an intra-axial from an extra-axial mass. In many cases, especially when the mass is large and associated with parenchymal changes, such as edema, localization can be more difficult. A number of features are helpful in suggesting that a mass or lesion is extra-axial, including:

\begin{itemize}
	\tightlist
	\item
	subarachnoid space
	
	\begin{itemize}
		\tightlist
		\item
		CSF cleft sign
		\item
		widening of adjacent subarachnoid space/cistern
		\item
		intervening pial arteries or veins
	\end{itemize}
	\item
	brain parenchyma
	
	\begin{itemize}
		\tightlist
		\item
		absence of a claw sign
		\item
		intervening cortex between mass and white matter
		\item
		white matter buckling sign
	\end{itemize}
	\item
	bone and meninges
	
	\begin{itemize}
		\tightlist
		\item
		dural tail sign
		\item
		erosion,invasion or destruction of adjacent bone
		\item
		hyperostosis
	\end{itemize}
\end{itemize}

\paragraph{Examples}

Examples (non-exhaustive)of extra-axial lesions include:

\begin{itemize}
	\tightlist
	\item
	neoplasms
	
	\begin{itemize}
		\tightlist
		\item
		tumors of the meninges
		
		\begin{itemize}
			\tightlist
			\item
			meningioma
			\item
			hemangiopericytoma
		\end{itemize}
		\item
		pituitary tumors
		
		\begin{itemize}
			\tightlist
			\item
			macroadenoma
			\item
			craniopharyngioma
		\end{itemize}
		\item
		pineal parenchymal tumors
		\item
		cranial nerve schwannomas
		
		\begin{itemize}
			\tightlist
			\item
			trigeminal schwannoma
			\item
			vestibular schwannoma
		\end{itemize}
	\end{itemize}
	\item
	benign masses
	
	\begin{itemize}
		\tightlist
		\item
		arachnoid cyst
		\item
		dermoid cyst
		\item
		epidermoid cyst
		\item
		intracranial lipoma
	\end{itemize}
	\item
	hemorrhages
	
	\begin{itemize}
		\tightlist
		\item
		subarachnoid hemorrhage
		\item
		subdural hemorrhage
		\item
		extradural hemorrhage
	\end{itemize}
	\item
	vascular
	
	\begin{itemize}
		\tightlist
		\item
		cerebral aneurysms
		\item
		some vascular malformations
	\end{itemize}
\end{itemize}

\subsection{Brain}

The \textbf{brain} (TA: encephalon) is the vital neurological organ composed of:

\begin{itemize}
	\item
	cerebrum
	\item
	diencephalon
	\item
	brainstem
	
	\begin{itemize}
		\item
		midbrain
		\item
		pons
		\item
		medulla
	\end{itemize}
	\item
	cerebellum
\end{itemize}

The brain is housed in the neurocraniumof the skull and bathed in cerebrospinal fluid.It is continuous with the cervical spinal cord at the cervicomedullary junction.The brain and spinal cord combined form the central nervous system (CNS).


\subsection{Intraventricular}

\textbf{Intraventricular} is a term used to denote lesions/processes that occur within either the ventricles of the brain or the ventricles of the heart.

In both cases, most lesions actually arise from the surrounding brain parenchyma/heart muscle and grow exophytically into the ventricles.
