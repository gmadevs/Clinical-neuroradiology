\chapter{Metastases and paraneoplastic syndromes}

\subsection{Hemorrhagic intracranial metastases (mnemonic)}

A \textbf{mnemonic} for primary malignancies responsible for hemorrhagic intracranial metastasesis:

\begin{itemize}
	\item
	\textbf{MR CT BB}
	\item
	\textbf{MR CT HBO}
\end{itemize}


\paragraph{Mnemonic}


\subparagraph{MR CT BB}

\begin{itemize}
	\item
	\textbf{M:}melanoma
	\item
	\textbf{R:}renal cell carcinoma
\end{itemize}

\begin{itemize}
	\item
	\textbf{C:}choriocarcinoma
	\item
	\textbf{T:}thyroid carcinoma, teratoma
\end{itemize}

\begin{itemize}
	\item
	\textbf{B:}bronchogenic carcinoma
	\item
	\textbf{B:}breast carcinoma
\end{itemize}


\subparagraph{MR CT HBO}

\begin{itemize}
	\item
	\textbf{M:}melanoma
	\item
	\textbf{R:}renal cell carcinoma
\end{itemize}

\begin{itemize}
	\item
	\textbf{C:}choriocarcinoma
	\item
	\textbf{T:}thyroid carcinoma,teratoma
\end{itemize}

\begin{itemize}
	\item
	\textbf{H:} hepatocellular carcinoma, hepatoblastoma (rare)
	\item
	\textbf{B:}bronchogenic carcinoma, breast carcinoma
	\item
	\textbf{O:} osteogenic sarcoma (rare)
\end{itemize}
\subsection{Leptomeningeal metastases}

\textbf{Leptomeningeal metastases}, also known as \textbf{carcinomatous meningitis} and \textbf{meningeal carcinomatosis}, refers to the spread of malignant cells through the CSF space. These cells can originate from primary CNS tumors (e.g. in the form of drop metastases), as well as from distant tumors that have metastasized via hematogenous spread.

This article has a focus on subarachnoid space involvement. Refer to intradural extramedullary metastases for a discussion of leptomeningeal metastases in the spine. For other intracranial metastatic locations, please refer to the main article on intracranial metastases.

\paragraph{Epidemiology}

The demographics follow those of the underlying malignancy. Meningeal metastases were found in 8\% of patients with metastatic cancer in cadaveric studies .

\paragraph{Clinical presentation}

Clinical presentation is varied, but most commonly includes headache, encephalopathy, nausea and vomiting, and/or progressive multifocal neurological deficits (e.g. multiple cranial neuropathies and radiculopathies, myelopathy) . Meningism is only present in a minority of patients (13\% ).

\paragraph{Pathology}

The primary intracranial malignancies that may cause metastases to the subarachnoid space are:

\begin{itemize}
	\item
	glioblastoma (GBM)and anaplastic astrocytoma
	\item
	medulloblastoma
	\item
	sPNET
	\item
	ependymoma
	\item
	germinoma
	\item
	choroid plexus carcinoma
	\item
	pineoblastoma/pineocytoma 
\end{itemize}

The vast majority of leptomeningeal metastases occur in the context of widespread metastatic disease, likely by hematogenous spread. Over 50\% of cases have concurrent brain (parenchymal) metastases .The most common primary sites are:

\begin{itemize}
	\item
	breast cancer(particularly infiltrating lobular carcinoma)
	\item
	lung cancer
	\item
	melanoma
	\item
	gastrointestinal (e.g. gastric carcinoma,colorectal cancer)
	\item
	hematological:lymphoma/leukemia (leptomeningeal lymphomatosis)
\end{itemize}

Less common reported primary sites include:

\begin{itemize}
	\item
	pancreatic carcinoma 
	\item
	ovarian cancer 
	\item
	renal cell cancer 
	\item
	carcinoma of the uterine cervix 
	\item
	adrenal cortical carcinoma 
	\item
	esophageal cancer 
	\item
	urinary bladder adenocarcinoma 
	\item
	prostate cancer 
	\item
	malignant pleural mesothelioma 
	\item
	testicular cancer 
	\item
	intradural malignant peripheral nerve sheath tumor (MPNST) 
	\item
	anaplastic thyroid carcinoma 
	\item
	transitional cell carcinoma of the bladder 
	\item
	cholangiocarcinoma(extremely rare)
\end{itemize}

\paragraph{Radiographic features}

\subparagraph{MRI}

\begin{itemize}
	\item
	\textbf{T1:}usually normal
	\item
	\textbf{T1 C+ (Gd):}leptomeningeal enhancement is the primary mode of diagnosis,often scattered over the brain in a 'sugar coated' manner
	\item
	\textbf{T2:}usually normal;may show hyperintensity (including the bloomy rind sign in the brainstem )
	\item
	\textbf{FLAIR}
	
	\begin{itemize}
		\item
		abnormally elevated signal within sulci  and rarely within the parenchymal surface (including the bloomy rind sign in the brainstem )
		\item
		can be performed both non-contrast and post-contrast but is slightly less specific if performed post-contrast 
	\end{itemize}
\end{itemize}

\paragraph{Treatment and prognosis}

Leptomeningeal metastases have a poor prognosis with patients usually succumbing within a few months (median overall survival 2.4 months ). Treatment may extend survival to 6-10 months . Treatment can consist of :

\begin{itemize}
	\item
	intrathecal chemotherapy
	\item
	radiotherapy
\end{itemize}

Resection is usually inappropriate due to the presence of widespread metastases at the time of diagnosis.

\paragraph{Differential diagnosis}

\begin{itemize}
	\item
	leptomeningitis
	\item
	slow flow in vessels
	\item
	other causes of subarachnoid FLAIR hyperintensity
\end{itemize}