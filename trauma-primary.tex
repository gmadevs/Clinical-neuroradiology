\chapter{Primary effect of trauma}

\subsection{Basilar fractures of the skull}

\textbf{Basilar fractures of the skull}, also known as \textbf{base of skull} or \textbf{skull base fractures},are a common form of skull fracture, particularly in the setting of severe traumatic head injury, and involve the base of the skull. They may occur in isolation or often in continuity with skull vault (calvarial) fractures or facial fractures.

\paragraph{Epidemiology}

The majority of basilar fractures occur as a result of motor vehicle accidents, with sports injuries, falls and assault being other frequently encountered antecedents . Clearly, the relative incidence and demographics affected will vary widely depending on regional differences and mechanism.

\paragraph{Clinical presentation}

Skull base fractures are often encountered in the setting of severe head injury and thus the damage to the underlying brain and/or intracranial hemorrhage dominates the clinical presentation.

It is also rare to not obtain a CT of the brain in all such cases, however, historically a number of signs were described as being helpful in suggesting the presence of a base of skull fracture:

\begin{itemize}
	\item
	anterior cranial fossa fracture
	
	\begin{itemize}
		\item
		CSF rhinorrhea
		\item
		raccoon eyes sign
	\end{itemize}
	\item
	petrous temporal bone fracture
	
	\begin{itemize}
		\item
		Battle sign
		\item
		CSF otorrhea
		\item
		otorrhagia
	\end{itemize}
\end{itemize}


\paragraph{Pathology}

Fractures may either occur at the site of direct impact or remotely due to forces passing through the skull . As a general rule most base of skull fractures result from impact to the skull around its base (e.g. occiput, temporal region, frontal region -- the so-called "hat band" distribution). Less commonly, base of skull fractures are extensions of fractures that have occurred due to impact at the vertex .

The specific pattern of fracture and the associated complications (e.g. CSF leak, sensorineural hearing loss, cranial nerve palsies etc.) will depend on the location of the fracture. Generally, the direction of a fracture will be in line with the direction of impact (i.e. a transverse fracture will result from an impact on the side of the head) .

As is the case elsewhere, fractures may be linear, comminuted, depressed or compound.


\paragraph{Radiographic features}

CT is the investigation of choice and is able to identify most fractures. CT features are discussed in the skull fracture article.

Specific types of base of skull fractures include:

\begin{itemize}
	\item
	clival fracture
	\item
	petrous temporal bone fracture
	\item
	transsphenoidal basilar skull fracture
	\item
	occipital condyle fracture
\end{itemize}

Fractures that cross the dural sinuses and/or jugular bulb should go onto have a CT venogram to asses for evidence of a traumatic dural venous sinus thrombosis, which, is associated with this injury .

\subsection{Transsphenoidal basilar skull fracture}

\textbf{Transsphenoidal basilar skull fractures} are a particularly serious type of basilar skull fracture usually occurring in the setting of severe traumatic brain injury and with potential for serious complications including damaging the internal carotid arteries and optic nerves as well as high incidence of dural tear with CSF leak.Venous thrombosis complicates up to 31\% of these fractures; as many as 75\% of caroticocavernous fistulae will have antecedent skull base fractures.

\paragraph{Pathology}

Due to the particulars of the anatomy of the base of skull, fractures that involve the sphenoid sinus tend to extend along a number of predefined pathways :
\subparagraph{Anterior transverse}

\begin{itemize}
	\item
	impact: lateral in the region of the temple
	\item
	coronal fracture plane
	
	\begin{itemize}
		\item
		extending from the squamous temporal bone
		\item
		through the base of the anterior clinoid processes anterior to the pituitary fossa
		\item
		continuing laterally along the contralateral sphenotemporal buttress +/- into the squamous temporal bone
	\end{itemize}
	\item
	may extend inferiorly to involve the pterygoid processes
\end{itemize}

\subparagraph{Lateral frontal diagonal}

\begin{itemize}
	\item
	impact:lateral frontal/anterior malar eminence
	\item
	oblique fracture plane
	
	\begin{itemize}
		\item
		extending from lateral frontal/lateral orbital roof
		\item
		through the sphenoid sinus
		\item
		through or adjacent to the contralateral carotid canal into sphenopetrosal synchondrosis
		\item
		extends as a petrous temporal bone fracture
	\end{itemize}
	\item
	often associated with maxillary sinus fractures and lateral orbital wall
\end{itemize}


\subparagraph{Posterior transverse}

\begin{itemize}
	\item
	impact: lateral, just anterior to the external acoustic meatus
	\item
	U-shaped fracture comprised of bilateral longitudinal temporal bone fractures(or mixed) united in the midline by a fracture through the posterior wall of sphenoid/clivus
	\item
	involves sphenopetrosal synchondrosis, foramen lacerum and carotid canal
\end{itemize}


\subparagraph{Mastoid diagonal}

\begin{itemize}
	\item
	impact: posterolateral in the mastoid region
	\item
	oblique fracture
	
	\begin{itemize}
		\item
		originating in the occipital bone
		\item
		extending to the jugular foramen and petro-occipital fissure
		\item
		diagonally passing through sphenoid
		\item
		into contralateral ethmoid air cells or orbital roof
	\end{itemize}
\end{itemize}