\chapter{General embryology}

\subsection{Meninx primitiva}

The \textbf{meninx primitiva} refers to the collection of neural crest and mesenchyme (mesoderm) that surround the developing brain during gestation . This process occurs between the 3 and 5 week of gestation. The meninx primitiva will eventually differentiate into the arachnoid mater, pia mater and dura mater by the end of the first trimester .Both the dura mater (pachymeninges), and arachnoid and pia mater (leptomeninges) are largely mesodermal in origin, although apparently the innermost layer of the pia is neuroectodermal.

\paragraph{Related pathology}

\begin{itemize}
	\item
	intracranial lipomas arise from persistence and maldifferentiation of the meninx primitiva
\end{itemize}