\chapter{Introduction}

\subsection{Neuroradiology: interpretation (curriculum)}

\textbf{Neuroradiology interpretation}is a key component of how to make the most of diagnostic imaging. You need to know how to look at the commonly performed radiology tests and how to make common diagnoses.In reality, this allows basic image interpretation in CT of the head and a very basic understanding of MRI.

\textbf{Neuroradiology curriculum for medical students}is broadly split into content that refers to imaging (the test and findings) and conditions that are considered key for this stage of training.

\subsection{Psychoradiology}

\textbf{Psychoradiology} is an emerging field that applies medical imaging technologies to the analysis of mental health, neurophysiology and psychiatric conditions . Psychoradiology is not a recognized subspecialty for clinical practice and relies on imaging data analysis rather than visual inspection of images.

\paragraph{Development}

Imaging techniques of the brain and nervous system have improved in sophistication, sensitivity and definition over time. Previously in psychiatry, only gross abnormalities could be detected with either the naked eye or computed tomography (e.g. severe frontal lobe and traumatic brain injuries). Since CT of patients with schizophrenia identified bilateral ventricular enlargement in 1976,the volume of descriptions of structural abnormalities in mental illness has increased . Modern imaging technologies such as magnetic resonance imaging (MRI) and derivatives (functional MRIincluding resting state and task-based functional MRI, MR spectroscopy, perfusion mapping, the application of diffusion-tensor imaging and tractography)have given rise to an increasing body of scientific literature that elucidates how various mental states, conditions and psychiatric diseases affect physical structures, their activity and neural circuits in the brain across time and treatment .

\paragraph{Application}

Examples include (but are not limited to):

\begin{itemize}
	\item
	depression
	
	\begin{itemize}
		\item
		MRI data (binary pattern classification, voxel-based morphometry) to predict response to ECT for acute major depressive disorder 
		
		\begin{itemize}
			\item
			structural impairment in the subgenual cingulate cortex prior to therapy was positively correlated with response to ECT 
			\item
			patients receiving ECT demonstrated an increase in hippocampal volume 
		\end{itemize}
	\end{itemize}
	\item
	bipolar I disorder
	
	\begin{itemize}
		\item
		whole-brain voxel-based analysis with diffusion tensor imagingshowed white matter and structural abnormalities in the corpus callosum, tapetum, fornix and stria terminalis 
	\end{itemize}
	\item
	borderline personality disorder
	
	\begin{itemize}
		\item
		myriad and often contradictory neuroimaging findings 
		\item
		decreased white matter integrity in the cingulum and fornix
		\item
		association of anger with fractional anisotropy in the cingulum 
		\item
		association of affective instability and abandonment avoidance with fractional anisotropy in the fornix 
	\end{itemize}
	\item
	individual recognition and 'fluid intelligence' - "the capacity for on-the-spot reasoning to discern patterns and solve problems independently of acquired knowledge"
	
	\begin{itemize}
		\item
		functional connectivity MRI used frontoparietal networks to accurately identify individuals from a pool of 126 subjects.
		\item
		individual connectivity profiles predicted fluid intelligence cognitive behavior
	\end{itemize}
	\item
	schizophrenia
	
	\begin{itemize}
		\item
		voxel-based morphometric changes, most consistently in the left superior temporal gyrus and left medial temporal lobe 
		\item
		some studies have focused on premorbid, high-risk populations 
	\end{itemize}
	\item
	some authors have queried the possibility of connectomes (cortical connectivity networks)guiding the monitoring of childhood development in the future 
\end{itemize}


\paragraph{Influence on diagnostic constructs in psychiatry}

"...if our diagnostic categories have not been valid until now, then research of any type -- epidemiological, etiological, pathogenetic, therapeutic, biological, psychological or social -- if carried out with these diagnoses as inclusion criterion, is equally invalid."

- Prof Heinz Katschnig, World Psychiatry Association, 2010 

Imaging findings and post-mortem analyzes have contributed to the understanding of biological pathophysiology and the underlying neural mechanisms of mental illnesses . Some authors have suggested that imaging may change diagnostic structures in psychiatry .

Using a categorical approach for the diagnosis of mental illness (such as the Diagnostic and Statistical Manual of Mental Disorders (DSM) or International Classification of Diseases),the presence of various symptoms and symptom constructs either at one point in time or over time are organized into categories, for example major depressive disorder, minor depressive disorder etc. This has been compared to taxonomy in botany; classifying a plant based on observation of the types and number of leaves (rather than any underlying factor) . In practice, there are spectra of patients that do not necessarily fit into these clusters, with a given patient arguably suiting one DSM diagnosis over another, depending on the interpretation and weighting of individual symptoms, in terms of severity and their time-course.

A system of classifying and diagnosing psychopathology that reflects modern advances in neuroscience may assist in guiding research and treatment.Some psychiatry academics advocate moving from a categorical approach towards one that gives pathophysiological and objective findings more weighting . Such a system is proposed by the US National Institute of Mental Health for research purposes;the Research Domain Criteria (RDoC) .

With a growing wave of objective digital imaging and tractographic data, a multimodal template for research in psychiatry (such as the RDoC), including objective anomalies of circuitry and cortical volume and activity will be increasingly significant and very likely will eventually guide ground level diagnosis and treatment of mental illnesses.

\paragraph{Current and future research}

With modern databases and the exponential increase in IT processing power and speed, various quantities of tractographic and cortical information are able to be collated and compared. This already occurs on a scale of tens of thousands of individual patients , allowing comparison of control with condition states across age-groups, and the analysis of treatment effects (including pharmaceutical and non-pharmaceutical treatments). Capacity for large-scale analyzes will increase in the future with advances in computer science.

Large-scale analysis of cerebral regional connectivity (i.e. the 'connectome') for the purpose of identifying regional abnormalities implicated in psychiatric disorders is complicated by the variability and unique 'fingerprint' of an individual . This characteristic connectivity and variation may be influenced by genes, perinatal events, life experiences and socioeconomic aspects . The Human Connectome Project (HCP) is a consortium of universities overseen by the USA National Institutes of Health that strives to identify the neural pathway basis of human brain physiology.

The Human Connectome Project and other datasets, and software for their analysis, are open-source; and have been released into the public domain for use by the scientific community.