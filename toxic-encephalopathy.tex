\chapter{Toxic encephalopathy}

\subsection{Hypoglycemic encephalopathy}

\textbf{Hypoglycemicencephalopathy} is a brain injury that results from prolonged or severe hypoglycemia.

On imaging, it can manifest on MRI as bilateral areas of increased signal on both T2 and FLAIR affecting the posterior limb of the internal capsule,cerebral cortex (in particular parieto-occipital and insula),hippocampus and basal ganglia. Restricted diffusion can be an earlier and sensitive tool, and is commonly reversible.

\paragraph{Clinical presentation}

Severe symptoms of hypoglycemia are present, such as altered conscious state, loss of consciousness, seizures, etc.

\paragraph{Pathology}

The pathophysiology is uncertain, but altered cellular physiology results in neuronal death . It is known that hypoglycemia leads to cellular energy failure, as the brain is an obligate glucose metabolizer. The resulting energy shortage results in sodium/potassium pump failure and cellular swelling and tissue alkalosis . Some theories are based on cell damage due to increased extracellular aspartate and glutamate .

\subparagraph{Etiology}

Any cause of profound hypoglycemia :

\begin{itemize}
	\item
	overdose of hypoglycemic medication (usually in diabetics)
	\item
	pancreatic insulinoma
\end{itemize}

\paragraph{Radiographic features}

As hypoglycemia is usually recognized and managed promptly, MRI scans are not routinely performed unless there is a complicated recovery.

\subparagraph{MRI}

There are characteristic changes typically affecting the posterior limb of the internal capsule,cerebral cortex (in particular parieto-occipital and insula), hippocampus, and basal ganglia . These are typically bilateral.The cerebellum, brainstem and thalami are usually spared in adults but they are also involved in neonates . The splenium of the corpus callosum can also be affected, producing the so-called boomerang sign.

\begin{itemize}
	\item
	\textbf{T1:} low signal
	\item
	\textbf{T2:} high signal
	\item
	\textbf{DWI/ADC:} can be an earlier and sensitive tool showing reversible diffusion restriction 
\end{itemize}


\paragraph{Treatment and prognosis}

The clinical outcome has a direct relation with the severity and duration of the hypoglycemic insult.


\paragraph{Differential diagnosis}

\begin{itemize}
	\item
	hypoxic-ischemic brain injury: may show symmetrical thalamic lesions 
	\item
	Creutzfeldt-Jakob disease(CJD): different clinical presentation
	\item
	ischemic infarct: usually focal and unilateral
	\item
	seizure-related changes
\end{itemize}
\subsection{Lead poisoning}

\textbf{Lead poisoning} or \textbf{plumbism}refers to the multiorgan toxicity exerted by exposure to lead. Manifestations differ based on a myriad of features including chronicity, exposure intensity, and age. Neurologic toxicity and hematologic toxicity are common features. Clinical manifestations vary, ranging from mild (or asymptomatic) cases to a severe life-threatening encephalopathy .

\paragraph{Epidemiology}

Common sources of exposure to this commonly-found metal may be broadly classified as occupational (or recreational), environmental, or perinatal (primarily transplacental). Environmental sources may include :

\begin{itemize}
	\item
	lead-based paint
	
	\begin{itemize}
		\item
		structural renovation (e.g. sanding) or senescence (e.g. flaking) results in incorporation in soil and/or dust contamination
		\item
		most common source of pediatric exposure
	\end{itemize}
	\item
	contamination of food or water
	
	\begin{itemize}
		\item
		degradation of lead-containing pipes and solder in plumbing
		\item
		cans may contain lead solder
		\item
		contamination of illegally distilled alcohol ("moonshine"), cooking spices, traditional remedies
	\end{itemize}
\end{itemize}

Sources of occupational and recreational exposures include:

\begin{itemize}
	\item
	lead smelting, metal work (e.g. welding)
	\item
	automobile work (especially related to radiators)
	\item
	firearm ranges
	\item
	construction workers, painters
\end{itemize}

\paragraph{Clinical presentation}

Presenting features vary as a function of factors such as age (pediatric predisposition to more severe neurologic toxicity), ingested dose, and chronicity of exposure.

\begin{itemize}
	\item
	neurologic
	
	\begin{itemize}
		\item
		milder pediatric cases may manifest with disturbances in behavior, growth, hearing and cognition
		\item
		severe pediatric cases may progress to encephalopathy with seizures, coma, ataxia, cerebral edema
		\item
		adults may similarly demonstrate mild (tired, irritable) symptoms or a severe encephalopathy
		
		\begin{itemize}
			\item
			peripheral motor neuropathy may also develop
		\end{itemize}
	\end{itemize}
	\item
	gastrointestinal
	
	\begin{itemize}
		\item
		abdominal pain
		\item
		vomiting
		\item
		constipation
	\end{itemize}
	\item
	hematologic
	
	\begin{itemize}
		\item
		anemia
		
		\begin{itemize}
			\item
			hemolysis may also be observed
			\item
			elevations (chronic) in zinc and erythrocyte protoporphyrin
		\end{itemize}
	\end{itemize}
	\item
	reproductive
	
	\begin{itemize}
		\item
		miscarriages
		\item
		infertility
		
		\begin{itemize}
			\item
			may impair spermatic function and spermatogenesis
		\end{itemize}
	\end{itemize}
	\item
	renal
	\item
	cardiovascular
	
	\begin{itemize}
		\item
		hypertension
	\end{itemize}
	\item
	musculoskeletal
	
	\begin{itemize}
		\item
		gout
	\end{itemize}
\end{itemize}


\paragraph{Pathology}

\begin{itemize}
	\item
	bones
	
	\begin{itemize}
		\item
		bone remodeling and growth may be affected in pediatric cases, with proposed mechanisms including 
		
		\begin{itemize}
			\item
			alterations in circulating endocrine factors such as parathyroid hormone and activated vitamin D
			\item
			derangement in paracrine signaling factors such as osteocalcin
			\item
			direct cellular toxicity, particularly affecting osteoclasts
		\end{itemize}
		\item
		growth trajectory and height may be affected
		\item
		pathologically increased deposition of calcium in the zones of provisional calcification responsible for the dense metaphyseal bands on radiographs referred to as "lead lines"
	\end{itemize}
	\item
	anemiais multifactorial
	
	\begin{itemize}
		\item
		\hspace{0pt}multistep inhibition in heme synthesis
		
		\begin{itemize}
			\item
			including aminolevulinic acid dehydratase and ferrochelatase 
		\end{itemize}
		\item
		may impair production of erythropoietin 
		\item
		erythrocyte functional and structural derangements
		
		\begin{itemize}
			\item
			inhibition of erythrocyte sodium-potassium pump 
			
			\begin{itemize}
				\item
				predisposing to breakdown of the cell membrane and hemolysis
			\end{itemize}
			\item
			inhibition of pyrimidine 5' nucleotidase
			
			\begin{itemize}
				\item
				residual nucleotide clumps in cytosol appear as basophilic stippling
			\end{itemize}
		\end{itemize}
	\end{itemize}
\end{itemize}


\paragraph{Radiographic features}

\subparagraph{Plain radiograph}

\begin{itemize}
	\item
	may show bands of increased density at the metaphyses
	\item
	can affect any metaphysis, but the involvement of the proximal fibula and distal ulnar metaphyses is highly suggestive
	\item
	may show bone-in-bone appearance
	\item
	abdominal radiographs utile for identification of exposure source and anatomic location if an ingestion is suspected 
	
	\begin{itemize}
		\item
		intraluminal radiopaque foreign body (or multiple punctate densities)
		\item
		serial radiographs may be used to monitor effectiveness of bowel decontamination
	\end{itemize}
\end{itemize}

\paragraph{Treatment and prognosis}

\begin{itemize}
	\item
	identification of source crucial to prevent ongoing toxicity 
	\item
	meticulous supportive care
	\item
	decontamination of ongoing sources of enteral absorption (paint chips, foreign bodies)
	
	\begin{itemize}
		\item
		endoscopic retrieval
		\item
		whole bowel irrigation
	\end{itemize}
	\item
	some patients may require chelating agents to enhance elimination
	
	\begin{itemize}
		\item
		dimercaprol
		
		\begin{itemize}
			\item
			also known as "British anti-Lewisite" or BAL
		\end{itemize}
		\item
		dimercaptosuccinic acid (DMSA)
		
		\begin{itemize}
			\item
			also known as succimer, the favored available oral chelating agent over d-penicillamine
		\end{itemize}
		\item
		edetate calcium disodium (CaNa\textsubscript{2}EDTA)
	\end{itemize}
\end{itemize}

\paragraph{Differential diagnosis}

\begin{itemize}
	\item
	healed rickets
	\item
	physiological appearances in \textless3 years' age group
\end{itemize}


\paragraph{See also}

\begin{itemize}
	\item
	differential for dense metaphyseal bands
	\item
	heavy metals
\end{itemize}