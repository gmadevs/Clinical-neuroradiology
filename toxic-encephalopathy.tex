\chapter{Toxic encephalopathy}

\subsection{Hypoglycemic encephalopathy}

\textbf{Hypoglycemicencephalopathy} is a brain injury that results from prolonged or severe hypoglycemia.

On imaging, it can manifest on MRI as bilateral areas of increased signal on both T2 and FLAIR affecting the posterior limb of the internal capsule,cerebral cortex (in particular parieto-occipital and insula),hippocampus and basal ganglia. Restricted diffusion can be an earlier and sensitive tool, and is commonly reversible.

\paragraph{Clinical presentation}

Severe symptoms of hypoglycemia are present, such as altered conscious state, loss of consciousness, seizures, etc.

\paragraph{Pathology}

The pathophysiology is uncertain, but altered cellular physiology results in neuronal death . It is known that hypoglycemia leads to cellular energy failure, as the brain is an obligate glucose metabolizer. The resulting energy shortage results in sodium/potassium pump failure and cellular swelling and tissue alkalosis . Some theories are based on cell damage due to increased extracellular aspartate and glutamate .

\subparagraph{Etiology}

Any cause of profound hypoglycemia :

\begin{itemize}
	\item
	overdose of hypoglycemic medication (usually in diabetics)
	\item
	pancreatic insulinoma
\end{itemize}

\paragraph{Radiographic features}

As hypoglycemia is usually recognized and managed promptly, MRI scans are not routinely performed unless there is a complicated recovery.

\subparagraph{MRI}

There are characteristic changes typically affecting the posterior limb of the internal capsule,cerebral cortex (in particular parieto-occipital and insula), hippocampus, and basal ganglia . These are typically bilateral.The cerebellum, brainstem and thalami are usually spared in adults but they are also involved in neonates . The splenium of the corpus callosum can also be affected, producing the so-called boomerang sign.

\begin{itemize}
	\item
	\textbf{T1:} low signal
	\item
	\textbf{T2:} high signal
	\item
	\textbf{DWI/ADC:} can be an earlier and sensitive tool showing reversible diffusion restriction 
\end{itemize}


\paragraph{Treatment and prognosis}

The clinical outcome has a direct relation with the severity and duration of the hypoglycemic insult.


\paragraph{Differential diagnosis}

\begin{itemize}
	\item
	hypoxic-ischemic brain injury: may show symmetrical thalamic lesions 
	\item
	Creutzfeldt-Jakob disease(CJD): different clinical presentation
	\item
	ischemic infarct: usually focal and unilateral
	\item
	seizure-related changes
\end{itemize}